\documentclass[a4paper,12pt]{article}
% \input{../../../config.tex}
\usepackage{mypackages}
\usepackage{macros}

\begin{document}

\title{1er cours - Présentations}
\author{N. Bancel}

\sloppy  % This will apply the sloppy setting to the entire document.
\maketitle

\section{Programme journée}

\begin{itemize}
\item[$\bullet$] 9h50 à 10h40 : Maths STD2A 1ère STD2A
\item[$\bullet$] 10h55 à 11h45 : Physique-Chimie 3ème
\item[$\bullet$] 13h25 à 14h15 : Physique-Chimie 3ème
\item[$\bullet$] 15h20 à 17h00 : Physique-Chimie 1ère STD2A
\end{itemize}

\section{Agenda (1h)}


\begin{enumerate}
  \item Présentation perso
  \item Objectifs de l'année et règles du jeu
  \item Présentations de chacun (ce qu'ils aiment, s'ils aiment la matière, difficultés sur le programme de l'année dernière)
  \item Révision de l'année dernière
\end{enumerate}

\subsection{Présentation perso}
\begin{itemize}
  \item[$\bullet$] 34 ans - ancien développeur en informatique, et en science des données.
  \item[$\bullet$] 4 classes x matières à Suger : Maths x Physique STD2A. Physique 3ème. Projets scientifiques
\end{itemize}


\subsection{Objectifs de l'année et règles du jeu}
\begin{itemize}
  \item[$\bullet$] Petite classe : coach. Individualisation. Pas hésiter à venir me voir
  \item[$\bullet$] Beaucoup d'engagement de mon côté, attente du même du votre. Apprentissage assez ludique, on va essayer d'axer l'apprentissage sous l'angle du projet. \par 
  Motivation passe par le fait que compréhension de à quoi ça sert. Me dire ce qui les intéresse.
  \item[$\bullet$] Prêts pour les examens (Physique STD2A + Brevet)
  \item[$\bullet$] Dévelopement compétences psychosociales
  \begin{itemize}
    \item[] Compétences cognitives (conscience de soi, maîtrise de soi, prendre des décisions constructives)
    \item[] Compétences émotionnelles (conscience des émotions, réguler ses émotions, gérer son stress)
    \item[] Compétences sociales (communiquer de façon constructive, développer des relations constructives, résoudre des difficultés)
  \end{itemize}
  \item[$\bullet$] Fonctionnement : routine en début de cours. Devoirs maisons non notés : pour pas que envie de tricher, aller sur ChatGPT. Faire effort de pas utiliser ChatGPT 
  \item[$\bullet$] Demander de l'aide entre eux, me demander de l'aide
  \item[$\bullet$] Certains devoirs maisons volontairement abstraits, assez larges. Pour qu'ils apprennent à raisonner sur des problèmes un peu plus complexes.
  \item[$\bullet$] Aucune moquerie - je veux que tout le monde prenne la parole, participe. Donc pas de moquerie, pas de question bête.
\end{itemize}

\end{document}
