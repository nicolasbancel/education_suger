\documentclass{article}
\usepackage{graphicx}
\usepackage{amsmath}
\usepackage{xcolor}

\title{Utilisation de la calculatrice NumWorks pour vérifier ses résultats}
\author{N. Bancel}
\date{Mars 2025}

\begin{document}

\maketitle

\section{Introduction}
La calculatrice NumWorks est un outil puissant qui peut être utilisé pour vérifier les résultats obtenus lors des devoirs et des examens. Cette fiche vous propose des exercices pratiques pour apprendre à exploiter ses fonctionnalités en lien avec le DST N°4.

\section{Vérification des résultats avec NumWorks}

\subsection{Exercice 1 : Fonction dérivée}
\begin{itemize}
\item NumWorks ne permet pas directement de calculer une dérivée avec des x, mais elle permet d'évaluer une dérivée en des points donnés, et donc de vérifier des résultats
\item 
\end{itemize}

\subsection{Exercice 2 : Analyse - Questions de cours}
\begin{itemize}
\item Dans l'application \textbf{Calculs}, testez différentes fonctions et utilisez la touche \textbf{=} pour observer leurs valeurs.
\item Pour la dérivation, ouvrez l'application \textbf{Calculs}, tapez une fonction et utilisez la commande \texttt{diff(f(x), x)} pour obtenir sa dérivée.
\end{itemize}

\subsection{Exercice 3 : Équations de droites}
\begin{itemize}
\item Dans l'application \textbf{Graphique}, entrez l'équation de la droite sous la forme $y = ax + b$ et vérifiez si elle correspond bien au tracé demandé.
\item Testez l'intersection de deux droites en les entrant dans l'application \textbf{Graphique} et en utilisant la fonction \textbf{intersection}.
\end{itemize}

\subsection{Exercice 4 : Étude des variations d'un polynôme}
\begin{itemize}
\item Dans l'application \textbf{Graphique}, entrez la fonction $f(x) = x^2 + 2x - 3$ et observez son graphique.
\item Utilisez la touche \textbf{f'(x)} pour afficher la dérivée et déterminer les variations de la fonction.
\item Vérifiez le tableau de signes en observant où la dérivée s'annule et où la fonction est croissante ou décroissante.
\end{itemize}

\section{Conclusion}
La calculatrice NumWorks est un excellent outil pour valider vos résultats. En utilisant les applications \textbf{Calculs}, \textbf{Graphique} et \textbf{Python}, vous pouvez gagner en autonomie et en précision dans vos résolutions d'exercices.

\end{document}
