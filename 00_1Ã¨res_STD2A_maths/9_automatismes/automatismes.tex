\documentclass[answers]{exam}
\usepackage{../../mypackages}
\usepackage{../../macros}

\SolutionEmphasis{\color{blue}}
\renewcommand{\solutiontitle}{\noindent}

\title{Mini-cours - Automatismes : Pourcentages et Taux d'évolution}

\begin{document}

\textbf{Collège Lycée Suger}
\hfill
\textbf{Mathématiques} \\

\textbf{Année 2024-2025 - 3ème trimestre}
\hfill
\textbf{Mini-cours} \par

{\let\newpage\relax\maketitle}

\section*{Les pourcentages et les coefficients multiplicateurs}


\paragraph{Définitions}
\begin{itemize}
    \item \textbf{Augmenter} une valeur de $t\,\%$ revient à la multiplier par $\boxed{1 + \dfrac{t}{100}}$.
    \item \textbf{Diminuer} une valeur de $t\,\%$ revient à la multiplier par $\boxed{1 - \dfrac{t}{100}}$.
\end{itemize}
Les nombres $1 + \dfrac{t}{100}$ et $1 - \dfrac{t}{100}$ sont appelés des \textbf{coefficients multiplicateurs}.

\paragraph{Exemples}
\begin{itemize}
    \item Un prix de 80€ augmente de $15\,\%$ : \\
    $80 \times 1.15 = \boxed{92\text{€}}$
    \item Ce même prix diminue de $15\,\%$ : \\
    $80 \times 0.85 = \boxed{68\text{€}}$
\end{itemize}

\section*{Le taux d'évolution}

\paragraph{Définition}
Le \textbf{taux d'évolution} entre une valeur initiale $V_0$ et une valeur finale $V_1$ est :
\[
t = \dfrac{V_1 - V_0}{V_0}, \quad \text{et en pourcentage : } \quad t\% = 100 \times \dfrac{V_1 - V_0}{V_0}
\]

\paragraph{Exemple}
La population passe de 8500 à 10400 habitants. \\
\[
t = \dfrac{10400 - 8500}{8500} \approx \boxed{0.224} \quad \Rightarrow \quad \boxed{22.4\,\%}
\]

\section*{Évolutions successives}

\paragraph{Propriété}
Le coefficient multiplicateur global d’évolutions successives est le produit des coefficients de chaque étape.

\paragraph{Exemple}
Une entreprise augmente ses ventes de $10\,\%$ puis les diminue de $5\,\%$ : \\
\[
C = 1.10 \times 0.95 = 1.045 \Rightarrow \text{augmentation globale de } \boxed{4.5\,\%}
\]

\section*{Évolution réciproque}

\paragraph{Définition}
L'évolution réciproque est celle qui permet de retrouver la valeur initiale après une évolution.

\paragraph{Propriété}
Le coefficient de l’évolution réciproque est l’inverse du coefficient initial :\\
Si $V_1 = V_0 \times (1 + \dfrac{t}{100})$, alors $V_0 = V_1 \times \dfrac{1}{1 + \dfrac{t}{100}}$.

\paragraph{Exemple}
Une baisse de $8\,\%$ donne $0.92$. L’évolution inverse est : \\
$\dfrac{1}{0.92} \approx 1.087 \Rightarrow$ il faut une hausse d’environ $\boxed{8.7\,\%}$.

\end{document}
