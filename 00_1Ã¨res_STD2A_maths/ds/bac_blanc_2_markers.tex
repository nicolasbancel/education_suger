\documentclass[answers]{exam}
\usepackage{../../mypackages}
\usepackage{../../macros}


\SolutionEmphasis{\color{blue}}
\renewcommand{\solutiontitle}{\noindent}


\title{BAC Blanc}
\author{N. Bancel}
\date{30 Avril 2025}

\begin{document}


\begin{figure}[H]
  \centering
  \includegraphics[width=0.4\linewidth]{img/bac/0.jpg}
\end{figure}

\vspace{1em}
\vspace{1em}

\textbf{BACCALAUREAT 3 - 1ères STD2A}

\textbf{Date de l'épreuve : } \textcolor{blue}{30 Avril 2025} \\
\textbf{Matière: } \textcolor{blue}{Mathématiques} \\
\textbf{Durée de l'épreuve : } \textcolor{blue}{2 heures} \\
\textbf{Classe : } \textcolor{blue}{1ère STD2A} \\
\textbf{Nom de l'enseignant : } \textcolor{blue}{Nicolas Bancel} \\
\textbf{L'usage de la calculatrice en mode examen est autorisée : } \textcolor{blue}{OUI} \\

\newpage

\textbf{Collège Lycée Suger}
\hfill
\textbf{Mathématiques} \\

\textbf{Année 2024-2025 - 3ème trimestre}
\hfill
\textbf{1ères STD2A} \par

{\let\newpage\relax\maketitle}

\begin{center}
\textbf{\textcolor{red}{Durée : 2 heures. La calculatrice en mode examen est autorisée}} \\
\textbf{\textcolor{red}{Une réponse donnée sans justification sera considérée comme fausse.}} \\
\textbf{Total sur 23 points (la note sera ramenée sur 20)}
\end{center}

\section*{Exercice 1 : L'arrosoir (16 points)}

\textit{Extrait du BAC Juin 2018 Métropole - La réunion}

\vspace{1em}

Un bureau de design doit créer un arrosoir pour une célèbre enseigne. Le profil de l'arrosoir est composé de six éléments géométriques. Le plan étant muni d'un repère orthonormal
$\left(O\mathpunct{} ; \ \overrightarrow{\imath}\mathpunct{},\ \overrightarrow{\jmath}\right)\mathpunct{}$, les six éléments géométriques qui composent l'arrosoir sont représentés en gras
dans la figure ci-dessous.

\begin{figure}[H]
  \centering
  \includegraphics[width=0.8\linewidth]{img/bac/1.jpg}
\end{figure}

\begin{compactitem}
  \item L'ensemble constitué du segment \([OA]\) et de l'arc de cercle de
        centre \(\Omega\) reliant les points \(A\) et \(B\) représente
        le \emph{bec verseur} de l'arrosoir ;
  \item la courbe reliant les points \(O\) et \(E\) représente le
        \emph{col} de l'arrosoir ;
  \item l'arc d'ellipse reliant les points \(C\) et \(D\) représente
        l'\emph{anse} de l'arrosoir.
\end{compactitem}

Dans cet exercice, on étudie le bec verseur, le col et l'anse de l'arrosoir.

\subsection*{Partie A : Etude du bec verseur (2.5 points)}

On admet que le point $A$ a pour coordonées $A(-3.2 ; 2.4)$

\begin{questions}
  \question[0.5] Déterminer les coordonées des points $O$ et $\Omega$ 
  
%Q1E1%
\question[1] Déterminer les coordonées des vecteurs $\overrightarrow{OA}$ et $\overrightarrow{\Omega A}$
  
%Q2E1%
\question[1] En utilisant la formule en annexe, démontrer que les droites $(OA)$ et $(\Omega A)$ sont perpendiculaires.
\end{questions}


%Q3E1%
\subsection*{Partie B : Etude du col de l'arrosoir (13.5 points)}

La courbe qui représente le col de l'arrosoir est un arc de la courbe
représentative d'une fonction polynôme $f$ de degré 3 définie, pour tout
nombre réel $x$, par

\[
  f(x)=-0.1x^{3}+0.6x^{2}+ax+b ,
\]

où $a$ et $b$ sont des nombres réels à déterminer.
On appelle $F$ la courbe représentative de la fonction $f$.

\begin{compactenum}
  \item \textbf{Contrainte 1 :} le point $O$ appartient à la courbe $F$ ;
  \item \textbf{Contrainte 2 :} la droite $(OA)$ est tangente à la courbe $F$ au point $O$.
\end{compactenum}


\begin{questions}

  \question[0.5] Montrer que $b = 0$.
  
  
%Q4E1%
\question[1] Déterminer l’expression de $f'(x)$, dérivée de $f$.
  
  
%Q5E1%
\question[2.5] Détermination de la valeur de $a$
  \begin{parts}
    \part[2] Démontrer que la droite $(OA)$ a pour équation $y = -0,75x$.
    \part[0.5] En utilisant la formule de la dérivée, en évaluant sa valeur en 0, et en se souvenant de son interprétation géométrique, en déduire la valeur de $a$.
  \end{parts}
  
%Q6E1%
\question[5.5] On admet pour la suite que
  
  \[
  f(x) = -0.1x^{3} + 0.6x^{2} - 0.75x
  \]
  
  et que l’arc de $F$ correspondant au col est défini pour $x \in [0\,;4]$.
  
  \begin{parts}
    \part[1] Déterminer la dérivée de $f$ notée $f'(x)$
    \part[1.5] Soient $x_1$ et $x_2$ les deux valeurs : 

    $$
x_1 \;=\; \frac{4-\sqrt{6}}{2} \;\approx\; 0{,}78,
\qquad
x_2 \;=\; \frac{4+\sqrt{6}}{2} \;\approx\; 3{,}22.
$$

Démontrer que $x_1$ est une racine de la dérivée $f'(x)$ (c'est-à-dire que $f'(x_1)$ = 0). On admettra que $f'(x_2) = 0$
\part[1] En déduire que $f'(x)$ peut s'écrire sous la forme 

$$
f'(x)
  \;=\;
  -\frac{3}{10} \Bigl(x-\frac{4-\sqrt{6}}{2}\Bigr) 
  \Bigl(x-\frac{4+\sqrt{6}}{2}\Bigr).
$$

\part[2] Étudier le signe de $f'(x)$ sur l’intervalle $[0;4]$ et en déduire le tableau de variations de $f$.
\end{parts}

%Q7E1%
\question[2] Compléter le tableau de valeurs de l’annexe 1 (arrondir au dixième).

%Q8E1%
\question[2] Placer les points obtenus dans le repère de l’annexe 1 puis tracer l’arc de $F$ représentant le col.
  
\end{questions}


%Q9E1%
\section*{Exercice 2 - QCM et questions courtes (3 points)}

\emph{Pour chaque question, déterminer la bonne réponse et justifier pourquoi}

\begin{questions}
  \question[1] La courbe représentative d’une fonction $f$ définie et dérivable sur l’intervalle $[-2;4]$ est donnée ci-dessous.

\begin{figure}[H]
  \centering
  \includegraphics[width=0.6\linewidth]{img/bac/4.jpg}
\end{figure}

\begin{enumerate}
  \item $f'(x) \ge 0 \text{ sur } [-2;1]$
  \item $f'(x) > 0$
  \item $f'(x) < 0 \text{ sur } [-2;2]$
  \item $f'(3) > 0$
\end{enumerate}

  
%Q1E2%
\question[1] On considère la suite $u_n$ définie par
  \[
    \begin{cases}
      u_0 = 9, \\[4pt]
      u_n = 4\,u_{n-1} + 2 
    \end{cases}
  \]
  Déterminer la valeur de $u_2$


%Q2E2%
\question[1] Lire graphiquement
\begin{compactenum}
\item l'image de 3
\item l'image de -1
\item le ou les antécédents de 5
\item le ou les antécédents de -4
\end{compactenum}
par la fonction $f$ représentée graphiquement ci-dessous.

\begin{figure}[H]
  \centering
  \includegraphics[width=0.6\linewidth]{img/bac/7.jpg}
\end{figure}

\end{questions}


%Q3E2%
\section*{Exercice 3 - Pavage (4 points)}

\textit{Extrait du BAC Septembre 2018 Métropole - La réunion}

\vspace{1em}

Le pavage représenté ci-dessous à gauche a été réalisé à l’aide des deux motifs "hexagone" et "étoile" représentés ci-dessous à droite.

\begin{figure}[H]
  \centering
  \includegraphics[width=\linewidth]{img/bac/5.jpg}
\end{figure}

\subsection*{Partie A : Les motifs - Rappels de collège (1 point)}

\textit{Rappel : un angle plat mesure 180 degrés. La somme des mesures des angles d'un triangle est de 180 degrés}

\begin{questions}
  \question[1] L'hexagone $ABCDEF$ est constitué d’un rectangle $ABDE$ tel que $AB=\SI{4}{\centi\metre}$ et $BD=\SI{2}{\centi\metre}$.
  Les triangles $AEF$ et $BCD$ sont équilatéraux et situés à l’extérieur du rectangle $ABDE$. Déterminer la mesure, en degrés, de l’angle $\widehat{CDE}$.
\end{questions}


%Q1E3%
\subsection*{Partie B : Etude du pavage (3 points)}

\begin{questions}
  \question[1] Donner une transformation du plan qui permet de passer du motif "hexagone" numéroté 1 au motif "hexagone" numéroté 2.
  
%Q2E3%
\question[2] On peut passer du motif "hexagone" numéroté 1 au motif "hexagone" numéroté 3 en appliquant successivement deux transformations du plan. Quelles sont ces transformations ?
\end{questions}



%Q3E3%
\section*{Annexe}

\subsection*{Exercice 1 - Partie A - Question 3}

Deux vecteurs $\overrightarrow{AB}$ et $\overrightarrow{CD}$ de coordonnées \\ 

\[
  \overrightarrow{AB} = 
  \begin{pmatrix}
    x_{AB} \\
    y_{AB} \\ 
    z_{AB}
  \end{pmatrix}
\] et
\[
  \overrightarrow{CD} = 
  \begin{pmatrix}
    x_{CD} \\
    y_{CD} \\ 
    z_{CD}
  \end{pmatrix}
\]
sont orthogonaux (cad qu'ils sont perpendiculaires) si et seulement si leur produit scalaire est égal à 0, c'est-à-dire : 
\[
  x_{AB} \cdot x_{CD} + y_{AB} \cdot y_{CD} + z_{AB} \cdot z_{CD} = 0
\]

\subsection*{Exercice 1 - Partie B - Question 5 (tableau de valeurs)}

\begin{figure}[H]
  \centering
  \includegraphics[width=\linewidth]{img/bac/2.jpg}
\end{figure}

\subsection*{Exercice 1 - Partie B - Question 6 (Tracé de la courbe)}

\begin{figure}[H]
  \centering
  \includegraphics[width=\linewidth]{img/bac/3.jpg}
\end{figure}

\subsection*{Exercice 3 - Partie B}

\begin{figure}[H]
  \centering
  \includegraphics[width=\linewidth]{img/bac/6.jpg}
\end{figure}

\end{document}
