\documentclass[answers]{exam}
\usepackage{../../mypackages}
\usepackage{../../macros}


\SolutionEmphasis{\color{blue}}
\renewcommand{\solutiontitle}{\noindent}

\title{Chapitre 7 - Statistiques \\
\vspace{1em}
\large Correction d'exercices}
\author{N. Bancel}
\date{10 Avril 2025}

\begin{document}

\textbf{Collège Lycée Suger}
\hfill
\textbf{Mathématiques} \\

\textbf{Année 2024-2025 - 3ème trimestre}
\hfill
\textbf{1ère STD2A} \par

{\let\newpage\relax\maketitle}

\section*{Exercice 4}

\begin{figure}[H]
  \centering
  \includegraphics[width=0.7\linewidth]{img/exo_01.jpg}
  \captionsetup{labelformat=empty}
\end{figure}

\begin{solution}
  
\section*{1. Interprétation des événements}

\begin{itemize}
    \item $\overline{A}$ : l'élève n'est \textbf{pas une fille}, donc c’est un garçon.
    \item $A \cup B$ : l'élève est \textbf{une fille} \textcolor{red}{OU} \textbf{intéressé par les cours de programmation} (ou les deux).
    \item $A \cap \overline{B}$ : l'élève est \textbf{une fille} \textcolor{red}{(ET ELLE EST)} \textbf{non intéressée} par les cours de programmation.
\end{itemize}

\section*{2. Cardinaux associés}

\begin{itemize}
    \item $\mathbf{card}(\overline{A}) = 20$ (il y a 40 élèves dont 20 filles, donc 20 garçons)
    \item $\mathbf{card}(A \cup B)$ : on utilise la formule :
    \[
      \mathbf{card}(A \cup B )= \mathbf{card}(A) + \mathbf{card}(B) - \mathbf{card}(A \cap B) = 20 + 30 - 15 = 35
    \]
    \item Sinon on peut compter : (soit on est intéressé par la programmation (il y a 30 élèves), soit on est une fille (il y a déjà 15 filles qui ont été comptées dans le groupe de ceux qui aiment la programmation. Il en reste 5 qui n'ont pas été comptées car elles n'aiment pas la programmation. Il y a donc 35 personnes au total qui remplissent le critère)
    \item $\mathbf{card}(A \cap \overline{B})$ : filles non intéressées = $20 - 15 = 5$
\end{itemize}
\end{solution}


\section*{Exercice 26}

\begin{figure}[H]
  \centering
  \includegraphics[width=0.7\linewidth]{img/exo_02.jpg}
  \captionsetup{labelformat=empty}
\end{figure}

\begin{solution}

  \begin{center}
    %\renewcommand{\arraystretch}{1.5}
    \begin{tabular}{>{\bfseries}lccc}
    \toprule
                             & CB     & Espèces & Total \\
    \midrule
    Alimentaires             & 150    & 50      & 200   \\
    Non alimentaires         & 80     & 20      & 100   \\
    \midrule
    Total                    & 230    & 70      & 300   \\
    \bottomrule
    \end{tabular}
    \end{center}

    \subsection*{1. Fréquences marginales (rapport à l'effectif total)}

\[
\text{Total} = 300
\]

On divise chaque valeur du tableau par l'effectif total, soit 300. Cela donne les fréquences marginales :

\begin{center}
  \renewcommand{\arraystretch}{2}
  \begin{tabular}{>{\bfseries}lccc}
  \toprule
   & CB & Espèces & Total \\
  \midrule
  Alimentaires & $\dfrac{150}{300} = 0.5$ & $\dfrac{50}{300} = 0.167$ & $\dfrac{200}{300} = 0.667$ \\
  Non alimentaires & $\dfrac{80}{300} = 0.267$ & $\dfrac{20}{300} = 0.067$ & $\dfrac{100}{300} = 0.333$ \\
  \midrule
  Total & $\dfrac{230}{300} = 0.767$ & $\dfrac{70}{300} = 0.233$ & $1$ \\
  \bottomrule
  \end{tabular}
  \end{center}


  \subsection*{2. Fréquences conditionnelles par lignes}

  On divise chaque valeur d'une ligne par le total de cette ligne. Cela permet de connaître, pour un type de produit donné, la répartition des modes de paiement.
  
  \begin{itemize}
      \item Pour les produits alimentaires (total : 200) :
      \[
      \text{CB : } \frac{150}{200} = 0.75 \quad \text{Espèces : } \frac{50}{200} = 0.25
      \]
      \item Pour les produits non alimentaires (total : 100) :
      \[
      \text{CB : } \frac{80}{100} = 0.80 \quad \text{Espèces : } \frac{20}{100} = 0.20
      \]
  \end{itemize}
  
  \begin{center}
  \begin{tabular}{>{\bfseries}lcc}
  \toprule
   & CB & Espèces \\
  \midrule
  Alimentaires & 0.75 & 0.25 \\
  Non alimentaires & 0.80 & 0.20 \\
  \bottomrule
  \end{tabular}
  \end{center}
  
  \vspace{1em}
  
  \subsection*{3. Fréquences conditionnelles par colonnes}
  
  On divise chaque valeur d'une colonne par le total de cette colonne. Cela permet de connaître, pour un mode de paiement donné, la répartition selon le type de produit.
\end{solution}

\newpage
\clearpage

\begin{solution}  

  \begin{itemize}
      \item Paiement par carte bancaire (total : 230) :
      \[
      \text{Alimentaires : } \frac{150}{230} \approx 0.652 \quad \text{Non alimentaires : } \frac{80}{230} \approx 0.348
      \]
      \item Paiement en espèces (total : 70) :
      \[
      \text{Alimentaires : } \frac{50}{70} \approx 0.714 \quad \text{Non alimentaires : } \frac{20}{70} \approx 0.286
      \]
  \end{itemize}
  
  \begin{center}
  \begin{tabular}{>{\bfseries}lcc}
  \toprule
   & CB & Espèces \\
  \midrule
  Alimentaires & 0.652 & 0.714 \\
  Non alimentaires & 0.348 & 0.286 \\
  \bottomrule
  \end{tabular}
  \end{center}

\end{solution}

\end{document}
