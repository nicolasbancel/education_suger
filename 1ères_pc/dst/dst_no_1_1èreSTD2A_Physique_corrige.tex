\documentclass{exam}
\usepackage{../../mypackages}
\usepackage{../../macros}
\usepackage{siunitx}
\usepackage{amsmath}

\setlength{\parindent}{0pt}

\title{Corrigé du Devoir sur table N°1 \par 
Généralités sur les matériaux \& L'atome}
\author{N. Bancel}
\date{9 Octobre 2024}

\begin{document}

\textbf{Collège Lycée Suger}
\hfill
\textbf{Physique-Chimie} \\

\textbf{Année 2024-2025}
\hfill
\textbf{1ère STD2A} \par

{\let\newpage\relax\maketitle}

\section*{Corrigé}

\section*{Exercice 1 [4 points] Les matériaux - Cours}

\begin{questions}

\question[1] \textbf{Lister les 3 grandes familles de matériaux, et en donner une définition rapide.}

Les 3 grandes familles de matériaux sont les suivantes :
\begin{itemize}
    \item \textbf{Matériaux métalliques} : Ce sont des éléments ou des alliages de métaux. Ils sont généralement ductiles, malléables et bons conducteurs de chaleur et d'électricité.
    \item \textbf{Matériaux organiques} : Ce sont des matériaux issus de matières naturelles ou synthétiques, souvent à base de carbone. Exemples : bois, plastique, cuir.
    \item \textbf{Matériaux minéraux (ou céramiques)} : Ce sont des matériaux inorganiques, souvent obtenus par des processus de cuisson ou de transformation thermique. Ils sont durs et résistants à haute température mais souvent cassants. Exemples : verre, céramique, béton.
\end{itemize}

\question[1] \textbf{Dresser un tableau des 3 familles de matériaux et classer les matériaux proposés.}

\begin{tabbing}
    \hspace*{3cm}\=\hspace*{4cm}\=\kill
    \textbf{Métalliques} \> \textbf{Organiques} \> \textbf{Minéraux} \\
    Cuivre \> Cuir \> Diamant \\
    Bronze \> Laine \> Porcelaine \\
    Fer \> Bois \> Verre \\
    \>\> Sable \\
    \> Matières plastiques \> \\
    \> Coton \> \\
\end{tabbing}

\question[1] \textbf{Quelle est la différence entre un métal pur et un alliage ? Quelle est la composition du bronze ?}

Un \textbf{métal pur} est constitué d'un seul élément chimique métallique (ex. : cuivre, fer), tandis qu'un \textbf{alliage} est un mélange de plusieurs éléments, dont au moins un métal, afin d'améliorer certaines propriétés. \\
Le \textbf{bronze} est un alliage composé principalement de \textbf{cuivre} (environ 88\%) et d'\textbf{étain} (environ 12\%).

\question[1] \textbf{Citer au moins 4 propriétés de matériaux et préciser si elles sont chimiques, physiques ou mécaniques.}

\begin{itemize}
    \item \textbf{Conductivité électrique (physique)} : Capacité d'un matériau à conduire le courant électrique.
    \item \textbf{Dureté (mécanique)} : Capacité d'un matériau à résister à la déformation ou à l'indentation.
    \item \textbf{Résistance à la corrosion (chimique)} : Aptitude à résister aux attaques chimiques (ex. : oxydation).
    \item \textbf{Densité (physique)} : Masse volumique d'un matériau.
    \item \textbf{Résistance à la traction (mécanique)} : Capacité d'un matériau à résister aux forces qui tendent à l'étirer.
\end{itemize}

\end{questions}

\section*{Exercice 2 [5 points] Structural Stripes}

\begin{questions}

\question[1] \textbf{Qu'est-ce qu'un matériau composite et de quoi est-il constitué ?}

Un \textbf{matériau composite} est un matériau composé de plusieurs composants distincts, qui sont généralement :
\begin{itemize}
    \item \textbf{La matrice} : Elle lie les composants et répartit les forces.
    \item \textbf{Le renfort} : Il confère au matériau ses propriétés mécaniques (ex. : fibres de verre ou de carbone).
\end{itemize}

\question[0.5] \textbf{À quelle catégorie de matériau appartiennent la fibre de verre et la fibre de carbone ? Justifier.}

La fibre de verre et la fibre de carbone sont des \textbf{matériaux composites}, car elles sont utilisées comme renforts dans des matrices polymères, formant un ensemble qui combine légèreté et résistance.

\question[0.5] \textbf{À quelle catégorie de matériau appartient "Structural Stripes" ? Justifier.}

"Structural Stripes" appartient à la catégorie des \textbf{métaux}, plus précisément de l'aluminium, car il est mentionné que l'œuvre utilise des couches de pièces d'aluminium pour former une structure autoportante.

\question[1] \textbf{Quels sont les avantages de "Structural Stripes" comparativement à la fibre de verre ou de carbone ?}

Les avantages de "Structural Stripes" par rapport à la fibre de verre ou de carbone sont :
\begin{itemize}
    \item \textbf{Recyclabilité} : L'aluminium est entièrement recyclable, ce qui n'est pas le cas des composites.
    \item \textbf{Résistance au feu} : L'aluminium ne brûle pas contrairement aux polymères qui peuvent fondre ou s'enflammer.
    \item \textbf{Ductilité} : L'aluminium est ductile et peut être façonné sans casser.
\end{itemize}

\question[2] \textbf{Le volume total de l'œuvre de Marc Fornes étant de \SI{12}{m^3}, calculer le poids de la structure.}

\begin{itemize}
    \item La \textbf{masse volumique de l'aluminium} est de \SI{2700}{kg/m^3}.
    \item Le volume de l'œuvre est de \SI{12}{m^3}.
\end{itemize}

\[
\text{Masse} = \text{volume} \times \text{masse volumique} = 12 \, \text{m}^3 \times 2700 \, \text{kg/m}^3 = 32\,400 \, \text{kg}.
\]

Le poids total de la structure est donc de \SI{32.4}{tonnes}.

\end{questions}

\section*{Exercice 3 [8 points] L'atome}

\begin{questions}

\question[1] \textbf{Donner la définition de (1) la règle du duet et de l'octet et (2) celle d'un gaz noble.}

\begin{itemize}
    \item \textbf{Règle du duet et de l'octet} : Un atome tend à avoir 2 (pour les plus légers) ou 8 électrons sur sa couche de valence pour être stable.
    \item \textbf{Gaz noble} : Ce sont des éléments du groupe 18 du tableau périodique, dont la couche de valence est complète, les rendant chimiquement inertes (ex. : Argon, Hélium).
\end{itemize}

\question[4] \textbf{Pour chaque atome, donner sa configuration électronique, couche de valence, et schéma de Lewis.}

\begin{itemize}
    \item \textbf{Carbone} (\ce{C}, Z = 6) :\\
    Configuration électronique : \( 1s^2 2s^2 2p^2 \)\\
    Couche de valence : \( n = 2 \), électrons de valence : 4\\
    Schéma de Lewis : \( \cdot \text{C} \cdot \)

    \item \textbf{Argon} (\ce{Ar}, Z = 18) :\\
    Configuration électronique : \( 1s^2 2s^2 2p^6 3s^2 3p^6 \)\\
    Couche de valence : \( n = 3 \), électrons de valence : 8\\
    Schéma de Lewis : \( :\text{Ar}: \)

    \item \textbf{Azote} (\ce{N}, Z = 7) :\\
    Configuration électronique : \( 1s^2 2s^2 2p^3 \)\\
    Couche de valence : \( n = 2 \), électrons de valence : 5\\
    Schéma de Lewis : \( \cdot \text{N} \cdot \)

    \item \textbf{Hydrogène} (\ce{H}, Z = 1) :\\
    Configuration électronique : \( 1s^1 \)\\
    Couche de valence : \( n = 1 \), électrons de valence : 1\\
    Schéma de Lewis : \( \text{H} \cdot \)

    \item \textbf{Oxygène} (\ce{O}, Z = 8) :\\
    Configuration électronique : \( 1s^2 2s^2 2p^4 \)\\
    Couche de valence : \( n = 2 \), électrons de valence : 6\\
    Schéma de Lewis : \( : \text{O} \cdot \)
\end{itemize}

\question[3] \textbf{Représentation de Lewis des molécules suivantes.}

\begin{itemize}
    \item \textbf{Eau} (\ce{H2O}) : \( \ce{H - O - H} \)
    \item \textbf{Méthane} (\ce{CH4}) : \( \ce{H - C - H} \) (avec 4 hydrogènes autour du carbone)
    \item \textbf{Ammoniac} (\ce{NH3}) : \( \ce{H - N - H} \) (avec un doublet non liant sur l'azote)
    \item \textbf{Dioxyde de carbone} (\ce{CO2}) : \( \ce{O = C = O} \)
\end{itemize}

\end{questions}

\section*{Exercice 4 [3 points] Masse volumique}

\begin{questions}

\question[2] \textbf{L'entreprise peut-elle transporter tout le sable dans un camion benne de \SI{21}{m^3} ? Pourquoi ?}

\begin{itemize}
    \item Masse volumique du sable : \SI{1850}{kg/m^3}.
    \item Masse de sable nécessaire : \SI{50}{tonnes} = \SI{50000}{kg}.
\end{itemize}

Calcul du volume de sable nécessaire :
\[
V = \frac{m}{\rho} = \frac{50000}{1850} \approx \SI{27.03}{m^3}.
\]

Le volume de sable nécessaire est \SI{27.03}{m^3}, supérieur à la capacité du camion (\SI{21}{m^3}). Donc, l'entreprise ne peut pas transporter tout le sable en une seule fois.

\question[1] \textbf{Quel est le \% de remplissage du 2ème camion ?}

\begin{itemize}
    \item Volume restant de sable : \( 27.03 - 21 = \SI{6.03}{m^3} \).
\end{itemize}

Pourcentage de remplissage du 2ème camion :
\[
\frac{6.03}{21} \times 100 \approx 28.7\%.
\]

Le 2ème camion sera rempli à 28.7\% de sa capacité.

\end{questions}

\end{document}
