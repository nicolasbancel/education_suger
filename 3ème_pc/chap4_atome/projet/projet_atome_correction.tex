\documentclass[a4paper,12pt]{article}
\usepackage{../../../mypackages}
\usepackage{../../../macros}
\usetikzlibrary{shapes.geometric, arrows}

\setlength{\parindent}{0pt}


\begin{document}

\title{Chapitre 4 : L'atome - Projet de fin de chapitre - CORRECTION}
\author{N. Bancel}
\date{Septembre 2024}
\maketitle

\tikzstyle{startstop} = [rectangle, rounded corners, 
minimum width=3cm, 
minimum height=1cm,
text centered, 
draw=black, 
fill=red!30]

\tikzstyle{process} = [rectangle, 
minimum width=3cm, 
minimum height=1cm, 
text centered, 
text width=3cm, 
draw=black, 
fill=orange!30]


\section*{Méthode}

Dans ce genre de problème, il faut se créer ses questions intermédiaires soi-même, \textbf{en commençant par la fin}. Puis reprendre le sujet en entier et répondre aux questions une par une.



\begin{tikzpicture}[node distance=2cm]
  \node (start) [startstop] {1. $Volume_{\text{petits cubes qui rassemblent les atomes de l'univers}} < Volume_{\text{dé à coudre}}$ ?};
  \node (step2) [process, below of=start] {Relation entre $Volume_{\text{petits cubes qui rassemblent les atomes de l'univers}}$ et $Volume_{\text{atomes de l'univers}}$};
  \node (step3) [process, below of=step2] {Relation entre $Volume_{\text{atomes de l'univers}}$ et $Volume_{\text{petits cubes}}$};
  \node (stop) [startstop, below of=step3] {Stop};
\end{tikzpicture}



\end{document}
