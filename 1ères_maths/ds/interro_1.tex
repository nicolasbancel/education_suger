\documentclass{exam}
\usepackage{../../mypackages}
\usepackage{../../macros}

\title{Contrôle N°1 - Suites numériques}
\author{N. Bancel}
\date{Novembre 2024}

\begin{document}

\textbf{Collège Lycée Suger}
\hfill
\textbf{Mathématiques} \\

\textbf{Année 2024-2025}
\hfill
\textbf{1ères STD2A} \par

{\let\newpage\relax\maketitle}

\begin{center}
\textbf{\textcolor{red}{Durée : 30 minutes. La calculatrice n'est pas autorisée}} \\
\textbf{\textcolor{red}{Une réponse donnée sans justification sera considérée comme fausse.}} \\
\end{center}

\section*{Partie 1 : Cours (2 points)}

\begin{questions}
  \question[0.5] Donner la définition d'une suite arithmétique
  \question[0.5] A quelles conditions une suite arithmétique est-elle croissante ou décroissante ?
  \question[0.5] Donner la définition d'une suite géométrique
  \question[0.5] A quelles conditions une suite géométrique est-elle croissante ou décroissante ?

\end{questions} 

\section*{Partie 2 : Suites définies de manière fonctionnelle et récurrente (5 points)}

\begin{questions}
  \question[2.5] Soit la suite $(U_n)$ définie pour tout $n \geq 0$ par la relation suivante :
  
  \[
    \left\{
      \begin{array}{ll}
            U_{n+1} = U_n + n \\
            U_0 = 2 
        \end{array}
      \right.
    \]
  
  \begin{parts}
    \part[1] Calculer les trois premiers termes de la suite $(U_n)$.
    \part[1] Calculer $U_{n+1} - U_{n}$
    \part[0.5] Cette suite est-elle croissante ou décroissante ? Justifier
  \end{parts}

  \question[3] Soit la suite \((V_n)\) définie par la relation fonctionnelle :
  \[
  V_n = 5 \times 2^n.
  \]
  \begin{parts}
    \part[1] Calculer les termes $V_0$, $V_1$ et $V_2$.
    \part[1] Calculer le rapport $\frac{V_{n+1}}{V_n}$
    \part[0.5] Déterminer si la suite $(V_n)$ est géométrique, et préciser sa raison. Cette suite est-elle croissante ou décroissante ? Justifier
  \end{parts}

\end{questions}

\section*{Partie 3 : Suites arithmétiques et géométriques (3 points)}

\begin{questions}
  \question[1.5] Soit la suite \((W_n)\) définie par la relation suivante :
  
  \[
    \left\{
      \begin{array}{ll}
        W_{n+1} = W_n + 4 \\
        W_1 = 6
        \end{array}
      \right.
    \]

  \begin{parts}
    \part[0.5] La suite est-elle arithmétique ? Géométrique ? Si oui, quelle est sa raison ?
    \part[1] Calculer $W_4$ et $W_6$ 
  \end{parts}

  \question[1.5] Soit la suite \((X_n)\) définie par la relation suivante :
  
  \[
    \left\{
      \begin{array}{ll}
        X_{n+1} = 2 X_{n} \\
        X_1 = 3
        \end{array}
      \right.
  \]
  
  \begin{parts}
    \part[0.5] La suite est-elle arithmétique ? Géométrique ? Si oui, quelle est sa raison ?
    \part[1] Calculer $X_3$ et $X_5$.
  \end{parts}


\end{questions}

\end{document}
