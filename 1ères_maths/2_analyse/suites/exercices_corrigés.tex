\documentclass[a4paper,12pt]{article}
\usepackage{../../../mypackages}
\usepackage{../../../macros}
\usepackage{pgfplots}
\usepackage{xcolor}
\usepackage{tcolorbox}

\pgfplotsset{
    compat=1.11,
}

\setlength{\parindent}{0pt}

\begin{document}

\title{Chapitre 2 - Les suites. Corrigés des exercices}
\author{N. Bancel}

\maketitle

\begin{tcolorbox}[colback=gray!10, colframe=gray!50, title=Exercice 19 page 122]
\noindent Déterminer l'expression de $u_{n+1}$ en fonction de $n$ pour chacune des suites suivantes :
\begin{enumerate}
    \item $u_n = 10^n$
    \item $v_n = n^3 - 2n$
    \item $w_n = \dfrac{3n^2 + 1}{n + 1}$
\end{enumerate}
\end{tcolorbox}

\bigskip

\textbf{Réponses :}

\begin{enumerate}
    \item \textbf{Suite} $u_n = 10^n$
    
    Pour cette suite, l'expression de $u_{n+1}$ s'obtient en remplaçant $n$ par $n+1$ dans la formule de $u_n$ :
    \[
    u_{n+1} = 10^{n+1}
    \]
  
    \item \textbf{Suite} $v_n = n^3 - 2n$
    
    Pour cette suite, nous devons calculer $v_{n+1}$ en remplaçant $n$ par $n+1$ dans l'expression de $v_n$ :
    \[
    v_{n+1} = (n+1)^3 - 2(n+1)
    \]
    Développons chaque terme :
    \[
    (n+1)^3 = n^3 + 3n^2 + 3n + 1
    \]
    et 
    \[
    -2(n+1) = -2n - 2
    \]
    En combinant les deux :
    \[
    v_{n+1} = n^3 + 3n^2 + 3n + 1 - 2n - 2 = n^3 + 3n^2 + n - 1
    \]
    Ainsi, l'expression de $v_{n+1}$ est $v_{n+1} = n^3 + 3n^2 + n - 1$. Nous remarquons que cette expression est plus complexe que $v_n$, car elle contient des termes supplémentaires liés à $n$.

    \item \textbf{Suite} $w_n = \dfrac{3n^2 + 1}{n + 1}$
    
    Pour cette suite, nous calculons $w_{n+1}$ en remplaçant $n$ par $n+1$ dans l'expression de $w_n$ :
    \[
    w_{n+1} = \dfrac{3(n+1)^2 + 1}{(n+1) + 1}
    \]
    Commençons par développer le numérateur :
    \[
    3(n+1)^2 = 3(n^2 + 2n + 1) = 3n^2 + 6n + 3
    \]
    Donc le numérateur devient :
    \[
    3n^2 + 6n + 3 + 1 = 3n^2 + 6n + 4
    \]
    Et le dénominateur est :
    \[
    n+1+1 = n+2
    \]
    Ainsi, l'expression de $w_{n+1}$ est :
    \[
    w_{n+1} = \dfrac{3n^2 + 6n + 4}{n + 2}
    \]
    
\end{enumerate}

\begin{tcolorbox}[colback=gray!10, colframe=gray!50, title=Exercice \textbf{66}]
  On considère la suite $(u_n)$ définie par $u_2 = -3$ et, pour tout entier naturel $n \geq 2$, $u_{n+1} = u_n^2 - 6$. Déterminer la valeur des quatre premiers termes de la suite.
  \end{tcolorbox}

  \textbf{Réponses :}

  Nous devons déterminer les quatre premiers termes de la suite $(u_n)$. Le premier terme donné est $u_2 = -3$, et la suite est définie par la relation de récurrence : 
  \[
  u_{n+1} = u_n^2 - 6
  \]
  C'est cette relation qui va nous permettre de calculer les termes suivants à partir des précédents.
  
  \begin{itemize}
      \item \textbf{Calcul de $u_2$ :} \\
      Nous avons directement $u_2 = -3$, qui est donné dans l'énoncé.
  
      \item \textbf{Calcul de $u_3$ :} \\
      Utilisons la relation de récurrence pour calculer $u_3$ à partir de $u_2$ :
      \[
      u_3 = u_2^2 - 6 = (-3)^2 - 6 = 9 - 6 = 3
      \]
      Donc, $u_3 = 3$.
  
      \item \textbf{Calcul de $u_4$ :} \\
      De même, calculons $u_4$ à partir de $u_3$ :
      \[
      u_4 = u_3^2 - 6 = 3^2 - 6 = 9 - 6 = 3
      \]
      Donc, $u_4 = 3$. Nous remarquons ici que le terme $u_4$ est identique à $u_3$.
  
      \item \textbf{Calcul de $u_5$ :} \\
      Continuons avec la même méthode pour calculer $u_5$ :
      \[
      u_5 = u_4^2 - 6 = 3^2 - 6 = 9 - 6 = 3
      \]
      Donc, $u_5 = 3$. Nous observons ici que la suite devient constante à partir de $u_3$.
  \end{itemize}
  
  Ainsi, les quatre premiers termes de la suite sont : 
  \[
  u_2 = -3, \quad u_3 = 3, \quad u_4 = 3, \quad u_5 = 3
  \]
  Nous constatons que la suite est constante à partir du terme $u_3$.



\begin{tcolorbox}[colback=gray!10, colframe=gray!50, title=Exercice \textbf{89}]
  Parmi les suites suivantes, repérer les suites arithmétiques et donner leur raison. Donner ensuite les trois premiers termes de ces suites.
  \[
  \begin{array}{ll}
  1. & \left\{
    \begin{array}{l}
    u_0 = -1 \\
    u_{n+1} = u_n + 4
    \end{array}
    \right. \\
  2. & \left\{
    \begin{array}{l}
    u_0 = -3 \\
    u_{n+1} = u_n - 7
    \end{array}
    \right. \\
  3. & \left\{
    \begin{array}{l}
    u_0 = 12 \\
    u_{n+1} = 8u_n
    \end{array}
    \right. \\
  4. & \left\{
    \begin{array}{l}
    u_0 = 7 \\
    u_n = u_{n-1} + 9
    \end{array}
    \right.
  \end{array}
  \]
  \end{tcolorbox}
  
  \bigskip
  
  \textbf{Réponses :}
  
  Pour chacune des suites, nous devons déterminer si elle est arithmétique et, le cas échéant, calculer sa raison et ses trois premiers termes.
  
  \begin{enumerate}
      \item \textbf{Suite 1 :} \\
      La suite est définie par la relation de récurrence $u_{n+1} = u_n + 4$. \\
      Cette relation indique que pour passer d'un terme au suivant, on ajoute toujours la même quantité, 4. \\
      La suite est donc arithmétique de raison $r = 4$. \\
      Les trois premiers termes de la suite sont :
      \[
      u_0 = -1, \quad u_1 = u_0 + 4 = -1 + 4 = 3, \quad u_2 = u_1 + 4 = 3 + 4 = 7
      \]
      Donc, les trois premiers termes sont $u_0 = -1$, $u_1 = 3$, $u_2 = 7$.
  
      \item \textbf{Suite 2 :} \\
      La suite est définie par $u_{n+1} = u_n - 7$. \\
      Ici, la relation indique que l'on soustrait 7 à chaque terme pour obtenir le suivant. \\
      Cette suite est donc arithmétique de raison $r = -7$. \\
      Les trois premiers termes de la suite sont :
      \[
      u_0 = -3, \quad u_1 = u_0 - 7 = -3 - 7 = -10, \quad u_2 = u_1 - 7 = -10 - 7 = -17
      \]
      Donc, les trois premiers termes sont $u_0 = -3$, $u_1 = -10$, $u_2 = -17$.
  
      \item \textbf{Suite 3 :} \\
      La suite est définie par $u_{n+1} = 8u_n$. \\
      Ici, nous n'avons pas une relation d'ajout ou de soustraction d'une constante, mais une multiplication par 8. \\
      Cette suite n'est donc pas arithmétique. \\
      Les trois premiers termes sont obtenus en multipliant par 8 à chaque étape :
      \[
      u_0 = 12, \quad u_1 = 8 \times 12 = 96, \quad u_2 = 8 \times 96 = 768
      \]
  
      \item \textbf{Suite 4 :} \\
      La suite est définie par $u_n = u_{n-1} + 9$. \\
      Cela signifie que pour passer d'un terme au suivant, on ajoute 9. \\
      Cette suite est donc arithmétique de raison $r = 9$. \\
      Les trois premiers termes sont :
      \[
      u_0 = 7, \quad u_1 = u_0 + 9 = 7 + 9 = 16, \quad u_2 = u_1 + 9 = 16 + 9 = 25
      \]
      Donc, les trois premiers termes sont $u_0 = 7$, $u_1 = 16$, $u_2 = 25$.
  \end{enumerate}
  
  \bigskip
  
  \begin{tcolorbox}[colback=gray!10, colframe=gray!50, title=Exercice \textbf{95} page 129]
  On considère la suite $(u_n)$ définie ci-dessous :
  \[
  \left\{
    \begin{array}{l}
    u_0 = 7 \\
    u_{n+1} = u_n + 10
    \end{array}
  \right.
  \]
  Donner son sens de variation.
  \end{tcolorbox}
  
  \bigskip
  
  \textbf{Réponses :}
  
  La suite $(u_n)$ est définie par la relation de récurrence $u_{n+1} = u_n + 10$. Cela signifie que pour passer d'un terme au suivant, on ajoute toujours 10.
  
  \begin{itemize}
      \item \textbf{Sens de variation :} \\
      Puisque l'on ajoute un nombre positif (10) à chaque étape, la suite est strictement croissante. En d'autres termes, chaque terme de la suite est plus grand que le précédent.
      
      \item \textbf{Justification :} \\
      Pour justifier cette affirmation, nous pouvons observer que la différence entre deux termes consécutifs est toujours positive :
      \[
      u_{n+1} - u_n = 10 > 0
      \]
      Donc, la suite est strictement croissante.
  \end{itemize}

  \begin{tcolorbox}[colback=gray!10, colframe=gray!50, title=Exercice \textbf{97}]
    Parmi les suites suivantes, repérer les suites arithmétiques, donner leur raison ainsi que leur sens de variation.
    \[
    \begin{array}{ll}
    1. & \left\{
      \begin{array}{l}
      u_1 = 3 \\
      u_{n-1} = u_{n-2} + 8
      \end{array}
      \right. \\
    2. & \left\{
      \begin{array}{l}
      u_1 = 8 \\
      u_{n+1} = u_n - 2
      \end{array}
      \right. \\
    3. & \left\{
      \begin{array}{l}
      u_0 = 1 \\
      u_{n+1} = -5 + u_n
      \end{array}
      \right. \\
    4. & u_n = 3n \\
    5. & \left\{
      \begin{array}{l}
      u_0 = 9 \\
      u_n = 4u_{n-1} + 2
      \end{array}
      \right. \\
    6. & u_{n+1} = 5 + n
    \end{array}
    \]
    \end{tcolorbox}
    
    \bigskip
    
    \textbf{Réponses :}
    
    \begin{enumerate}
        \item \textbf{Suite 1 :} \\
        La suite est définie par la relation $u_{n-1} = u_{n-2} + 8$ et par $u_1 = 3$. \\
        Pour cette suite, on voit que la relation de récurrence implique une différence constante de 8 entre les termes consécutifs $u_{n-1}$ et $u_{n-2}$. Cela signifie que pour passer d'un terme au suivant, on ajoute 8. \\
        La suite est donc arithmétique de raison $r = 8$. \\
        Les premiers termes de la suite peuvent être calculés :
        \[
        u_1 = 3, \quad u_2 = u_1 + 8 = 3 + 8 = 11, \quad u_3 = u_2 + 8 = 11 + 8 = 19
        \]
        \textbf{Sens de variation :} \\
        Puisque la raison est positive ($r = 8$), la suite est strictement croissante.
    
        \item \textbf{Suite 2 :} \\
        La suite est définie par la relation $u_{n+1} = u_n - 2$. Cette relation montre que pour passer d'un terme au suivant, on soustrait 2. \\
        La suite est donc arithmétique de raison $r = -2$. \\
        \textbf{Sens de variation :} \\
        Puisque la raison est négative ($r = -2$), la suite est strictement décroissante.
    
        \item \textbf{Suite 3 :} \\
        La relation de récurrence est $u_{n+1} = -5 + u_n$. On observe que pour passer d’un terme au suivant, on soustrait 5 à chaque étape. \\
        La suite est donc arithmétique de raison $r = -5$. \\
        \textbf{Sens de variation :} \\
        Comme la raison est négative, la suite est strictement décroissante.
    
        \item \textbf{Suite 4 :} \\
        Cette suite est définie par $u_n = 3n$. On reconnaît ici une suite arithmétique car chaque terme est un multiple de $n$ avec un coefficient constant de 3. \\
        La raison est donc $r = 3$. \\
        \textbf{Sens de variation :} \\
        Puisque la raison est positive, la suite est strictement croissante.
    
        \item \textbf{Suite 5 :} \\
        La relation est $u_n = 4u_{n-1} + 2$. Ce n’est pas une suite arithmétique, car pour une suite arithmétique, la relation entre les termes doit impliquer une addition ou soustraction d'une constante, sans multiplication. Ici, le terme est multiplié par 4 à chaque étape.
    
        \item \textbf{Suite 6 :} \\
        La relation donnée est $u_{n+1} = 5 + n$. Cette suite n’est pas arithmétique car le terme $n$ change à chaque étape (il n'y a pas une constante ajoutée ou soustraite de manière fixe). En revanche, elle est définie par une relation linéaire en fonction de $n$.
    \end{enumerate}


    \begin{tcolorbox}[colback=gray!10, colframe=gray!50, title=Exercice \textbf{100}]
      Soit $(v_n)$ la suite définie pour tout $n \in \mathbb{N}$ par $v_n = 4n + 1$. Montrer que $v$ est une suite arithmétique.
      \end{tcolorbox}
      
      \bigskip
      
      \textbf{Réponses :}
      
      Nous devons montrer que la suite $(v_n)$ définie par $v_n = 4n + 1$ est arithmétique.
      
      \textbf{Étape 1 : Identifier la forme d'une suite arithmétique.} \\
      Une suite est dite arithmétique s'il existe une constante $r$ (appelée raison) telle que :
      \[
      v_{n+1} = v_n + r
      \]
      Autrement dit, pour chaque terme de la suite, il faut qu'on puisse passer d'un terme au suivant en ajoutant une constante.
      
      \textbf{Étape 2 : Calculer la différence entre deux termes consécutifs.} \\
      Calculons la différence $v_{n+1} - v_n$ :
      \[
      v_{n+1} = 4(n+1) + 1 = 4n + 4 + 1 = 4n + 5
      \]
      \[
      v_n = 4n + 1
      \]
      Ainsi, la différence entre $v_{n+1}$ et $v_n$ est :
      \[
      v_{n+1} - v_n = (4n + 5) - (4n + 1) = 4
      \]
      Cette différence est constante et égale à 4, ce qui prouve que $(v_n)$ est une suite arithmétique de raison $r = 4$.
      
      \textbf{Conclusion :} \\
      La suite $(v_n)$ est bien arithmétique avec une raison de $4$.
      
      \bigskip
      
      \begin{tcolorbox}[colback=gray!10, colframe=gray!50, title=Exercice \textbf{103}]
      Soit $(u_n)$ la suite définie pour tout $n \in \mathbb{N}$ par $u_n = n - 1$.
      \begin{enumerate}
          \item Calculer ses quatre premiers termes et les représenter graphiquement.
          \item D'après votre graphique, la suite peut-elle être arithmétique ?
          \item Démontrer que $u$ est une suite arithmétique.
      \end{enumerate}
      \end{tcolorbox}
      
      \bigskip
      
      \textbf{Réponses :}
      
      \begin{enumerate}
          \item \textbf{Calcul des quatre premiers termes :} \\
          La suite est définie par $u_n = n - 1$. Calculons les quatre premiers termes :
          \[
          u_0 = 0 - 1 = -1, \quad u_1 = 1 - 1 = 0, \quad u_2 = 2 - 1 = 1, \quad u_3 = 3 - 1 = 2
          \]
          Les quatre premiers termes sont donc $u_0 = -1$, $u_1 = 0$, $u_2 = 1$, $u_3 = 2$.
          
          \textbf{Représentation graphique :} \\
          Représentons ces quatre points sur un repère. Les points sont $(0, -1)$, $(1, 0)$, $(2, 1)$, $(3, 2)$. La courbe formée est une droite.
      
          \item \textbf{La suite peut-elle être arithmétique ?} \\
          D'après le graphique, les points semblent alignés sur une droite. Ceci est une indication forte que la suite est arithmétique.
      
          \item \textbf{Démonstration que $(u_n)$ est arithmétique :} \\
          Pour prouver que $(u_n)$ est une suite arithmétique, il suffit de vérifier que la différence entre deux termes consécutifs est constante. Calculons $u_{n+1} - u_n$ :
          \[
          u_{n+1} = (n+1) - 1 = n, \quad u_n = n - 1
          \]
          Ainsi :
          \[
          u_{n+1} - u_n = n - (n - 1) = 1
          \]
          La différence est constante et égale à 1, ce qui prouve que $(u_n)$ est une suite arithmétique de raison $r = 1$.
      \end{enumerate}
      

      \begin{tcolorbox}[colback=gray!10, colframe=gray!50, title=Exercice \textbf{124} page 132]
        Parmi les suites suivantes, repérer les suites géométriques et donner leur raison. Donner ensuite les 3 premiers termes de ces suites.
        \[
        \begin{array}{ll}
        1. & \left\{
          \begin{array}{l}
          u_0 = -1 \\
          u_{n+1} = 3u_n
          \end{array}
          \right. \\
        2. & \left\{
          \begin{array}{l}
          u_1 = 3 \\
          u_{n+1} = u_n - 5
          \end{array}
          \right. \\
        3. & \left\{
          \begin{array}{l}
          u_0 = 2 \\
          u_{n+1} = \dfrac{1}{2} u_n
          \end{array}
          \right. \\
        4. & \left\{
          \begin{array}{l}
          u_3 = 3 \\
          u_n = 4u_{n-1}
          \end{array}
          \right. \\
        5. & \left\{
          \begin{array}{l}
          u_0 = 2 \\
          u_n = -2u_{n-1} + 4
          \end{array}
          \right. \\
        6. & u_{n+1} = 8 + n
        \end{array}
        \]
        \end{tcolorbox}
        
        \bigskip
        
        \textbf{Réponses :}
        
        \begin{enumerate}
            \item \textbf{Suite 1 :} \\
            La relation de récurrence est $u_{n+1} = 3u_n$. Cette relation montre que pour passer d’un terme au suivant, on multiplie le terme précédent par 3. La suite est donc géométrique avec une raison $r = 3$. \\
            Les trois premiers termes de la suite sont :
            \[
            u_0 = -1, \quad u_1 = 3 \times (-1) = -3, \quad u_2 = 3 \times (-3) = -9
            \]
            \item \textbf{Suite 2 :} \\
            La relation de récurrence est $u_{n+1} = u_n - 5$. Ici, il s'agit d'une suite arithmétique, pas géométrique, car la différence entre les termes est constante, pas le quotient.
            \item \textbf{Suite 3 :} \\
            La relation de récurrence est $u_{n+1} = \frac{1}{2} u_n$. Pour passer d’un terme au suivant, on divise chaque terme par 2. La suite est donc géométrique de raison $r = \frac{1}{2}$. \\
            Les trois premiers termes sont :
            \[
            u_0 = 2, \quad u_1 = \frac{1}{2} \times 2 = 1, \quad u_2 = \frac{1}{2} \times 1 = \frac{1}{2}
            \]
            \item \textbf{Suite 4 :} \\
            La relation est $u_n = 4u_{n-1}$. Cela montre que pour passer d’un terme au suivant, on multiplie par 4. La suite est donc géométrique avec une raison $r = 4$. Les trois premiers termes (à partir de $u_3$) sont :
            \[
            u_3 = 3, \quad u_4 = 4 \times 3 = 12, \quad u_5 = 4 \times 12 = 48
            \]
            \item \textbf{Suite 5 :} \\
            Cette suite n'est pas géométrique car la relation de récurrence inclut une addition, donc elle ne respecte pas la forme d'une suite géométrique.
            \item \textbf{Suite 6 :} \\
            La relation de récurrence est $u_{n+1} = 8 + n$, ce qui n’est pas une suite géométrique car elle ne multiplie pas chaque terme par une constante.
        \end{enumerate}
        
        \bigskip
        
        \begin{tcolorbox}[colback=gray!10, colframe=gray!50, title=Exercice \textbf{130}]
        On considère la suite $(u_n)$ définie ci-dessous :
        \[
        \left\{
          \begin{array}{l}
          u_0 = 2 \\
          u_{n+1} = 9u_n
          \end{array}
        \right.
        \]
        Donner son sens de variation.
        \end{tcolorbox}
        
        \bigskip
        
        \textbf{Réponses :}
        
        La suite $(u_n)$ est définie par la relation $u_{n+1} = 9u_n$. Il s’agit d’une suite géométrique avec une raison $r = 9$.
        
        \textbf{Sens de variation :} \\
        Comme la raison est un nombre positif et supérieur à 1, la suite est strictement croissante. En effet, à chaque étape, le terme suivant est 9 fois plus grand que le précédent, ce qui entraîne une augmentation rapide de la valeur des termes.
        
        \bigskip
        
        \begin{tcolorbox}[colback=gray!10, colframe=gray!50, title=Exercice \textbf{134} page 133]
        Soit $v_n$ une suite géométrique telle que $v_7 = 36$ et $v_9 = 9$. Donner sa raison et en déduire son sens de variation.
        \end{tcolorbox}
        
        \bigskip
        
        \textbf{Réponses :}
        
        Nous avons deux termes de la suite : $v_7 = 36$ et $v_9 = 9$. Comme il s'agit d'une suite géométrique, on sait que :
      \[
      \begin{aligned}
      v_9 &= v_8 \times r \\
      \text{et} \quad v_8 &= v_7 \times r \\
      \text{donc} \quad v_9 &= (v_7 \times r) \times r \\
      \text{En simplifiant :} \quad v_9 &= v_7 \times r^2
      \end{aligned}
      \]
        
        En remplaçant par les valeurs données :
        \[
        9 = 36 \times r^2
        \]
        On résout cette équation pour $r$ :
        \[
        r^2 = \frac{9}{36} = \frac{1}{4} \quad \Rightarrow \quad r = \frac{1}{2} \text{ ou } r = -\frac{1}{2}
        \]

        Dans le cadre du programme de 1ère STD2A, on se limite aux suites dont la raison (pour une suite géométrique) est \textbf{positive}.
        Donc la seule solution acceptée ici est $r = \frac{1}{2}$
        
        \textbf{Sens de variation :} \\
        $r = \frac{1}{2}$, donc $0 < r < 1$ : la suite est strictement décroissante car à chaque étape, on divise la valeur du terme précédent par 2. \\
        
        \bigskip
        
        \begin{tcolorbox}[colback=gray!10, colframe=gray!50, title=Exercice \textbf{135} page 133]
        La suite $(v_n)$ est définie pour tout entier naturel $n$ par $v_n = 5 \times 2^n$.
        \begin{enumerate}
            \item Calculer $v_0$, $v_1$, et $v_2$. Quelle semble être la nature de la suite $(v_n)$ ?
            \item Exprimer $v_{n+1}$ en fonction de $n$.
            \item Calculer le rapport $\dfrac{v_{n+1}}{v_n}$. Qu'en déduisez-vous ?
        \end{enumerate}
        \end{tcolorbox}
        
        \bigskip
        
        \textbf{Réponses :}
        
        \begin{enumerate}
            \item \textbf{Calcul de $v_0$, $v_1$, et $v_2$ :} \\
            La suite est définie par $v_n = 5 \times 2^n$. Calculons les premiers termes :
            \[
            v_0 = 5 \times 2^0 = 5, \quad v_1 = 5 \times 2^1 = 10, \quad v_2 = 5 \times 2^2 = 20
            \]
            Il semble que la suite soit géométrique, car on multiplie chaque terme par une constante (ici $2$).
        
            \item \textbf{Expression de $v_{n+1}$ en fonction de $n$ :} \\
            En utilisant la définition de $v_n$, on a :
            \[
            v_{n+1} = 5 \times 2^{n+1} = 5 \times 2 \times 2^n = 2 \times v_n
            \]
            Cette expression montre clairement que $v_{n+1}$ est le double de $v_n$.
        
            \item \textbf{Calcul du rapport $\dfrac{v_{n+1}}{v_n}$ :} \\
            Calculons le rapport :
            \[
            \dfrac{v_{n+1}}{v_n} = \dfrac{2 \times v_n}{v_n} = 2
            \]
            Le rapport entre deux termes consécutifs est constant et égal à 2, ce qui confirme que la suite est géométrique de raison $r = 2$.
        \end{enumerate}
        

\end{document}
