\documentclass[a4paper,12pt]{article}
\usepackage{../../mypackages}
\usepackage{../../macros}

\setlength{\parindent}{0pt}


\begin{document}

\title{Chapitre 1 : Matériaux - Généralités}
\author{N. Bancel}
\date{Septembre 2024}
\maketitle

\section{Les familles de matériaux}

\subsection{Généralités}

Il existe trois grandes familles de matériaux :

\begin{itemize}[noitemsep]
  \item les matériaux métalliques (métaux ou alliages)
  \item les matériaux organiques (origine végétale ou animale, ou synthétiques). Ils sont essentiellement composés de \textbf{carbone}
  \item Les matériaux minéraux (Roches, céramiques, verres). Pourquoi le verre : vient du sable. Céramique : vient de la terre (argile cuite). (Les minéraux sont des non-métaux solides dans la terre / nature)
\end{itemize}

\begin{tabular}{p{5cm}p{5cm}p{5cm}}
  \toprule
  {Matériaux métalliques} & {Matériaux organiques} & {Matériaux minéraux} \\
  \midrule
  \begin{itemize}[noitemsep]
    \item Fer
    \item Cuivre
    \item Bronze
  \end{itemize} & 
  \begin{itemize}[noitemsep]
    \item Bois
    \item Coton
    \item Papier
    \item Cuir
    \item Laine
    \item Matières plastiques (car synthétisées à partir du pétrôle)
  \end{itemize}
   & 
   \begin{itemize}[noitemsep]
    \item Diamant
    \item Sable
    \item Porcelaine
    \item Verre 
  \end{itemize} \\
  \bottomrule
\end{tabular}


\subsection{Les matériaux métalliques}

Les matériaux métalliques peuvent être :

\begin{itemize}[noitemsep]
    \item des métaux purs, c’est-à-dire composés d’un seul élément chimique, comme le fer,
    \item ou des alliages, c’est-à-dire composés d'au moins deux éléments chimiques, comme :
    \begin{itemize}[noitemsep]
        \item le bronze (alliage de cuivre et d’étain),
        \item le laiton (alliage de cuivre et de zinc)
    \end{itemize}
\end{itemize}


\section{Propriétés des matériaux et impact}

\subsection{Propriétés des matériaux}

Pour répondre à des cahiers des charges et contraintes de fabrication, le choix d'un matériau se fait suivant des critères chimiques, physiques, ou mécaniques.
\begin{itemize}[noitemsep]
\item Trampoline (Elasticité)
\item Tamis d'une raquette de tennis (Elasticité)
\item Pont (Résistance à la rupture / Solidité)
\item Tableau (Résistance à humidité)
\item Vélo (Résistance à corrosion + légèreté)

\end{itemize}


\begin{tabular}{p{5cm}p{5cm}p{5cm}}
  \toprule
  {Contrainte Physique} & {Contrainte chimique} & {Contrainte mécanique} \\
  \midrule
  \begin{itemize}[noitemsep]
    \item Poids / Masse volumique (poids d'un matériau par unité de volume)
    \item Conductibilité (capacité à conduire de l'électricité ou de la chaleur)
  \end{itemize} & 
  \begin{itemize}[noitemsep]
    \item Résistance à la corrosion due à l'eau / Détérioration d'un matériaux
  \end{itemize}
   & 
   \begin{itemize}[noitemsep]
    \item Résistance à la rupture : effort qu'il faut pour qu'un matériau se casse
    \item Elasticité : Aptitude d'un corps à subir des déformations temporaires 
  \end{itemize} \\
  \bottomrule
\end{tabular}

\vspace{1em}

\textbf{Formulation plus technique}


\begin{itemize}[noitemsep]
  \item Contraintes physiques : xxx
  \item Contraintes chimiques : transformations subies par le matériau sous l'action d'autres espèces chimiques
  \item Contraintes mécaniques : comportement du matériau lorsqu'il est soumis à des déformations
\end{itemize}


\subsection{Impact sur l'environnement}

Contraintes environnementales : 
\begin{itemize}[noitemsep]
  \item Respecter l'environnement
  \item Etre recyclable
  \item Réduire l'impact environnemental au moment de la production et de l'opération (utilisation de matière première, empreinte carbone (rejet de CO2 dans l'air)) etc
\end{itemize}

2 nouveaux types de matériaux se distinguent : 
\begin{itemize}[noitemsep]
  \item Les matériaux biodégradables. Qui se dégradent en eau, dioxyde de carbone, biomasse sous l’action de micro-organismes (bactéries, champignons) 
  \item Les matériaux biosourcés : issus de la matière organique, végétale, ou animale (bois, paille etc)
\end{itemize}


\begin{tcolorbox}[colback=blue!10!white, colframe=blue!75!black, title=Exemples - Application]
  \begin{itemize}[noitemsep]
    \item Exo 1 page 17
    \item Exo 2 page 17
  \end{itemize}
\end{tcolorbox}

\subsection{Les matériaux innovants}

\subsubsection{Les matériaux composites}

\textit{Comment maximiser et combiner les propriétés de 2 matériaux ? Grâce aux matériaux composites}

\begin{tcolorbox}[colback=green!10!white, colframe=green!75!black, title=Définition : Matériaux composites]
  Les matériaux composites sont le produit de la combinaison d’au moins deux matériaux différents non miscibles. \\
  Le but est d’obtenir un matériau présentant des caractéristiques que les composants, seuls, n’avaient pas (très souvent la légèreté et la rigidité).
\end{tcolorbox}

\vspace{1em}

Composition d'un matériau composite : 
\begin{itemize}[noitemsep]
  \item \textbf{La matrice} : le liant. Assure la cohésion, isole de l'extérieur et protège le ernfort
  \item \textbf{Le renfort} : squelette du matériau. Assure la solidité. Ils peuvent être de 2 types : 
  \begin{itemize}[noitemsep]
    \item des particules (cailloux, billes de verre)
    \item des fibres (fibres de carbone, fibres de verre)
  \end{itemize}
\end{itemize}

\subsubsection{Les nanomatériaux}

\begin{tcolorbox}[colback=green!10!white, colframe=green!75!black, title=Définition : Matériaux composites]
  Les nanomatériaux sont des matériaux dont l'une au moins des dimensions est comprise
  entre 1 et 100 \unit{\nm} (\(10^{-9}\) mètres).
\end{tcolorbox}

\section{Masse volumique}

\subsection{Masse et volume}

\begin{tcolorbox}
  La masse d'un corps représente la quantité de matière qui le compose. \par
  \textbf{Remarque} : Elle est constante quelque soit l'endroit où l'on se trouve (sur Mars, sur Terre, sur la Lune) \par
  \textbf{Unité} : Dans le système international, l'unité légale de la masse est le kilogramme (symbole \textbf{kg})
  
\end{tcolorbox}
  
\begin{tcolorbox}
    Le volume d'un corps est la grandeur qui indique l'espace qu'il occupe. \par
    \textbf{Unité} : Dans le système international, l'unité légale du volume est le mètre cube (symbole \textbf{$m^3$})
\end{tcolorbox}
  
Cela représente le volume d'un cube d'un mètre de côté.
  
  \begin{tcolorbox}
    Le masse volumique est la masse de ce matériau par unité de volume \par
    \textbf{Unité} : Unité légale de la masse volumique est le kilogramme par mètre cube (symbole \textbf{$kg / m^3$})
  \end{tcolorbox}
  
\subsection{Relation entre masse, volume et masse volumique}
  
Pour un matériau \textbf{plein} donné, la masse et le volume sont proportionnels. 
  
\begin{tcolorbox}
  La relation de proportionnalité entre la masse m et la volume V du matériau s'écrit
  \(m = \rho * V\)
  Le coefficient de proportionnalité \(\rho\) est la masse volumique du matériau
\end{tcolorbox}


\begin{tcolorbox}[colback=blue!10!white, colframe=blue!75!black, title=Exemples - Application]
  \begin{itemize}[noitemsep]
    \item Faire le TP de la page 13
  \end{itemize}
\end{tcolorbox}


\end{document}
