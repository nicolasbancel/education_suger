\documentclass[a4paper,12pt]{article}
\usepackage{../../mypackages}
\usepackage{../../macros}

\setlength{\parindent}{0pt}

\begin{document}

\title{Chapitre 8 : Poids et masse d'un corps}
\author{N. Bancel}
\date{Octobre 2024}
\maketitle

\section*{Activité documentaire : Calcul de la force d'attraction gravitationnelle}

\subsection*{1. La force d’attraction gravitationnelle est-elle une action de contact ou une action à distance ?}
La force gravitationnelle est une action à distance, car elle s’exerce sans contact direct entre les deux corps.

\subsection*{2. Calcul de la force gravitationnelle exercée par la Terre sur la Lune}
\begin{compactenum}
    \item \textbf{Théorie / Formule} \\
    La loi de la gravitation universelle de Newton s'écrit :
    \begin{equation}
        F = G \times \frac{m_1 \times m_2}{d^2}
    \end{equation}
    avec :
    \begin{compactenum}
        \item $G = 6,67 \times 10^{-11}$ N.m$^2$/kg$^2$
        \item $m_1 = 5,97 \times 10^{24}$ kg (masse de la Terre)
        \item $m_2 = 7,35 \times 10^{22}$ kg (masse de la Lune)
        \item $d = 3,844 \times 10^8$ m (distance Terre-Lune)
    \end{compactenum}

    \item \textbf{Application numérique} \\
    \begin{equation}
        F = \frac{(6,67 \times 10^{-11}) \times (5,97 \times 10^{24}) \times (7,35 \times 10^{22})}{(3,844 \times 10^8)^2}
    \end{equation}
    \begin{equation}
        F \approx 1,98 \times 10^{20} N
    \end{equation}

    \item \textbf{Interprétation} \\
    La force d’attraction gravitationnelle exercée par la Terre sur la Lune est donc de $1,98 \times 10^{20}$ N.
\end{compactenum}

\subsection*{3. Valeur de la force exercée par la Lune sur la Terre}
D’après la troisième loi de Newton (principe des actions réciproques), la force exercée par la Lune sur la Terre est de même intensité que celle exercée par la Terre sur la Lune : $1,98 \times 10^{20}$ N.

\subsection*{4. Calcul des forces gravitationnelles entre la Terre et le module LEM}
\begin{compactenum}
    \item \textbf{Théorie / Formule} \\
    On utilise la même formule de la gravitation universelle.

    \item \textbf{Données} \\
    \begin{compactenum}
        \item $m_{LEM} = 15 \times 10^3$ kg (masse du LEM)
        \item $r_{Terre} = 6,38 \times 10^6$ m (rayon terrestre)
        \item $r_{Lune} = 1,74 \times 10^6$ m (rayon lunaire)
    \end{compactenum}

    \item \textbf{Application numérique} \\
    \textbf{Force entre la Terre et le LEM} :
    \begin{equation}
        F_{Terre-LEM} = \frac{(6,67 \times 10^{-11}) \times (5,97 \times 10^{24}) \times (15 \times 10^3)}{(6,38 \times 10^6)^2}
    \end{equation}
    \begin{equation}
        F_{Terre-LEM} \approx 147000 N
    \end{equation}
    
    \textbf{Force entre la Lune et le LEM} :
    \begin{equation}
        F_{Lune-LEM} = \frac{(6,67 \times 10^{-11}) \times (7,35 \times 10^{22}) \times (15 \times 10^3)}{(1,74 \times 10^6)^2}
    \end{equation}
    \begin{equation}
        F_{Lune-LEM} \approx 2450 N
    \end{equation}

    \item \textbf{Interprétation} \\
    La force gravitationnelle exercée par la Terre sur le module est bien plus grande que celle exercée par la Lune, expliquant pourquoi le module a plus de poids sur Terre que sur la Lune.
\end{compactenum}

\subsection*{5. Pourquoi la loi de la gravitation est-elle qualifiée d’universelle ?}
La loi de la gravitation de Newton est qualifiée d’universelle car elle s’applique à tous les corps possédant une masse, quelle que soit leur position dans l’univers.

\subsection*{6. Conclusion sur la loi de gravitation universelle}
La loi de Newton énonce que tout corps exerçant une masse attire un autre corps selon une force appelée force de gravitation. Cette force est une interaction attractive à distance qui dépend :
\begin{compactenum}
    \item des masses des deux corps,
    \item de la distance qui les sépare,
    \item de la constante gravitationnelle universelle $G$.
\end{compactenum}

\section*{Activité documentaire : Relation entre poids et masse de corps}

\subsection*{1. Quel instrument permet de mesurer la masse et le poids d'un corps ?}
\begin{compactenum}
    \item La masse d'un corps est mesurée avec une balance et s'exprime en kilogrammes (kg).
    \item Le poids d'un corps est mesuré avec un dynamomètre et s'exprime en newtons (N).
\end{compactenum}

\subsection*{2. Poids maximal mesurable avec le dynamomètre du document 3}
D'après le tableau des caractéristiques du dynamomètre :
\begin{compactenum}
    \item Plage de mesure : de $0$ à $5$ N.
    \item Le poids maximal mesurable est donc $5$ N.
\end{compactenum}

\subsection*{3. Mesure du poids pour différentes masses}
Les valeurs expérimentales doivent être reportées dans un tableau et comparées.

\subsection*{4. Tracé du graphique de l'évolution du poids en fonction de la masse}
On trace un graphique avec :
\begin{compactenum}
    \item L'axe des abscisses ($x$) représentant la masse (en kg).
    \item L'axe des ordonnées ($y$) représentant le poids (en N).
    \item Une courbe qui doit être une droite passant par l'origine.
\end{compactenum}

\subsection*{5. Analyse du graphique obtenu}
\begin{compactenum}
    \item La courbe obtenue est une droite passant par l'origine, ce qui signifie que le poids est proportionnel à la masse.
    \item Cela implique que le rapport $P / m$ est constant.
\end{compactenum}

\subsection*{6. Relation entre le poids, la masse et l'intensité de la pesanteur}
La relation fondamentale du poids est :
\begin{equation}
    P = m \times g
\end{equation}
avec :
\begin{compactenum}
    \item $P$ : poids (N)
    \item $m$ : masse (kg)
    \item $g$ : intensité de la pesanteur (N/kg)
\end{compactenum}
D'après les mesures sur Terre, la valeur de $g$ est environ $9,8$ N/kg.

\subsection*{7. Calcul de l'intensité de la pesanteur sur la Terre et la Lune}
\begin{compactenum}
    \item \textbf{Théorie / Formule} \\
    L'intensité de la pesanteur est donnée par :
    \begin{equation}
        g = G \times \frac{M}{R^2}
    \end{equation}
    avec :
    \begin{compactenum}
        \item $G = 6,67 \times 10^{-11}$ N.m$^2$/kg$^2$
        \item $M$ : masse de l'astre (kg)
        \item $R$ : rayon de l'astre (m)
    \end{compactenum}
    
    \item \textbf{Application numérique} \\
    \textbf{Sur Terre} :
    \begin{equation}
        g_{Terre} = \frac{(6,67 \times 10^{-11}) \times (5,97 \times 10^{24})}{(6,38 \times 10^6)^2}
    \end{equation}
    \begin{equation}
        g_{Terre} \approx 9,81 \text{ N/kg}
    \end{equation}
    
    \textbf{Sur la Lune} :
    \begin{equation}
        g_{Lune} = \frac{(6,67 \times 10^{-11}) \times (7,35 \times 10^{22})}{(1,74 \times 10^6)^2}
    \end{equation}
    \begin{equation}
        g_{Lune} \approx 1,62 \text{ N/kg}
    \end{equation}
    
    \item \textbf{Interprétation} \\
    La pesanteur sur la Lune est beaucoup plus faible que sur Terre, expliquant pourquoi un objet y pèse moins.
\end{compactenum}

\subsection*{8. Conclusion sur la gravitation et la pesanteur}
La loi de Newton et les calculs montrent que :
\begin{compactenum}
    \item Le poids est proportionnel à la masse.
    \item Le coefficient de proportionnalité est l'intensité de la pesanteur $g$.
    \item Plus l'astre est massif et grand, plus $g$ est élevé.
    \item Il est plus facile de s'échapper de la gravité de la Lune car $g_{Lune} \ll g_{Terre}$.
\end{compactenum}

\end{document}





\end{document}

