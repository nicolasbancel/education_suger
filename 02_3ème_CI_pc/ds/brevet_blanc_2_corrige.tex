\documentclass[answers]{exam}
\usepackage{../../mypackages}
\usepackage{../../macros}


\SolutionEmphasis{\color{blue}}
\renewcommand{\solutiontitle}{\noindent}


\title{Brevet blanc N°2 - Physique-Chimie}

\begin{document}

\textbf{Collège Lycée Suger}
\hfill
\textbf{Physique-Chimie} \\

\textbf{Année 2024-2025 - 3ème trimestre}
\hfill
\textbf{Brevet Blanc} \par

{\let\newpage\relax\maketitle}

\begin{center}
\textbf{\textcolor{red}{Durée : 30 minutes. \\
L'usage de la calculatrice avec mode examen actif est autorisé. \\ 
L'usage de la calculatrice sans mémoire, « type collège » est autorisé.}} \\
\textbf{Toute réponse, même incomplète, montrant la démarche de recherche du candidat sera prise en compte dans la notation} \\
\textbf{L'épreuve de Physique comporte 8 questions et est notée sur 25.} \\
\end{center}

\section*{Mission Alpha}

\textit{Extrait du Brevet 2022}
\vspace{1em}

Le 23 avril 2021, l'astronaute français Thomas Pesquet a décollé depuis la base de Cap Canaveral en Floride (USA) pour rejoindre la station spatiale internationale en orbite autour de la Terre, avec trois autres membres d'équipage : c'est la mission Alpha.
L'équipage a rejoint la station spatiale internationale à bord du vaisseau spatial Crew Dragon, lancé par une fusée Falcon 9.

\begin{figure}[H]
  \centering
  \includegraphics[width=0.4\linewidth]{img/brevet_01.jpg}
\end{figure}


\vspace{0.3cm}
\textbf{Les parties 1 et 2 sont indépendantes.}

\subsection*{Partie 1 – L’eau et l’air dans la station spatiale.}

\textit{L’eau et l’air sont nécessaires à la vie des astronautes : leurs besoins sont assurés par différents procédés.}

\begin{questions}

  \question[4] Parmi les formules chimiques ci-dessous, recopier sur la copie les noms de celles qui correspondent à des molécules. Justifier la réponse.
    
  \begin{center}
  \begin{tabular}{lll}
  Diazote : \ce{N2} & Dioxygène : \ce{O2} \\
  Hydrogène : \ce{H} & Oxygène : \ce{O} \\
  Eau : \ce{H2O} & Azote : \ce{N} \\
  \end{tabular}
  \end{center}



\begin{solution}
\subsection*{Correction}

Par définition, une molécule est constituée de deux atomes ou plus liés chimiquement. Parmi les formules chimiques fournies, nous devons identifier celles qui représentent des molécules.

Analysons les formules données :

\[
\begin{array}{lll}
\text{Diazote : } \ce{N2} & \text{Dioxygène : } \ce{O2} \\
\text{Hydrogène : } \ce{H} & \text{Oxygène : } \ce{O} \\
\text{Eau : } \ce{H2O} & \text{Azote : } \ce{N} \\
\end{array}
\]

Pour déterminer lesquelles sont des molécules :

\begin{compactitem}
\item \ce{N2} : Constituée de deux atomes d'azote, c'est une molécule (\text{Diazote}).
\item \ce{O2} : Constituée de deux atomes d'oxygène, c'est une molécule (\text{Dioxygène}).
\item \ce{H2O} : Constituée de deux atomes d'hydrogène et un atome d'oxygène, c'est une molécule (\text{Eau}).
\end{compactitem}

En revanche, \ce{H}, \ce{O}, et \ce{N} sont composés d'un seul atome. Par conséquent, ils correspondent à des éléments chimiques, non des molécules.

Les molécules sont donc : \ce{N2} (Diazote), \ce{O2} (Dioxygène), et \ce{H2O} (Eau).
\end{solution}

\subsection*{Partie 2 – "Regardez le monde défiler"}
  
\textit{Thomas Pesquet a proposé de nombreuses photos et vidéos au cours des six mois passés dans la station spatiale internationale.}

  \question[4] La station spatiale est en mouvement circulaire et uniforme par rapport au centre de la Terre. Thomas Pesquet reste au hublot de la station spatiale pour prendre des photos.
  
  Déterminer si les deux affirmations suivantes sont vraies ou fausses et justifier.
  
  \begin{compactitem}
  \item Affirmation A : Thomas Pesquet est immobile par rapport à la station spatiale.
  \item Affirmation B : Thomas Pesquet est en mouvement par rapport au centre de la Terre.
  \end{compactitem}
  
  

\begin{solution}
\subsection*{Correction}

Analysons les deux affirmations :

\begin{compactitem}
\item \textbf{Affirmation A : Thomas Pesquet est immobile par rapport à la station spatiale.}

Thomas Pesquet est à l'intérieur de la station spatiale et reste au hublot pour prendre des photos. Puisque lui et la station spatiale partagent le même référentiel (la station elle-même), il ne se déplace pas par rapport à celle-ci. 

Cette affirmation est \textbf{vraie}.

\item \textbf{Affirmation B : Thomas Pesquet est en mouvement par rapport au centre de la Terre.}

Bien que Thomas Pesquet soit immobile par rapport à la station spatiale, la station elle-même est en mouvement circulaire et uniforme autour de la Terre. Étant donné qu'il est à bord de cette station, il partage son mouvement autour de la Terre. 

Cette affirmation est \textbf{vraie}.
\end{compactitem}

En conclusion, Thomas Pesquet est immobile par rapport à la station spatiale mais en mouvement par rapport à la Terre.
\end{solution}

\question[4] \textbf{Données :}
  
  \begin{compactitem}
  \item Vitesse moyenne de la station spatiale internationale sur son orbite autour de la Terre : $v = \num{27600} \unit[per-mode = symbol]{\kilo\meter\per\hour}$.
  \item Distance moyenne parcourue par la station spatiale internationale sur son orbite autour de la Terre, pour un tour : $d = \SI{42600}{\kilo\meter}$.
  \item La durée $t$ (en h) nécessaire pour parcourir une distance $d$ (en km) à une vitesse moyenne $v$ (en \unit[per-mode = symbol]{\kilo\meter\per\hour}) s'écrit :
  \[
  t = \frac{d}{v}
  \]
  \end{compactitem}
  
  En 24 heures, la station spatiale internationale réalise plusieurs fois le tour de la Terre : ses occupants peuvent ainsi assister à de nombreux levers et couchers du Soleil.
  
  Montrer, par un calcul, que la durée $t$ nécessaire à la station spatiale internationale pour faire le tour de la Terre vaut environ \SI{1.5}{\hour}, soit \SI{1}{\hour} \SI{30}{\minute}.
  
  

\begin{solution}
\subsection*{Calcul de la durée $t$ pour un tour complet de la Terre}

Pour déterminer la durée nécessaire à la station spatiale internationale pour effectuer un tour complet autour de la Terre, nous devons appliquer la formule :

\[
t = \frac{d}{v}
\]

où 

\begin{addmargin}[4em]{1em}
  \begin{compactitem}
      \item [$t$]: désigne la durée recherchée en heures
      \item [$d$]: représente la distance du tour complet, soit \SI{42600}{\kilo\meter}
      \item [$v$]: indique la vitesse moyenne de la station, soit \num{27600} \unit[per-mode = symbol]{\kilo\meter\per\hour}
  \end{compactitem}
\end{addmargin}

\paragraph{Application numérique} 

Calculons $t$:

\[
t = \frac{42600}{27600}
\]

\[
t \approx 1.543
\]

\paragraph{Conclusion} 
Ainsi, la durée nécessaire à la station spatiale internationale pour faire le tour de la Terre est d'environ \SI{1.5}{\hour}, qui se traduit par \SI{1}{\hour} et \SI{30}{\minute}.
\end{solution}

\question[2] \textbf{Question Bonus} En se basant sur le résultat précédent, combien de fois la station spatiale internationale fait-elle le tour de la terre en une journée complète ?



\begin{solution}
\subsection*{Calcul du nombre de tours en une journée}

Pour déterminer combien de fois la station spatiale internationale fait le tour de la Terre en une journée complète, utilisons la durée d’un tour complet calculée précédemment.

\paragraph{Calcul}
La durée pour un tour complet est de \SI{1.5}{\hour}. Une journée complète contient 24 heures.

Raisonnons en divisant le total des heures dans une journée par la durée nécessaire pour un tour :

\[
n = \frac{24}{1.5}
\]

où
\begin{addmargin}[4em]{1em}
  \begin{compactitem}
      \item [$n$]: nombre de tours complets effectués en 24 heures
      \item [24]: nombre d'heures dans une journée
      \item [1.5]: durée d'un tour complet en heures
  \end{compactitem}
\end{addmargin}

\paragraph{Application numérique} 

\[
n = \frac{24}{1.5} \approx 16
\]

\paragraph{Conclusion} 
Ainsi, la station spatiale internationale effectue environ 16 tours de la Terre en une journée complète.
\end{solution}

\section*{La Salanité de l'eau}

\textit{Extrait du Brevet 2021}
\vspace{1em}

La salinité d’une eau désigne la masse de sel dissous dans un litre de cette eau.
Le tableau suivant donne les caractéristiques de quatre eaux différentes.

\begin{figure}[H]
  \centering
  \includegraphics[width=0.7\linewidth]{img/brevet_02.jpg}
\end{figure}

  \question[3] Parmi les relations suivantes, indiquer celle qui permet de calculer la masse volumique $\rho$. Préciser ce que représentent $m$ et $V$.

  \begin{center}
    \renewcommand{\arraystretch}{2} % augmente l'espacement vertical
    \setlength{\tabcolsep}{20pt}   % augmente l'espacement horizontal
    \begin{tabular}{|c|c|c|}
    \hline
    Relation A & Relation B & Relation C \\
    \hline
    $\rho = \dfrac{m}{V}$ & $\rho = m \times V$ & $\rho = \dfrac{V}{m}$ \\
    \hline
    \end{tabular}
    \end{center}
  
  Pour trouver la masse volumique de l’eau à la surface de l’océan Atlantique Nord, on prélève un échantillon de \SI{50.0}{\milli\liter} de cette eau et on mesure sa masse soit \SI{51.2}{\gram}.
  
  

\begin{solution}
\subsection*{Identification de la relation correcte}

La relation qui permet de calculer la masse volumique $\rho$ est la Relation A :

\[
\rho = \frac{m}{V}
\]

où 

\begin{addmargin}[4em]{1em}
  \begin{compactitem}
    \item [$\rho$]: représente la masse volumique de l'objet, exprimée en grammes par millilitre (\si{\gram\per\milli\liter}).
    \item [$m$]: représente la masse de l'échantillon, exprimée en grammes (\si{\gram}).
    \item [$V$]: représente le volume de l'échantillon, exprimé en millilitres (\si{\milli\liter}).
  \end{compactitem}
\end{addmargin}

\subsection*{Application à l'échantillon d'eau}

Pour l'échantillon de l'eau à la surface de l’océan Atlantique Nord, la masse $m$ est de \SI{51.2}{\gram} et le volume $V$ est de \SI{50.0}{\milli\liter}.

1. \textbf{Raisonnement avec formules mathématiques}

\[
\rho = \frac{m}{V}
\]

2. \textbf{Conversions dans les bonnes unités}

Les valeurs sont déjà exprimées dans les unités adéquates : $\si{\gram}$ et $\si{\milli\liter}$.

3. \textbf{Application numérique}

\[
\rho = \frac{51.2}{50.0} = 1.024
\]

4. \textbf{Conclusion / Interprétation}

La masse volumique de l'eau de l'échantillon prélevé à la surface de l'océan Atlantique Nord est de \SI{1.024}{\gram\per\milli\liter}. Cette valeur indique que l'eau est légèrement plus dense que l'eau douce standard, qui a une masse volumique de \SI{1.00}{\gram\per\milli\liter}.
\end{solution}

\question[4] Calculer la masse volumique de cette eau.
  
  

\begin{solution}

\subsection*{Calcul de la masse volumique de l'eau}

Pour calculer la masse volumique $\rho$ de l'eau à la surface de l'océan Atlantique Nord, nous utilisons la relation suivante :

\[
\rho = \frac{m}{V}
\]

où :

\begin{addmargin}[4em]{1em}
  \begin{compactitem}
    \item [$\rho$] : représente la masse volumique de l'eau, exprimée en \si{\gram\per\milli\liter}.
    \item [$m$] : représente la masse de l'échantillon, exprimée en grammes (\si{\gram}).
    \item [$V$] : représente le volume de l'échantillon, exprimé en millilitres (\si{\milli\liter}).
  \end{compactitem}
\end{addmargin}

L'échantillon d'eau a une masse $m$ de \SI{51.2}{\gram} et un volume $V$ de \SI{50.0}{\milli\liter}. 

1. \textbf{Raisonnement avec formules mathématiques}

\[
\rho = \frac{m}{V}
\]

2. \textbf{Conversions dans les bonnes unités}

Les valeurs de masse et de volume sont déjà dans les unités adéquates (\si{\gram} et \si{\milli\liter}).

3. \textbf{Application numérique}

\[
\rho = \frac{51.2}{50.0} = 1.024
\]

4. \textbf{Conclusion / Interprétation}

Ainsi, la masse volumique de l'eau de cet échantillon est de \SI{1.024}{\gram\per\milli\liter}. Cette valeur indique que l'eau océanique est légèrement plus dense que l'eau douce standard, qui a une masse volumique de \SI{1.00}{\gram\per\milli\liter}.

\end{solution}

\question[2] En exploitant les données du tableau et le résultat de la question 2, indiquer comment la masse volumique évolue en fonction de la salinité.
  
  

\begin{solution}

\subsection*{Évolution de la masse volumique en fonction de la salinité}

En étudiant le tableau fourni, on observe les masses volumiques et les salinités de différentes eaux :

\begin{compactitem}
    \item Eau douce : $\rho = \SI{1.00}{\gram\per\milli\liter}$, salinité nulle.
    \item Eau de l'océan Atlantique Nord : $\rho = \SI{1.024}{\gram\per\milli\liter}$, \SI{35}{\gram} de sel par litre.
    \item Eau de la mer Rouge : $\rho = \SI{1.04}{\gram\per\milli\liter}$, \SI{55}{\gram} de sel par litre.
    \item Eau de la mer Morte : $\rho = \SI{1.24}{\gram\per\milli\liter}$, \SI{200}{\gram} de sel par litre.
\end{compactitem}

\subsection*{Analyse et conclusion}

En analysant les données, nous constatons que la masse volumique augmente avec la salinité. Le tableau montre clairement cette tendance :

\begin{compactitem}
    \item Quand la salinité est nulle (eau douce), la masse volumique est minimale à \SI{1.00}{\gram\per\milli\liter}.
    \item Avec une salinité de \SI{35}{\gram} par litre, la masse volumique augmente à \SI{1.024}{\gram\per\milli\liter}.
    \item Une salinité accrue à \SI{55}{\gram} par litre porte la masse volumique à \SI{1.04}{\gram\per\milli\liter}.
    \item Enfin, avec une très forte salinité de \SI{200}{\gram} par litre, la masse volumique atteint \SI{1.24}{\gram\per\milli\liter}.
\end{compactitem}

Ainsi, nous pouvons conclure que la masse volumique d'une eau augmente à mesure que sa salinité augmente.

\end{solution}

\question[2] Indiquer si la masse volumique d’une eau et sa salinité sont deux grandeurs proportionnelles. Justifier la réponse.


  \end{questions}




\begin{solution}

\subsection*{Proportionalité entre masse volumique et salinité}

Pour déterminer si la masse volumique et la salinité sont deux grandeurs proportionnelles, examinons les données fournies dans le tableau.

\begin{center}
\begin{tabular}{|c|c|c|}
\hline
Salinité (\si{\gram\per\litre}) & Masse volumique (\si{\gram\per\milli\liter}) \\
\hline
0 & 1.00 \\
35 & 1.024 \\
55 & 1.04 \\
200 & 1.24 \\
\hline
\end{tabular}
\end{center}

La proportionalité entre deux grandeurs signifie que le rapport entre elles reste constant. Calculons ce rapport pour plusieurs valeurs.

\[
\text{Rapport} = \frac{\text{Masse volumique}}{\text{Salinité}}
\]

Calculons ces rapports pour les différentes salinités :

\begin{align*}
\text{Pour } 35\, \si{\gram/\liter}, &\quad \frac{1.024}{35} \approx 0.029 \\
\text{Pour } 55\, \si{\gram/\liter}, &\quad \frac{1.04}{55} \approx 0.0189 \\
\text{Pour } 200\, \si{\gram/\liter}, &\quad \frac{1.24}{200} = 0.0062
\end{align*}

\textbf{Conclusion :} Les rapports ne sont pas constants, ce qui signifie que la masse volumique et la salinité ne sont pas proportionnelles. Autrement dit, la masse volumique n'augmente pas linéairement avec la salinité.

\end{solution}

\end{document}
