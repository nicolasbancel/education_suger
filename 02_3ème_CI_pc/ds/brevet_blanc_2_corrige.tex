\documentclass[answers]{exam}
\usepackage{../../mypackages}
\usepackage{../../macros}


\SolutionEmphasis{\color{blue}}
\renewcommand{\solutiontitle}{\noindent}


\title{Brevet blanc N°2 - Physique-Chimie}

\begin{document}

\textbf{Collège Lycée Suger}
\hfill
\textbf{Physique-Chimie} \\

\textbf{Année 2024-2025 - 3ème trimestre}
\hfill
\textbf{Brevet Blanc} \par

{\let\newpage\relax\maketitle}

\begin{center}
\textbf{\textcolor{red}{Durée : 30 minutes. \\
L'usage de la calculatrice avec mode examen actif est autorisé. \\ 
L'usage de la calculatrice sans mémoire, « type collège » est autorisé.}} \\
\textbf{Toute réponse, même incomplète, montrant la démarche de recherche du candidat sera prise en compte dans la notation} \\
\textbf{L'épreuve de Physique comporte 8 questions et est notée sur 25.} \\
\end{center}

\section*{Mission Alpha}

\textit{Extrait du Brevet 2022}
\vspace{1em}

Le 23 avril 2021, l'astronaute français Thomas Pesquet a décollé depuis la base de Cap Canaveral en Floride (USA) pour rejoindre la station spatiale internationale en orbite autour de la Terre, avec trois autres membres d'équipage : c'est la mission Alpha.
L'équipage a rejoint la station spatiale internationale à bord du vaisseau spatial Crew Dragon, lancé par une fusée Falcon 9.

\begin{figure}[H]
  \centering
  \includegraphics[width=0.4\linewidth]{img/brevet_01.jpg}
\end{figure}


\vspace{0.3cm}
\textbf{Les parties 1 et 2 sont indépendantes.}

\subsection*{Partie 1 – L’eau et l’air dans la station spatiale.}

\textit{L’eau et l’air sont nécessaires à la vie des astronautes : leurs besoins sont assurés par différents procédés.}

\begin{questions}

  \question[4] Parmi les formules chimiques ci-dessous, recopier sur la copie les noms de celles qui correspondent à des molécules. Justifier la réponse.
    
  \begin{center}
  \begin{tabular}{lll}
  Diazote : \ce{N2} & Dioxygène : \ce{O2} \\
  Hydrogène : \ce{H} & Oxygène : \ce{O} \\
  Eau : \ce{H2O} & Azote : \ce{N} \\
  \end{tabular}
  \end{center}



\begin{solution}
Pour déterminer quelles formules correspondent à des molécules, il faut comprendre qu'une molécule est une entité électriquement neutre constituée d'au moins deux atomes liés ensemble.

Examinons les formules chimiques proposées :

\begin{compactitem}
    \item \textbf{Diazote : \ce{N2}} 
    \begin{compactitem}
        \item Il s'agit de deux atomes d'azote liés ensemble. Comme c'est constitué de deux atomes, cela correspond bien à une molécule.
    \end{compactitem}
    
    \item \textbf{Dioxygène : \ce{O2}} 
    \begin{compactitem}
        \item Deux atomes d'oxygène liés ensemble forment aussi une molécule.
    \end{compactitem}

    \item \textbf{Hydrogène : \ce{H}} 
    \begin{compactitem}
        \item Ce n'est qu'un seul atome non lié, donc ce n'est pas une molécule.
    \end{compactitem}

    \item \textbf{Oxygène : \ce{O}} 
    \begin{compactitem}
        \item Ceci est également un seul atome non lié, donc pas une molécule.
    \end{compactitem}
    
    \item \textbf{Eau : \ce{H2O}} 
    \begin{compactitem}
        \item Composée de deux atomes d'hydrogène et un atome d'oxygène, c'est manifestement une molécule.
    \end{compactitem}

    \item \textbf{Azote : \ce{N}} 
    \begin{compactitem}
        \item Un seul atome d'azote, ce n'est pas une molécule.
    \end{compactitem}
\end{compactitem}

En conclusion, les formules qui correspondent à des molécules sont \ce{N2}, \ce{O2}, et \ce{H2O}.
\end{solution}

\subsection*{Partie 2 – "Regardez le monde défiler"}
  
\textit{Thomas Pesquet a proposé de nombreuses photos et vidéos au cours des six mois passés dans la station spatiale internationale.}

  \question[4] La station spatiale est en mouvement circulaire et uniforme par rapport au centre de la Terre. Thomas Pesquet reste au hublot de la station spatiale pour prendre des photos.
  
  Déterminer si les deux affirmations suivantes sont vraies ou fausses et justifier.
  
  \begin{compactitem}
  \item Affirmation A : Thomas Pesquet est immobile par rapport à la station spatiale.
  \item Affirmation B : Thomas Pesquet est en mouvement par rapport au centre de la Terre.
  \end{compactitem}
  
  

\begin{solution}
Pour analyser ces affirmations, examinons les concepts de mouvement relatif :

\begin{compactitem}
    \item \textbf{Affirmation A : Thomas Pesquet est immobile par rapport à la station spatiale.}
    
    Cette affirmation est \textbf{vraie}. Thomas Pesquet, étant à l'intérieur de la station spatiale et ne se déplaçant pas par rapport à elle (puisqu'il reste au hublot pour prendre des photos), est immobile relativement à la station spatiale. Toute personne ou objet à l'intérieur de la station sans mouvement propre est immobile par rapport à celle-ci.

    \item \textbf{Affirmation B : Thomas Pesquet est en mouvement par rapport au centre de la Terre.}

    Cette affirmation est \textbf{vraie}. La station spatiale est en mouvement circulaire et uniforme autour de la Terre, ce qui signifie qu'elle se déplace à une vitesse constante le long d'une trajectoire circulaire centrée sur la Terre. Puisque Thomas Pesquet est à bord de cette station, il partage ce mouvement. Ainsi, par rapport au centre de la Terre, Thomas Pesquet est en mouvement.
\end{compactitem}

En conclusion, les deux affirmations sont correctes lorsqu'on considère la notion de mouvement relatif.
\end{solution}

\question[4] \textbf{Données :}
  
  \begin{compactitem}
  \item Vitesse moyenne de la station spatiale internationale sur son orbite autour de la Terre : $v = \num{27600} \unit[per-mode = symbol]{\kilo\meter\per\hour}$.
  \item Distance moyenne parcourue par la station spatiale internationale sur son orbite autour de la Terre, pour un tour : $d = \SI{42600}{\kilo\meter}$.
  \item La durée $t$ (en h) nécessaire pour parcourir une distance $d$ (en km) à une vitesse moyenne $v$ (en \unit[per-mode = symbol]{\kilo\meter\per\hour}) s'écrit :
  \[
  t = \frac{d}{v}
  \]
  \end{compactitem}
  
  En 24 heures, la station spatiale internationale réalise plusieurs fois le tour de la Terre : ses occupants peuvent ainsi assister à de nombreux levers et couchers du Soleil.
  
  Montrer, par un calcul, que la durée $t$ nécessaire à la station spatiale internationale pour faire le tour de la Terre vaut environ \SI{1.5}{\hour}, soit \SI{1}{\hour} \SI{30}{\minute}.
  
  

\begin{solution}
Pour déterminer la durée $t$ nécessaire à la station spatiale internationale pour faire le tour de la Terre, nous utilisons la formule donnée :
\[
t = \frac{d}{v}
\]
où

\begin{addmargin}[4em]{1em}
    \begin{compactitem}
        \item [$t$] : durée nécessaire (en heures)
        \item [$d$] : distance parcourue pour un tour (\SI{42600}{\kilo\meter})
        \item [$v$] : vitesse moyenne de la station (\num{27600} \unit[per-mode = symbol]{\kilo\meter\per\hour})
    \end{compactitem}
\end{addmargin}

1. **Raisonnement mathématique**  
   On utilise la formule :  
   \[
   t = \frac{d}{v}
   \]

2. **Application numérique**  
   \[
   t = \frac{42600}{27600} \approx 1.543
   \]

3. **Conclusion**  
   La durée $t$ est d'environ \SI{1.543}{\hour}, ce qui est d'environ \SI{1}{\hour} \SI{30}{\minute}. Ceci est en accord avec la valeur donnée dans l'énoncé.
\end{solution}

\question[2] \textbf{Question Bonus} En se basant sur le résultat précédent, combien de fois la station spatiale internationale fait-elle le tour de la terre en une journée complète ?



\begin{solution}
Pour déterminer combien de fois la station spatiale internationale fait le tour de la Terre en une journée, nous utiliserons la durée $t$ nécessaire pour un tour de la Terre, trouvée précédemment.

1. **Raisonnement mathématique**  
   On connaît la durée nécessaire pour un tour :
   \[
   t = \SI{1.5}{\hour}
   \]
   La durée d'une journée complète est de \SI{24}{\hour}. Le nombre de tours $n$ que la station spatiale réalise en une journée est donné par :
   \[
   n = \frac{\text{durée d'une journée}}{t}
   \]

2. **Application numérique**  
   \[
   n = \frac{24}{1.5} = 16
   \]

3. **Conclusion**  
   La station spatiale internationale réalise environ 16 tours de la Terre en une journée complète.
\end{solution}

\section*{La Salanité de l'eau}

\textit{Extrait du Brevet 2021}
\vspace{1em}

La salinité d’une eau désigne la masse de sel dissous dans un litre de cette eau.
Le tableau suivant donne les caractéristiques de quatre eaux différentes.

\begin{figure}[H]
  \centering
  \includegraphics[width=0.7\linewidth]{img/brevet_02.jpg}
\end{figure}

  \question[3] Parmi les relations suivantes, indiquer celle qui permet de calculer la masse volumique $\rho$. Préciser ce que représentent $m$ et $V$.

  \begin{center}
    \renewcommand{\arraystretch}{2} % augmente l'espacement vertical
    \setlength{\tabcolsep}{20pt}   % augmente l'espacement horizontal
    \begin{tabular}{|c|c|c|}
    \hline
    Relation A & Relation B & Relation C \\
    \hline
    $\rho = \dfrac{m}{V}$ & $\rho = m \times V$ & $\rho = \dfrac{V}{m}$ \\
    \hline
    \end{tabular}
    \end{center}
  
  Pour trouver la masse volumique de l’eau à la surface de l’océan Atlantique Nord, on prélève un échantillon de \SI{50.0}{\milli\liter} de cette eau et on mesure sa masse soit \SI{51.2}{\gram}.
  
  

\begin{solution}
La relation qui permet de calculer la masse volumique $\rho$ est la suivante :

\[
\rho = \frac{m}{V}
\]

où 

\begin{addmargin}[4em]{1em}
\begin{compactitem}
    \item [$\rho$]: représente la masse volumique, exprimée en \(\si{\kilo\gram\per\cubic\meter}\) ou \(\si{\gram\per\milli\liter}\).
    \item [$m$]: représente la masse de l'échantillon, exprimée en \(\si{\kilo\gram}\) ou \(\si{\gram}\).
    \item [$V$]: représente le volume de l'échantillon, exprimé en \(\si{\cubic\meter}\) ou \(\si{\milli\liter}\).
\end{compactitem}
\end{addmargin}

Pour calculer la masse volumique de l'eau de l'océan Atlantique Nord :

1. \textbf{Raisonnement avec formules mathématiques} \\
   En utilisant la formule de la masse volumique, nous avons :
   \[
   \rho = \frac{m}{V}
   \]

2. \textbf{Conversions dans les bonnes unités} \\
   La masse $m$ est donnée en \(\SI{51.2}{\gram}\) et le volume $V$ en \(\SI{50.0}{\milli\liter}\).

3. \textbf{Application numérique} \\
   \[
   \rho = \frac{51.2}{50.0} = 1.024
   \]

4. \textbf{Conclusion / Interprétation} \\
   La masse volumique de l'eau de l'océan Atlantique Nord est \(\SI{1.024}{\gram\per\milli\liter}\).
\end{solution}

\question[4] Calculer la masse volumique de cette eau.
  
  

\begin{solution}
Nous avons déjà calculé la masse volumique de l'eau à la surface de l'océan Atlantique Nord dans la question précédente. Reprenons les données et le raisonnement pour confirmer ce calcul :

1. \textbf{Raisonnement avec formules mathématiques} \\
   La relation utilisée pour la masse volumique est :
   \[
   \rho = \frac{m}{V}
   \]
   où
   \begin{addmargin}[4em]{1em}
   \begin{compactitem}
       \item [$\rho$]: la masse volumique, exprimée en \(\si{\gram\per\milli\liter}\).
       \item [$m$]: la masse de l'échantillon en \(\si{\gram}\).
       \item [$V$]: le volume de l'échantillon en \(\si{\milli\liter}\).
   \end{compactitem}
   \end{addmargin}

2. \textbf{Conversions dans les bonnes unités} \\
   La masse $m$ est \(\SI{51.2}{\gram}\) et le volume $V$ est \(\SI{50.0}{\milli\liter}\).

3. \textbf{Application numérique} \\
   \[
   \rho = \frac{51.2}{50.0} = 1.024
   \]
   
4. \textbf{Conclusion / Interprétation} \\
   La masse volumique de cette eau est \(\SI{1.024}{\gram\per\milli\liter}\).
\end{solution}

\question[2] En exploitant les données du tableau et le résultat de la question 2, indiquer comment la masse volumique évolue en fonction de la salinité.
  
  

\begin{solution}
En analysant les données du tableau ainsi que le résultat de la question 2, nous pouvons observer comment la masse volumique évolue en fonction de la salinité.

1. **Observation des données :**
   \begin{compactitem}
      \item L'eau douce a une masse volumique de \SI{1.00}{\gram\per\milli\liter} et une salinité nulle.
      \item L'eau à la surface de la mer Rouge a une masse volumique de \SI{1.04}{\gram\per\milli\liter} avec une salinité de \SI{55}{\gram\per\liter}.
      \item L'eau à la surface de la mer Morte a une masse volumique de \SI{1.24}{\gram\per\milli\liter} avec une salinité de \SI{200}{\gram\per\liter}.
   \end{compactitem}

2. **Analyse :**
   \begin{compactitem}
      \item On observe que plus la salinité augmente, plus la masse volumique de l'eau augmente.
   \end{compactitem}

3. **Conclusion :**
   L'évolution montre une relation positive entre la masse volumique de l'eau et la salinité : à mesure que la quantité de sel dissous par litre d'eau augmente, la masse volumique de cette eau augmente également.
\end{solution}

\question[2] Indiquer si la masse volumique d’une eau et sa salinité sont deux grandeurs proportionnelles. Justifier la réponse.


  \end{questions}




\begin{solution}
Pour déterminer si la masse volumique d'une eau et sa salinité sont deux grandeurs proportionnelles, analysons les données fournies dans le tableau.

1. **Observation des données :**
   \begin{compactitem}
      \item Eau douce : Masse volumique = \SI{1.00}{\gram\per\milli\liter}, Salinité = Nulle
      \item Eau de l'océan Atlantique Nord : Salinité = \SI{35}{\gram\per\liter}, Masse volumique = \SI{1.024}{\gram\per\milli\liter}
      \item Eau de la mer Rouge : Salinité = \SI{55}{\gram\per\liter}, Masse volumique = \SI{1.04}{\gram\per\milli\liter}
      \item Eau de la mer Morte : Salinité = \SI{200}{\gram\per\liter}, Masse volumique = \SI{1.24}{\gram\per\milli\liter}
   \end{compactitem}

2. **Analyse :**
   \begin{compactitem}
      \item Une relation de proportionnalité implique que le rapport entre deux grandeurs doit être constant.
      \item Calculons le rapport masse volumique/salinité pour chaque eau où la salinité est non nulle :
        \begin{align*}
        \text{Eau Atlantique Nord: } & \frac{\SI{1.024}{}}{\SI{35}{}} \approx \SI{0.0293}{} \\
        \text{Eau Mer Rouge: } & \frac{\SI{1.04}{}}{\SI{55}{}} \approx \SI{0.0189}{} \\
        \text{Eau Mer Morte: } & \frac{\SI{1.24}{}}{\SI{200}{}} \approx \SI{0.0062}{}
        \end{align*}
   \end{compactitem}

3. **Conclusion :**
   \begin{compactitem}
      \item Les rapports ne sont pas constants, ce qui signifie que la masse volumique de l'eau et sa salinité ne sont pas proportionnelles.
   \end{compactitem}
\end{solution}

\end{document}
