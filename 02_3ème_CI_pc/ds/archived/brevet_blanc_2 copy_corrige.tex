\documentclass[answers]{exam}
\usepackage{../../mypackages}
\usepackage{../../macros}


\SolutionEmphasis{\color{blue}}
\renewcommand{\solutiontitle}{\noindent}


\title{Brevet blanc N°2 - Physique-Chimie}

\begin{document}

\textbf{Collège Lycée Suger}
\hfill
\textbf{Physique-Chimie} \\

\textbf{Année 2024-2025 - 3ème trimestre}
\hfill
\textbf{Brevet Blanc} \par

{\let\newpage\relax\maketitle}

\begin{center}
\textbf{\textcolor{red}{Durée : 30 minutes. \\
L'usage de la calculatrice avec mode examen actif est autorisé. \\ 
L'usage de la calculatrice sans mémoire, « type collège » est autorisé.}} \\
\textbf{Toute réponse, même incomplète, montrant la démarche de recherche du candidat sera prise en compte dans la notation} \\
\textbf{L'épreuve de Physique comporte 8 questions et est notée sur 25.} \\
\end{center}

\section*{Mission Alpha}

\textit{Extrait du Brevet 2022}
\vspace{1em}

Le 23 avril 2021, l'astronaute français Thomas Pesquet a décollé depuis la base de Cap Canaveral en Floride (USA) pour rejoindre la station spatiale internationale en orbite autour de la Terre, avec trois autres membres d'équipage : c'est la mission Alpha.
L'équipage a rejoint la station spatiale internationale à bord du vaisseau spatial Crew Dragon, lancé par une fusée Falcon 9.

\begin{figure}[H]
  \centering
  \includegraphics[width=0.4\linewidth]{img/brevet_01.jpg}
\end{figure}


\vspace{0.3cm}
\textbf{Les parties 1 et 2 sont indépendantes.}

\subsection*{Partie 1 – L’eau et l’air dans la station spatiale.}

\textit{L’eau et l’air sont nécessaires à la vie des astronautes : leurs besoins sont assurés par différents procédés.}

\begin{questions}

  \question[4] Parmi les formules chimiques ci-dessous, recopier sur la copie les noms de celles qui correspondent à des molécules. Justifier la réponse.
    
  \begin{center}
  \begin{tabular}{lll}
  Diazote : \ce{N2} & Dioxygène : \ce{O2} \\
  Hydrogène : \ce{H} & Oxygène : \ce{O} \\
  Eau : \ce{H2O} & Azote : \ce{N} \\
  \end{tabular}
  \end{center}



\begin{solution}

    \subsection*{Justification des molécules}

    Une molécule est formée par l'association d'au moins deux atomes. Examinons chacune des formules pour déterminer lesquelles correspondent à des molécules : 

    \begin{compactitem}
      \item \ce{N2} (Diazote) : Composée de deux atomes d'azote, c'est une molécule.
      \item \ce{O2} (Dioxygène) : Composée de deux atomes d'oxygène, c'est une molécule.
      \item \ce{H} (Hydrogène) : Atome simple, ce n'est pas une molécule.
      \item \ce{O} (Oxygène) : Atome simple, ce n'est pas une molécule.
      \item \ce{H2O} (Eau) : Composée de deux atomes d'hydrogène et un atome d'oxygène, c'est une molécule.
      \item \ce{N} (Azote) : Atome simple, ce n'est pas une molécule.
    \end{compactitem}

    Par conséquent, les formules chimiques qui correspondent à des molécules sont : \ce{N2}, \ce{O2}, et \ce{H2O}.

\end{solution}

\subsection*{Partie 2 – "Regardez le monde défiler"}
  
\textit{Thomas Pesquet a proposé de nombreuses photos et vidéos au cours des six mois passés dans la station spatiale internationale.}

  \question[4] La station spatiale est en mouvement circulaire et uniforme par rapport au centre de la Terre. Thomas Pesquet reste au hublot de la station spatiale pour prendre des photos.
  
  Déterminer si les deux affirmations suivantes sont vraies ou fausses et justifier.
  
  \begin{compactitem}
  \item Affirmation A : Thomas Pesquet est immobile par rapport à la station spatiale.
  \item Affirmation B : Thomas Pesquet est en mouvement par rapport au centre de la Terre.
  \end{compactitem}
  
  

\begin{solution}

\subsection*{Analyse des affirmations}

Pour déterminer si les affirmations sont vraies ou fausses, nous devons examiner les référentiels impliqués.

\begin{compactitem}
    \item \textbf{Affirmation A :} Thomas Pesquet est immobile par rapport à la station spatiale.

    \begin{compactitem}
        \item Dans le référentiel de la station spatiale, Thomas Pesquet ne se déplace pas car il reste près du hublot. Il ne change donc pas de position à l'intérieur de la station. 
        \item Conclusion : Cette affirmation est \textbf{vraie}.
    \end{compactitem}

    \item \textbf{Affirmation B :} Thomas Pesquet est en mouvement par rapport au centre de la Terre.

    \begin{compactitem}
        \item La station spatiale est en mouvement circulaire et uniforme autour de la Terre. Par conséquent, tout objet ou personne à bord, y compris Thomas Pesquet, est également en mouvement par rapport au centre de la Terre.
        \item Conclusion : Cette affirmation est \textbf{vraie}.
    \end{compactitem}
\end{compactitem}

\end{solution}

\question[4] \textbf{Données :}
  
  \begin{compactitem}
  \item Vitesse moyenne de la station spatiale internationale sur son orbite autour de la Terre : $v = \num{27600} \unit[per-mode = symbol]{\kilo\meter\per\hour}$.
  \item Distance moyenne parcourue par la station spatiale internationale sur son orbite autour de la Terre, pour un tour : $d = \SI{42600}{\kilo\meter}$.
  \item La durée $t$ (en h) nécessaire pour parcourir une distance $d$ (en km) à une vitesse moyenne $v$ (en \unit[per-mode = symbol]{\kilo\meter\per\hour}) s'écrit :
  \[
  t = \frac{d}{v}
  \]
  \end{compactitem}
  
  En 24 heures, la station spatiale internationale réalise plusieurs fois le tour de la Terre : ses occupants peuvent ainsi assister à de nombreux levers et couchers du Soleil.
  
  Montrer, par un calcul, que la durée $t$ nécessaire à la station spatiale internationale pour faire le tour de la Terre vaut environ \SI{1.5}{\hour}, soit \SI{1}{\hour} \SI{30}{\minute}.
  
  

\begin{solution}

\subsection*{Calcul de la durée nécessaire pour un tour de la Terre}

Pour calculer le temps $t$ nécessaire à la station spatiale internationale pour faire un tour complet de la Terre, nous utilisons la formule :

\[
t = \frac{d}{v}
\]

où 

\begin{addmargin}[4em]{1em}
  \begin{compactitem}
    \item [\(t\)] : temps nécessaire pour un tour (en heures),
    \item [\(d\)] : distance parcourue pour un tour complet = \SI{42600}{\kilo\meter},
    \item [\(v\)] : vitesse moyenne de la station = \SI{27600}{\kilo\meter\per\hour}.
  \end{compactitem}
\end{addmargin}

\subsection*{Calcul}

1. **Raisonnement avec formules :**

\[
t = \frac{42600}{27600}
\]

2. **Application numérique :**

\[
t \approx 1.543
\]

3. **Conclusion / Interprétation :**

La durée nécessaire pour que la station spatiale internationale fasse un tour complet de la Terre est d'environ \SI{1.5}{\hour}, soit \SI{1}{\hour} \SI{30}{\minute}, ce qui correspond bien à l'énoncé du problème. 

\end{solution}

\question[2] \textbf{Question Bonus} En se basant sur le résultat précédent, combien de fois la station spatiale internationale fait-elle le tour de la terre en une journée complète ?



\begin{solution}

\subsection*{Calcul du nombre de tours par jour}

Pour déterminer combien de fois la Station Spatiale Internationale fait le tour de la Terre en une journée complète, nous devons utiliser le temps nécessaire pour un tour complet obtenu précédemment.

\subsection*{Données:}
\begin{compactitem}
    \item Durée d'un tour complet: \( t = \SI{1.5}{\hour} \).
    \item Durée d'une journée complète: \SI{24}{\hour}.
\end{compactitem}

\subsection*{Formule et calcul:}

1. **Raisonnement avec formules :**
   \[
   N = \frac{T_{\text{jour}}}{t}
   \]
   où
   \begin{addmargin}[4em]{1em}
     \begin{compactitem}
         \item [\(N\)]: nombre de tours par jour,
         \item [\(T_{\text{jour}}\)]: durée d'une journée = \SI{24}{\hour},
         \item [\(t\)]: durée pour un tour = \SI{1.5}{\hour}.
     \end{compactitem}
   \end{addmargin}

2. **Application numérique :**
   \[
   N = \frac{24}{1.5} = 16
   \]

3. **Conclusion / Interprétation :** 

La Station Spatiale Internationale fait environ 16 fois le tour de la Terre en une journée complète de 24 heures.

\end{solution}

\section*{La Salanité de l'eau}

\textit{Extrait du Brevet 2021}
\vspace{1em}

La salinité d’une eau désigne la masse de sel dissous dans un litre de cette eau.
Le tableau suivant donne les caractéristiques de quatre eaux différentes.

\begin{figure}[H]
  \centering
  \includegraphics[width=0.7\linewidth]{img/brevet_02.jpg}
\end{figure}

  \question[3] Parmi les relations suivantes, indiquer celle qui permet de calculer la masse volumique $\rho$. Préciser ce que représentent $m$ et $V$.

  \begin{center}
    \renewcommand{\arraystretch}{2} % augmente l'espacement vertical
    \setlength{\tabcolsep}{20pt}   % augmente l'espacement horizontal
    \begin{tabular}{|c|c|c|}
    \hline
    Relation A & Relation B & Relation C \\
    \hline
    $\rho = \dfrac{m}{V}$ & $\rho = m \times V$ & $\rho = \dfrac{V}{m}$ \\
    \hline
    \end{tabular}
    \end{center}
  
  Pour trouver la masse volumique de l’eau à la surface de l’océan Atlantique Nord, on prélève un échantillon de \SI{50.0}{\milli\liter} de cette eau et on mesure sa masse soit \SI{51.2}{\gram}.
  
  

\begin{solution}
Pour déterminer la relation correcte permettant de calculer la masse volumique $\rho$, examinons les relations proposées :

\begin{compactitem}
    \item Relation A : $\rho = \dfrac{m}{V}$
    \item Relation B : $\rho = m \times V$
    \item Relation C : $\rho = \dfrac{V}{m}$
\end{compactitem}

La relation correcte est la Relation A : $\rho = \dfrac{m}{V}$, où :

\begin{compactitem}
    \item $m$ représente la masse de l'échantillon, en grammes (\SI{}{\gram}).
    \item $V$ représente le volume de l'échantillon, en millilitres (\SI{}{\milli\liter}).
\end{compactitem}

La masse volumique est définie comme la masse par unité de volume, ce qui correspond à l'expression $\rho = \frac{m}{V}$.

\subsection*{Application à l'échantillon d'eau}

L'échantillon d'eau de l'océan Atlantique Nord a une masse de \SI{51.2}{\gram} et un volume de \SI{50.0}{\milli\liter}. 

\begin{align*}
    \rho &= \frac{m}{V} \\
    &= \frac{51.2}{50.0}
\end{align*}

En simplifiant, nous trouvons :

\[
\rho = \SI{1.024}{\gram\per\milli\liter}
\]

La masse volumique de l'eau à la surface de l'océan Atlantique Nord est donc de \SI{1.024}{\gram\per\milli\liter}.
\end{solution}

\question[4] Calculer la masse volumique de cette eau.
  
  

\begin{solution}
Pour calculer la masse volumique de cette eau, nous utiliserons la relation :

\[
\rho = \frac{m}{V}
\]

où 

\begin{addmargin}[4em]{1em}
\begin{compactitem}
    \item [$\rho$]: représente la masse volumique de l'eau
    \item [$m$]: représente la masse de l'échantillon, en grammes (\SI{}{\gram})
    \item [$V$]: représente le volume de l'échantillon, en millilitres (\SI{}{\milli\liter})
\end{compactitem}
\end{addmargin}

Pour l'échantillon d'eau de l'océan Atlantique Nord, nous avons :
- $m = \SI{51.2}{\gram}$
- $V = \SI{50.0}{\milli\liter}$

Calculons la masse volumique :

\begin{align*}
    \rho &= \frac{m}{V} \\
    &= \frac{51.2}{50.0} 
\end{align*}

L'application numérique donne :

\[
\rho = \SI{1.024}{\gram\per\milli\liter}
\]

En conclusion, la masse volumique de l'eau à la surface de l'océan Atlantique Nord est de \SI{1.024}{\gram\per\milli\liter}.
\end{solution}

\question[2] En exploitant les données du tableau et le résultat de la question 2, indiquer comment la masse volumique évolue en fonction de la salinité.
  
  

\begin{solution}
Pour comprendre comment la masse volumique évolue en fonction de la salinité, observons les données du tableau :

\begin{center}
\begin{tabular}{|l|c|c|c|c|}
\hline
& Eau douce & Eau de l'Atlantique Nord & Eau de la mer Rouge & Eau de la mer Morte \\
\hline
Masse volumique (\si{\gram\per\milli\liter}) & 1.00 & 1.024 & 1.04 & 1.24 \\
\hline
Salinité (g de sel/litre) & Nulle & 35 & 55 & 200 \\
\hline
\end{tabular}
\end{center}

Analysons ces valeurs :

\begin{compactitem}
    \item L'eau douce a une salinité nulle et une masse volumique de \SI{1.00}{\gram\per\milli\liter}.
    \item Pour l'eau de l'Atlantique Nord, la salinité est de \SI{35}{\gram} de sel par litre, et sa masse volumique est de \SI{1.024}{\gram\per\milli\liter}.
    \item Pour l'eau de la mer Rouge, la salinité est de \SI{55}{\gram} de sel par litre avec une masse volumique de \SI{1.04}{\gram\per\milli\liter}.
    \item Pour l'eau de la mer Morte, la salinité est de \SI{200}{\gram} de sel par litre et la masse volumique atteint \SI{1.24}{\gram\per\milli\liter}.
\end{compactitem}

Conclusion :

On observe que la masse volumique de l'eau augmente avec l'augmentation de la salinité. Autrement dit, plus il y a de sel dissous dans l'eau, plus l'eau est dense.
\end{solution}

\question[2] Indiquer si la masse volumique d’une eau et sa salinité sont deux grandeurs proportionnelles. Justifier la réponse.


  \end{questions}




\begin{solution}
Pour déterminer si la masse volumique d'une eau et sa salinité sont des grandeurs proportionnelles, examinons les données du tableau et analysons leur comportement.

\subsection*{Analyse des données}
Observons les valeurs fournies :

\begin{center}
\begin{tabular}{|l|c|c|c|c|}
\hline
& Eau douce & Eau de l'Atlantique Nord & Eau de la mer Rouge & Eau de la mer Morte \\
\hline
Masse volumique (\si{\gram\per\milli\liter}) & 1,00 & 1,024 & 1,04 & 1,24 \\
\hline
Salinité (g de sel/litre) & Nulle & 35 & 55 & 200 \\
\hline
\end{tabular}
\end{center}

\subsection*{Proportionnalité}
Deux grandeurs sont proportionnelles si le rapport de l'une à l'autre est constant. Calculons les rapports correspondants :

\begin{align*}
\text{Eau de l'Atlantique Nord :} & \quad \frac{1.024}{35} \approx 0.0293 \\
\text{Eau de la mer Rouge :} & \quad \frac{1.04}{55} \approx 0.0189 \\
\text{Eau de la mer Morte :} & \quad \frac{1.24}{200} \approx 0.0062
\end{align*}

\subsection*{Conclusion}
Les rapports ne sont pas constants. Cela indique que les grandeurs ne sont pas proportionnelles. La relation entre la masse volumique et la salinité n'est donc pas linéaire, et la masse volumique n'augmente pas de manière proportionnelle avec l'augmentation de la salinité.
\end{solution}

\end{document}
