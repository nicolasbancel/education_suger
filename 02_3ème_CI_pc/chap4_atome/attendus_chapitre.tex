\documentclass[a4paper,12pt]{article}
\usepackage{../../mypackages}
\usepackage{../../macros}


\begin{document}

\title{Chapitre 4 : L'atome - Les attendus du chapitre}
\author{N. Bancel}
\date{Novembre 2024}
\maketitle


\begin{tcolorbox}[colback=blue!10!white, colframe=blue!75!black, title=A savoir]
  \begin{itemize}[noitemsep]
    \item Connaître les couleurs associées à chaque type d'atome 
    \item Connaître le symbole des atomes, la définition du numéro atomique, et du nombre de masse 
    \item Connaître sur le bout des doigts la différence entre un proton, un neutron, un électron, et ce que signifie un nucléon
    \item Savoir effectuer un rapport de proportionnalité (De combien de fois un nucléon a une masse plus importante qu'un électron ? De combien de fois le volume de l'atome est plus élevé que celui du noyau ?)
    \item Faire la distinction entre un atome et une molécule 
    \item Etre capable de lister le type et le nombre d'atomes qui constituent une molécule 
    \item Etre capable de déterminer si une équation chimique est équilibrée ou non
    \item Comprendre et définir la notion de conservation de la masse 
    \item Identifier les réactifs et les produits d'une réaction chimique
    \item Savoir reconnaître les molécules typiques 
    \begin{itemize}
      \item Dioxygène : \ce{O2}
      \item Dioxyde de carbone : \ce{CO2}
      \item Eau : \ce{H2O}
      \item Diohydrogène : \ce{H2}
    \end{itemize}
    \item Ecrire une équation chimique "en toutes lettres" (sans se soucier des formules des molécules : simplement en nommant les espèces chimiques)
  \end{itemize}
\end{tcolorbox}

\begin{tcolorbox}[colback=red!10!white, colframe=red!75!black, title=Les choses qui sont considérées comme acquises]
  Des questions peuvent tomber sur ces sujets, même s'ils ne portent pas directement sur le chapitre
  \begin{itemize}[noitemsep]
    \item La formule de la masse volumique 
    \item Les 2 autres formules qui découlent de la formule de la masse volumique ((1) comment trouver la masse quand on a la masse volumique et le volume (2) comment trouver le volume quand on la masse volumique et la masse)
  \end{itemize}
\end{tcolorbox}

\begin{tcolorbox}[colback=red!10!white, colframe=red!75!black, title=Les choses à faire obligatoirement]
  \begin{itemize}[noitemsep]
    \item Rédiger : une réponse donnée sans justification sera comptée fausse.
    \item Autant que possible : distinguer dans le raisonnement 3 parties : (1) Raisonnement théorique avec des \textbf{formules} (2) Application numérique (on effectue le calcul) (3) Conclusion
    \item Quand on montre qu'une équation est équilibrée ou pas : faire un tableau pour compter les quantités de chaque atome à gauche et à droite 
  \end{itemize}
\end{tcolorbox}


\begin{tcolorbox}[colback=gray!10!white, colframe=gray!75!black, title=Les erreurs à ne pas faire]
  \begin{itemize}[noitemsep]
    \item \ce{6 CO2} n'est pas un réactif. C'est la molécule de \ce{CO2} qui est un réactif. Sa quantité n'importe pas au moment d'identifier les réactifs et les produits 
    \item Dans la formule d'une molécule, on ne met pas de + : \ce{CO2}, pas $C + O + O$
  \end{itemize}
\end{tcolorbox}

\end{document}
