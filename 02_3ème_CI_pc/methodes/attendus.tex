\documentclass[12pt]{article}
\usepackage{../../mypackages}
\usepackage{../../macros}

\titleformat{\section}{\large\bfseries}{}{0em}{}

\title{\vspace{-2cm}Rappels importants pour les devoirs surveillés en physique}

\title{Attendus pour les devoirs surveillés en physique}
\author{N. Bancel}
\date{31 Mars 2025}

\begin{document}

\textbf{Collège Lycée Suger}
\hfill
\textbf{Physique-Chimie} \\

\textbf{Année 2024-2025 - 3ème trimestre}
\hfill
\textbf{3ème CI} \par

{\let\newpage\relax\maketitle}

\section*{1. Utiliser les unités du Système International (SI)}
Toutes les réponses numériques doivent comporter une unité en unités SI. 

\textbf{Exemple : calcul du poids}
\begin{itemize}[label=--]
  \item \textbf{Faux :} $P = 500 \times 10 = 5000$ \textit{(sans unité ou en grammes)}
  \item \textbf{Juste :} $P = 50 \, \si{kg} \times 10 \, \si{\newton\per\kilogram} = 500 \, \si{\newton}$
\end{itemize}

\section*{2. Lire les énoncés attentivement}
Avant de répondre :
\begin{itemize}[label=--]
  \item Repérer les données, les unités, les formules utiles.
  \item Respecter les attendus de forme : explications claires, unités, encadrer les résultats.
\end{itemize}

\section*{3. Écriture correcte des unités}
\begin{itemize}[label=--]
  \item \si{\meter\per\second} s'écrit aussi \si{\meter\cdot\second^{-1}}
  \item Ces deux écritures sont équivalentes car elles désignent la même grandeur physique (division par le temps).
\end{itemize}

\section*{4. Schémas de forces}
Un bon schéma de force doit :
\begin{itemize}[label=--]
  \item Être réalisé avec une règle graduée.
  \item Avoir le point d'application clairement placé.
  \item Montrer des flèches orientées avec une longueur proportionnelle.
  \item Indiquer le nom de la force (\og $\vec{P}$\fg{}, $\vec{R}$, etc.)
\end{itemize}

\section*{5. Manipulations d'équations}
Savoir isoler une variable :
\begin{itemize}[label=--]
  \item \textbf{Formule de base :} $P = m \times g$
  \item \textbf{On cherche $m$ :} \quad $m = \dfrac{P}{g}$
\end{itemize}

\section*{6. Différencier les notions proches}
\begin{itemize}[label=--]
  \item \textbf{Intensité de pesanteur $g$} : \si{\newton\per\kilogram} (valeur sur Terre \~10).
  \item \textbf{Constante gravitationnelle $G$} : \si{6.67e-11\newton\meter\squared\per\kilogram\squared}, utilisée pour calculer la force gravitationnelle universelle.
  \item \textbf{Force gravitationnelle $F$} : force entre deux objets massiques, formule $F = G \cdot \dfrac{m_1 m_2}{d^2}$
\end{itemize}

\section*{7. Exemple de rédaction attendue pour une formule}

\textit{L'énergie cinétique correspond à l'énergie que possède un objet en mouvement.}

\[
E_c = \frac{1}{2} \times m \times v^2
\]

\begin{adjustwidth}{4em}{1em}
\begin{itemize}[label=--, leftmargin=1.5em]
  \item[$E_c$] : énergie cinétique (en \si{\joule})
  \item[$m$] : masse de l’objet (en \si{\kilogram})
  \item[$v$] : vitesse de l’objet (en \si{\meter\per\second})
\end{itemize}
\end{adjustwidth}

\end{document}
