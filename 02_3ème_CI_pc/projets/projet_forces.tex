\documentclass[a4paper,12pt]{article}
\usepackage{../../mypackages}
\usepackage{../../macros}

\setlength{\parindent}{0pt}


\begin{document}

\title{Projet - Mouvement et modélisation de forces en action}
\author{N. Bancel}
\date{Février 2025}
\maketitle

\section*{Présentation}

Le but de ce projet est de concevoir un dispositif qui permet de transporter un objet
(ex. une figurine, un petit véhicule, une masse) d'un point A à un point B en utilisant un moyen mécanique décidé à l'avance.
\textbf{Le mouvement peut être vertical ou horizontal.}

\section*{Questions / Enoncé}

\begin{compactenum}
\item A partir des matériaux que vous voudrez (idéalement assez simples : fil, bois, papier, carton, ballons, élastiques, cordes, poulies, petites masses), construire un dispositif qui permet de déplacer un objet de votre choix d'un point A à un point B
\item Réaliser (1) un diagramme interaction-objet (2) un schéma des forces en présence qui s'appliquent sur l'objet qui va être mis en mouvement dans les 2 cas de figure suivants 
  \begin{compactitem}
    \item Objet au repos
    \item Objet en mouvement
  \end{compactitem}
  \textbf{Il y a donc 4 schémas à réaliser.} 
  \textbf{Note importante : Exemple : si vous propulsez une balle avec un ressort, les diagrammes interaction-objet et schémas des forces en présence sont à faire sur la balle.}
\item Mettre votre système en marche, et \textbf{filmez} votre objet en mouvement entre le point A et le point B. Vous ferez un (ou plusieurs ?) calculs de la vitesse moyenne de l'objet entre le point de départ et le point d'arrivée en utilisant la formule ci-dessous 
\[
  v = \frac{d}{t}
  \]
  où 

  \begin{addmargin}[4em]{1em}
    \begin{compactitem}
        \item [v]: représente la vitesse de l'objet
        \item [d]: représente la distance parcourue
        \item [t]: représente le temps écoulé pour que l'objet parcourt la distance
    \end{compactitem}
    \end{addmargin}
    \item En gardant la même distance entre A et B, qu'auriez-vous pu changer dans votre dispositif pour que votre vitesse moyenne de déplacement de l'objet soit plus grande ? Pour chaque idée que vous soumettez, sur quelle force votre idée d'amélioration a-t-elle une influence ? Cette force a-t-elle augmenté ou diminué ?
  \end{compactenum}

\section*{Notions abordées}

\vspace{1em}
Notions abordées / Compétences à mettre en place :
\begin{itemize}[noitemsep]
  \item Construire un dispositif
  \item Appliquer ses connaissances de cours
  \item Réaliser une expérience, relever des mesures. 
  \item Formuler des hypothèses
  \item Interpréter des résultats / Réfléchir à des améliorations d'un dispositif
\end{itemize}

\section*{Le barème}

\begin{table}[H]
  \centering
  \renewcommand{\arraystretch}{1.3} % Ajuste l'espacement vertical dans le tableau
  \begin{tabular}{|m{5cm}|c|m{9cm}|}
      \hline
      \textbf{Catégorie / Compétence} & \textbf{Points} & \textbf{Description / Attendu} \\
      \hline
      Anticipation \par Ponctualité & 5 & Si votre devoir est rendu à l'heure et que vous avez itéré avec moi en avance. Les devoirs nécessitent des échanges pour clarifier des points, donc poser des questions est attendu. Vous perdrez des points si vous vous y prenez à la dernière minute sans me solliciter. \\
      \hline
      Construction d'un dispositif & 3 & Effort et soin que vous mettez dans la construction du dispositif. \\
      \hline
      Unités \par Conversions \par Formules & 2 & Justesse de vos calculs + toutes les variables doivent avoir une unité cohérente et être exprimées dans le système international (SI). Les conversions doivent être correctes (ex : ne pas mélanger km/h avec des minutes). \\
      \hline
      Sources & 3 & Les valeurs, méthodes utilisées doivent être justifiées avec une source (site web, expérience personnelle, etc.). \\
      \hline
      Qualité de la rédaction et du rendu & 3 & Le devoir doit être bien rédigé, clair, structuré, en format numérique. \\
      \hline
      Bon sens / "Jugeote" & 4 & Vous devez être capable d'interpréter vos résultats et de vérifier leur cohérence avec la réalité. Une conclusion qui semble absurde doit vous amener à questionner vos calculs et hypothèses. \\
      \hline
  \end{tabular}
  \caption{Barème d'évaluation}
  \label{tab:bareme}
\end{table}

\section*{Les attendus}

\begin{compactitem}
\item La vidéo doit être envoyée par email à l'adresse suivante : nicolas.bancel@ecole-suger.com 
\item Il est recommandé d'incorporer des photos du dispositif. Un format de rendu numérique Google Doc ou Word est donc plus adapté.
\item Citer vos sources. Même quand vous construisez votre dispositif, je veux voir tous les sites web qui vous ont permis de le construire ou de réfléchir à son design.
\end{compactitem}


\end{document}
