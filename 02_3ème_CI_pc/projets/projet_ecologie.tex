\documentclass[a4paper,12pt]{article}
\usepackage{../../mypackages}
\usepackage{../../macros}

\setlength{\parindent}{0pt}


\begin{document}

\title{Projet - Petit exercice sur l'écologie}
\author{N. Bancel}
\date{Decembre 2024}
\maketitle

\section*{Le bain ou la douche ?}

%colback=green!10!white, colframe=green!75!black

\begin{tcolorbox}[colback=gray!30, colframe=black]
  \textbf{Notes} : Cet exercice est volontairement peu guidé, il nécessite de formuler un problème de manière indépendante. Pour autant : ne pas hésiter à solliciter un/une camarade de classe, ou me solliciter directement. \par
  \vspace{1em}
  Notions abordées / Compétences à mettre en place :
  \begin{itemize}[noitemsep]
    \item Formules
    \item Poser un problème / Définir les variables qui permettent sa résolution
    \item Faire des estimations / Formuler des hypothèses
    \item Interpréter des résultats
  \end{itemize}
  \end{tcolorbox}

\subsection*{Enoncé}

\textcolor{blue}{\textbf{En utilisant des nombres, justifier ce qui est le plus écologique entre prendre une douche ou prendre un bain.}} \par
\vspace{1em}
Il n'est pas autorisé de simplement faire une recherche internet qui donne la réponse. Il faut passer par des formules qui permettent d'estimer des grandeurs. \par
\vspace{1em}
Pour déterminer le moyen le plus écologique de se laver, on ne comparera PAS les consommations d'électricité. La comparaison d'une autre grandeur (à trouver) est attendue.

\subsection*{Hypothèses - Informations}

\begin{itemize}[noitemsep]
  \item Pour la douche, il est suggéré d'utiliser la notion de débit. On peut faire une hypothèse en regardant les propriétés de la pomme de douche de son appartement, ou en cherchant des estimations de valeurs sur Internet.
  \item Pour le bain, il est utile d'essayer de dimensionner une baignoire. Vous pouvez la modéliser par un parallélépipède rectangle.
\end{itemize}
\vspace{1em}


\subsection*{Question supplémentaire}

\begin{itemize}[noitemsep]
  \item Vous avez à priori fait une hypothèse sur la valeur du débit dans la question précédente. Quelle expérience permettrait de déterminer le débit de votre douche chez vous ?
  \item Réalisez cette expérience et documentez là en prenant des photos ou une vidéo 
\end{itemize}

\end{document}
