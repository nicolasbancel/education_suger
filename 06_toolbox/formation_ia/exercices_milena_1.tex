\documentclass[a4paper,12pt]{article}
\usepackage[utf8]{inputenc}
\usepackage[T1]{fontenc}
\usepackage{amsmath,amssymb}
\usepackage{geometry}
\geometry{margin=2cm}

\title{Fiche d'exercices : Développement, factorisation, puissances et racines des polynômes}
\author{Pour Milena}
\date{}

\begin{document}

\maketitle

\section*{Partie 1 : Exercices corrigés}

\subsection*{1. Développement et factorisation}
\textbf{Exemple 1 : Développer $(x+3)(x-2)$}
\begin{align*}
  (x+3)(x-2) &= x \times x + x \times (-2) + 3 \times x + 3 \times (-2) \\
             &= x^2 - 2x + 3x - 6 \\
             &= x^2 + x - 6.
\end{align*}
\textbf{Méthode :} Utiliser la distributivité $a(b+c) = ab + ac$ pour développer.

\textbf{Exemple 2 : Factoriser $x^2 + 5x + 6$}
\begin{align*}
  x^2 + 5x + 6 &= (x+2)(x+3).
\end{align*}
\textbf{Méthode :} Trouver deux nombres dont le produit est 6 (le terme constant) et la somme est 5 (le coefficient de $x$).

\subsection*{2. Calculs de puissances}
\textbf{Exemple 1 : Calculer $2^3 \times 2^4$}
\begin{align*}
  2^3 \times 2^4 &= 2^{3+4} = 2^7 = 128.
\end{align*}
\textbf{Méthode :} Utiliser la règle $a^m \times a^n = a^{m+n}$.

\textbf{Exemple 2 : Calculer $(3^2)^3$}
\begin{align*}
  (3^2)^3 &= 3^{2 \times 3} = 3^6 = 729.
\end{align*}
\textbf{Méthode :} Utiliser la règle $(a^m)^n = a^{m \times n}$.

\subsection*{3. Racines d'un polynôme factorisé}
\textbf{Exemple 1 : Trouver les racines de $(x-2)(x+3) = 0$}
\begin{align*}
  (x-2)(x+3) = 0 &\implies x-2=0 \text{ ou } x+3=0 \\
  &\implies x=2 \text{ ou } x=-3.
\end{align*}
\textbf{Méthode :} Utiliser le fait qu'un produit est nul si et seulement si l'un des facteurs est nul.

\textbf{Exemple 2 : Trouver les racines de $x(x-4) = 0$}
\begin{align*}
  x(x-4) = 0 &\implies x=0 \text{ ou } x-4=0 \\
  &\implies x=0 \text{ ou } x=4.
\end{align*}

\section*{Partie 2 : Exercices non corrigés}

\subsection*{1. Développement et factorisation}
\begin{enumerate}
  \item Développer : $(x+2)(x+5)$.
  \item Factoriser : $x^2 + 7x + 10$.
  \item Développer et simplifier : $(2x-3)(x+4)$.
\end{enumerate}

\subsection*{2. Calculs de puissances}
\begin{enumerate}
  \item Calculer : $5^3 \times 5^2$.
  \item Calculer : $(2^4)^2$.
  \item Simplifier : $\frac{3^5}{3^2}$.
\end{enumerate}

\subsection*{3. Racines d'un polynôme factorisé}
\begin{enumerate}
  \item Trouver les racines de $(x+4)(x-3) = 0$.
  \item Trouver les racines de $x(x+5) = 0$.
  \item Trouver les racines de $(2x-1)(x+6) = 0$.
\end{enumerate}

\end{document}
