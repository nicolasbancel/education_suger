\documentclass[a4paper,12pt]{article}
\usepackage[french]{babel}
\usepackage{amsmath, amssymb}
\usepackage{xcolor}
\usepackage{tcolorbox}

\setlength{\parindent}{0pt}

\begin{document}

\title{Fiche d'exercice : Masse volumique}
\author{N. Bancel}
\date{Février 2025}

\maketitle

\section{Exercice 1}

\subsection{Problème}

\begin{tcolorbox}[colback=gray!10, colframe=black, title=\textbf{Exercice 1}]
Une entreprise de transport routier possède un poids lourd dont les dimensions utiles de la remorque sont :
\begin{itemize}
    \item Longueur : 13,70 m
    \item Largeur : 2,48 m
    \item Hauteur : 2,45 m
\end{itemize}
La charge maximale autorisée est de 26 tonnes.

On dispose de plusieurs types de bois avec des masses volumiques différentes (exprimées en kg/m$^3$). 

\textbf{Question :} Peut-on charger la remorque au maximum avec n'importe quel bois ?
\end{tcolorbox}

\subsection{Solution}

\textbf{Étape 1 : Calcul du volume de la remorque}

Le volume utile de la remorque est donné par :
\[
V = \text{longueur} \times \text{largeur} \times \text{hauteur}
\]
\[
V = 13,70 \times 2,48 \times 2,45
\]
\[
V = 83,1544 \text{ m}^3
\]

\textbf{Étape 2 : Calcul de la masse de bois maximale supportée}

La masse maximale autorisée étant de 26 tonnes (soit 26000 kg), la masse volumique limite que peut supporter la remorque est :
\[
\rho_{\max} = \frac{\text{masse maximale}}{\text{volume}}
\]
\[
\rho_{\max} = \frac{26000}{83,1544} \approx 312.6 \text{ kg/m}^3
\]

\textbf{Étape 3 : Comparaison avec les masses volumiques des bois}

On compare cette densité limite avec celles des bois disponibles :
\begin{itemize}
    \item Balsa : 140 kg/m$^3$ (ok)
    \item Chêne : 610-980 kg/m$^3$ (trop lourd)
    \item Chêne (cœur) : 1170 kg/m$^3$ (trop lourd)
    \item Contreplaqué : 440-880 kg/m$^3$ (trop lourd)
    \item Ébène : 1150 kg/m$^3$ (trop lourd)
    \item Hêtre : 800 kg/m$^3$ (trop lourd)
    \item Pin : 500 kg/m$^3$ (trop lourd)
    \item Sapin : 450 kg/m$^3$ (trop lourd)
    \item Teck : 860 kg/m$^3$ (trop lourd)
\end{itemize}

\textbf{Conclusion :} Seul le bois de \textbf{balsa} respecte la contrainte de masse. Pour tous les autres types de bois, la charge maximale de la remorque serait dépassée.

\end{document}
