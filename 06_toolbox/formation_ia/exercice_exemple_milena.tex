\documentclass[a4paper,11pt]{article}
\usepackage[french]{babel}
\usepackage{amsmath, amssymb}
\usepackage{geometry}
\geometry{top=2cm, bottom=2cm, left=2cm, right=2cm}

\begin{document}

\title{Fiche d'exercices : Développement, Factorisation, Puissances et Racines de Polynômes}
\author{Classe de 3ème}
\date{}
\maketitle

\section{Exercices corrigés}

\subsection{Développement et factorisation}

\textbf{Exemple 1 : Développement}

Développer et réduire : $(x+3)(x-5)$

\textbf{Correction :}
\begin{align*}
(x+3)(x-5) &= x \times x + x \times (-5) + 3 \times x + 3 \times (-5) \\
&= x^2 - 5x + 3x - 15 \\
&= x^2 - 2x - 15.
\end{align*}

\textbf{Exemple 2 : Factorisation}

Factoriser $x^2 - 4x + 3$.

\textbf{Correction :}
On cherche deux nombres dont le produit est $3$ et la somme est $-4$. On trouve $-1$ et $-3$.
\begin{align*}
x^2 - 4x + 3 &= (x - 1)(x - 3).
\end{align*}

\subsection{Calculs de puissances}

\textbf{Exemple 3 : Calcul de puissances}

Calculer $2^3 \times 2^4$.

\textbf{Correction :}
\begin{align*}
2^3 \times 2^4 &= 2^{3+4} = 2^7 = 128.
\end{align*}

\textbf{Exemple 4 : Puissances et fractions}

Calculer $\left(\frac{3}{2}\right)^2$.

\textbf{Correction :}
\begin{align*}
\left(\frac{3}{2}\right)^2 &= \frac{3^2}{2^2} = \frac{9}{4}.
\end{align*}

\subsection{Racines d'un polynôme factorisé}

\textbf{Exemple 5 : Trouver les racines}

Déterminer les racines de $f(x) = (x - 2)(x + 5)$.

\textbf{Correction :}
On cherche les valeurs de $x$ pour lesquelles $f(x) = 0$.
\begin{align*}
(x - 2)(x + 5) = 0 &\Rightarrow x - 2 = 0 \quad \text{ou} \quad x + 5 = 0 \\
&\Rightarrow x = 2 \quad \text{ou} \quad x = -5.
\end{align*}
Les racines sont donc $x = 2$ et $x = -5$.

\section{Exercices à résoudre}

\subsection{Développement et factorisation}
1. Développer et réduire : $(x+4)(x-6)$.
2. Factoriser : $x^2 - 7x + 10$.
3. Factoriser : $x^2 - 9$.

\subsection{Calculs de puissances}
1. Calculer $3^2 \times 3^3$.
2. Calculer $\left(\frac{5}{4}\right)^2$.
3. Simplifier $\frac{2^5}{2^3}$.

\subsection{Racines d'un polynôme factorisé}
1. Trouver les racines de $g(x) = (x - 3)(x + 7)$.
2. Trouver les racines de $h(x) = (x + 2)(x - 4)$.
3. Trouver les racines de $k(x) = (x - 1)(x - 5)$.

\end{document}
