\documentclass[a4paper,12pt]{article}

\setlength{\parindent}{0pt}

\begin{document}

\title{Fiche d'exercices - Faustine}
\author{N. Bancel}
\date{}
\maketitle

\section*{Exercice 1 : Compréhension des formules}
Calcule la masse d'un objet ayant une masse volumique de \( 8 \, \text{g/cm}^3 \) et un volume de \( 25 \, \text{cm}^3 \). \\
\textit{Rappel : la formule est \( m = \rho \times V \).}

\section*{Exercice 2 : Conversions d'unités}
Exprime les volumes suivants en litres :
\begin{enumerate}
    \item \( 250 \, \text{cm}^3 \)
    \item \( 1,5 \, \text{m}^3 \)
    \item \( 0,075 \, \text{dm}^3 \)
\end{enumerate}

\section*{Exercice 3 : Structure de l'atome}
Complète le tableau suivant avec les informations manquantes :

\begin{tabular}{|c|c|c|c|}
\hline
Atome & Protons & Neutrons & Électrons \\
\hline
Hydrogène (\( H \)) & 1 & 0 & \_\_ \\
\hline
Carbone (\( C \)) & 6 & 6 & \_\_ \\
\hline
Oxygène (\( O \)) & \_\_ & 8 & 8 \\
\hline
\end{tabular}

\section*{Exercice 4 : Équilibrage d'équations chimiques}
Équilibre les équations suivantes :
\begin{enumerate}
    \item \( H_2 + O_2 \rightarrow H_2O \)
    \item \( C + O_2 \rightarrow CO_2 \)
    \item \( Fe + O_2 \rightarrow Fe_2O_3 \)
\end{enumerate}

\section*{Exercice 5 : Analyse et justification}
Explique pourquoi la molécule \( H_2O \) ne peut pas être modifiée en \( H_2O_2 \) dans une équation chimique. \\
\textit{Indice : Pense à la structure des molécules et à leur signification.}

\end{document}
