\documentclass[a4paper,12pt]{article}
\usepackage{../../mypackages}
\usepackage{../../macros}

\setlength{\parindent}{0pt}

\begin{document}

\title{Fiche d'exercices - Faustine}
\author{N. Bancel}
\date{}
\maketitle

\section{Notions travaillées dans cette fiche d'exercices}
\begin{itemize}
    \item Utilisation et manipulation des formules (e.g., calcul de la masse volumique).
    \item Conversions d'unités.
    \item Structure de l'atome : protons, neutrons, électrons, numéro atomique et nombre de masse.
    \item Équilibrage des équations chimiques.
    \item Lecture et compréhension des énoncés scientifiques.
\end{itemize}

\section{Exercices}

\subsection*{Exercice 1 : Calcul de masse volumique}
Un cylindre de cuivre a une masse de \( 1780 \, \text{g} \) et un volume de \( 200 \, \text{cm}^3 \). Calcule sa masse volumique.

\subsection*{Exercice 2 : Conversions d'unités}
Convertis les volumes suivants en litres :
\begin{enumerate}
    \item \( 500 \, \text{cm}^3 \)
    \item \( 2,3 \, \text{m}^3 \)
    \item \( 0,25 \, \text{dm}^3 \)
\end{enumerate}

\subsection*{Exercice 3 : Identification des constituants de l'atome}
Complète le tableau :
\begin{tabular}{|c|c|c|c|}
\hline
Atome & Protons & Neutrons & Électrons \\
\hline
Carbone (\( C \)) & 6 & 6 & \_\_ \\
\hline
Oxygène (\( O \)) & \_\_ & 8 & 8 \\
\hline
Sodium (\( Na \)) & 11 & \_\_ & 11 \\
\hline
\end{tabular}

\subsection*{Exercice 4 : Définition de l'atome}
Explique pourquoi un atome est électriquement neutre.

\subsection*{Exercice 5 : Équilibrage d'équations chimiques}
Équilibre les équations suivantes :
\begin{enumerate}
    \item \( H_2 + O_2 \rightarrow H_2O \)
    \item \( Fe + O_2 \rightarrow Fe_2O_3 \)
    \item \( CH_4 + O_2 \rightarrow CO_2 + H_2O \)
\end{enumerate}

\subsection*{Exercice 6 : Compréhension des molécules}
Pourquoi ne peut-on pas modifier une molécule comme \( H_2O \) en \( H_2O_2 \) ?

\subsection*{Exercice 7 : Lecture d'un énoncé}
Un objet en bois flotte dans l'eau car sa masse volumique est inférieure à celle de l'eau. Explique ce phénomène en termes scientifiques.

\subsection*{Exercice 8 : Calcul scientifique}
Un objet a une masse volumique de \( 2,7 \, \text{g/cm}^3 \). Si son volume est \( 1,5 \, \text{dm}^3 \), quelle est sa masse ?

\subsection*{Exercice 9 : Différences entre atomes et molécules}
Décris les différences entre un atome et une molécule. Donne un exemple de chaque.

\subsection*{Exercice 10 : Vocabulaire scientifique}
Associe les termes suivants à leur définition : électron, proton, neutron, numéro atomique, nombre de masse.

\section{Corrigés}

\subsection*{Exercice 1 :}
La masse volumique est \( \rho = \frac{m}{V} = \frac{1780}{200} = 8,9 \, \text{g/cm}^3 \).

\subsection*{Exercice 2 :}
\begin{enumerate}
    \item \( 500 \, \text{cm}^3 = 0,5 \, \text{L} \)
    \item \( 2,3 \, \text{m}^3 = 2300 \, \text{L} \)
    \item \( 0,25 \, \text{dm}^3 = 0,25 \, \text{L} \)
\end{enumerate}

\subsection*{Exercice 3 :}
\begin{tabular}{|c|c|c|c|}
\hline
Atome & Protons & Neutrons & Électrons \\
\hline
Carbone (\( C \)) & 6 & 6 & 6 \\
\hline
Oxygène (\( O \)) & 8 & 8 & 8 \\
\hline
Sodium (\( Na \)) & 11 & 12 & 11 \\
\hline
\end{tabular}

\subsection*{Exercice 4 :}
Un atome est neutre car le nombre de protons (charges positives) est égal au nombre d'électrons (charges négatives).

\subsection*{Exercice 5 :}
\begin{enumerate}
    \item \( 2H_2 + O_2 \rightarrow 2H_2O \)
    \item \( 4Fe + 3O_2 \rightarrow 2Fe_2O_3 \)
    \item \( CH_4 + 2O_2 \rightarrow CO_2 + 2H_2O \)
\end{enumerate}

\subsection*{Exercice 6 :}
La molécule \( H_2O \) représente l’eau. La modifier en \( H_2O_2 \) changerait sa composition chimique et ses propriétés.

\subsection*{Exercice 7 :}
Un objet flotte car sa masse volumique est inférieure à celle de l’eau, ce qui réduit la force gravitationnelle agissant sur lui.

\subsection*{Exercice 8 :}
\( m = \rho \times V = 2,7 \times 1500 = 4050 \, \text{g} = 4,05 \, \text{kg} \).

\subsection*{Exercice 9 :}
Un atome est la plus petite unité de matière (e.g., \( H \)). Une molécule est un groupe d’atomes liés (e.g., \( H_2O \)).

\subsection*{Exercice 10 :}
\begin{itemize}
    \item Électron : Particule chargée négativement.
    \item Proton : Particule chargée positivement.
    \item Neutron : Particule neutre.
    \item Numéro atomique : Nombre de protons dans un atome.
    \item Nombre de masse : Somme des protons et neutrons.
\end{itemize}

\end{document}
