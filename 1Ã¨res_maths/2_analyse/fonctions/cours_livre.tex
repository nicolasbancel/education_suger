\documentclass[a4paper,12pt]{article}
\usepackage{../../../mypackages}
\usepackage{../../../macros}


\usepackage{pgfplots}
    \pgfplotsset{
    compat=1.11,
  }


\begin{document}

\title{Fonctions - Rappels de 2nde}
\author{N. Bancel}

\section*{Bases}
\subsection*{Définition}
Une \textbf{fonction} modélise la dépendance d'une quantité vis-à-vis d'une autre.

On note \( f(x) \) cette fonction où \( x \) est la variable dont dépend la quantité étudiée.

\subsection*{Exemple}
On s'intéresse à la trajectoire d'un ballon dans l'air lors d'un lancer franc au basketball :
\[
f(x) = -x^2 + 4x + 12
\]
On s'aperçoit que sa hauteur dépend de la distance par rapport au lanceur. On peut modéliser cela par une fonction \( f \) qui associe la hauteur \( z \) à la distance par rapport au lanceur. Ici, cette fonction sera :
\[
f(x) = -x^2 + 4x + 12
\]
Que l'on pourra noter également sous la forme :
\[
f : x \mapsto -x^2 + 4x + 12
\]

\section*{Image, antécédent et représentation graphique}
\subsection*{Vocabulaire}
En prenant une valeur réelle et en calculant \( f(a) \), on obtient l'\textbf{image} de \( a \) par la fonction \( f \). On dit que \( a \) est un \textbf{antécédent}.

\subsection*{Exemple}
Dans l'exemple précédent, comme \( f(2) = 16 \), on dit que 16 est l'image de 2. Et que 2 est l'antécédent de 16 par la fonction \( f \).

\subsection*{Propriétés}
Pour calculer une image, on substitue \( x \) par la valeur donnée dans \( f(x) \). Pour trouver un antécédent, on doit résoudre une équation.

\subsection*{Exemple}
Pour calculer l'image de 2 par la fonction \( f(x) = -x^2 + 4x + 12 \), on remplace \( x \) par 2 :
\[
f(2) = -2^2 + 4 \times 2 + 12 = -4 + 8 + 12 = 16
\]
L'image de 2 est donc 16. \par
Pour trouver l'antécédent de 8 par la fonction \( g(x) = 3x - 7 \), on résout :
\[
3x - 7 = 8
\]
\[
3x = 8 + 7
\]
\[
3x = 15
\]
\[
x = \frac{15}{3}
\]
\[
x = 5
\]

\section*{Représentation graphique}
Pour tracer la représentation graphique de la fonction \( f \), que l'on nomme \textbf{courbe représentative de} \( f \), ou tout simplement \( Cf \) :
\begin{enumerate}
    \item On prend plusieurs valeurs de \( x \) (dans le domaine de définition de \( f \)).
    \item Pour chacune de ces valeurs, on calcule l'image \( f(x) \).
    \item Les points de coordonnées \( (x ; f(x)) \) seront les points de la représentation graphique de \( f \). Il suffit de les placer et de les relier pour obtenir \( Cf \).
\end{enumerate}

\subsection*{Exemple}
Dans l'exemple précédent, tracer la représentation graphique de \( f(x) = -x^2 + 4x + 12 \).

\textbf{Réponse} : On construit un tableau de valeurs :
\[
\begin{array}{|c|c|c|c|c|c|}
\hline
x & 0 & 1 & 2 & 3 & 4 \\
\hline
f(x) & 12 & 15 & 16 & 15 & 12 \\
\hline
\end{array}
\]

\end{document}
