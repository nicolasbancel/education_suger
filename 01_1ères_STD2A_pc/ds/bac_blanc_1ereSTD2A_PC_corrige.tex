\documentclass[a4paper,12pt]{article}
\usepackage[utf8]{inputenc}
\usepackage[french]{babel}
\usepackage{amsmath,amssymb}
\usepackage{chemfig}
\usepackage{siunitx}
\usepackage{graphicx}
\usepackage{xcolor}
\usepackage{tcolorbox}

\setlength{\parindent}{0pt}
\setlength{\parskip}{1em}

\title{Correction - Interrogation Physique-Chimie}
\author{N. Bancel}
\date{\today}

\begin{document}

\maketitle

\section*{Exercice 1 [6 points] - Questions de cours}

\subsection*{Les matériaux - 2 points}
\begin{itemize}
    \item \textbf{Question 1 [1 point]} : Les 3 grandes familles de matériaux sont :
    \begin{itemize}
        \item \textbf{Métaux :} Exemple : fer, aluminium.
        \item \textbf{Polymères :} Exemple : polypropylène, PVC.
        \item \textbf{Céramiques :} Exemple : verre, porcelaine.
    \end{itemize}

    \item \textbf{Question 2 [0.5 point]} : Un alliage est un mélange de métaux (ou d'un métal et d'autres éléments), tandis qu'un métal pur contient un seul type d'atome métallique.
    \item \textbf{Question 3 [0.5 point]} : Un matériau composite est constitué de deux composants :
    \begin{itemize}
        \item Une matrice
        \item Un renfort
    \end{itemize}
    Exemple : béton armé (matrice : béton, renfort : acier).
\end{itemize}

\subsection*{L'atome - 1 point}
\begin{itemize}
    \item \textbf{Question 1 [1 point]} : Pour l'atome d'oxygène (\ce{O}) :
    \begin{itemize}
        \item Configuration électronique : $1s^2 2s^2 2p^4$.
        \item Couche de valence : 2 ; électrons de valence : 6.
        \item Schéma de Lewis :
        \[
        \ce{O: \dots \cdots }
        \]
    \end{itemize}
\end{itemize}

\subsection*{Les polymères - 1 point}
\begin{itemize}
    \item \textbf{Question 1 [1 point]} :
    \begin{itemize}
        \item \textbf{a [0.25 point]} : Un alcane est un hydrocarbure saturé (liaisons simples).
        \item \textbf{b [0.25 point]} : Un alcène est un hydrocarbure avec au moins une double liaison.
        \item \textbf{c [0.5 point]} : Un composé aromatique contient un cycle de carbones conjugués.
    \end{itemize}
\end{itemize}

\subsection*{Oxydoréduction - 2 points}
\begin{itemize}
    \item \textbf{Question 1 [0.5 point]} : Demi-équation pour \ce{Cu^{2+}/Cu} : 
    \[
    \ce{Cu^{2+} + 2e^- -> Cu}
    \]
    \item \textbf{Question 2 [0.75 point]} : Demi-équation pour \ce{I2/I^-} :
    \[
    \ce{I2 + 2e^- -> 2I^-}
    \]
    \item \textbf{Question 3 [0.75 point]} : Demi-équation pour \ce{H+/H2} :
    \[
    \ce{2H+ + 2e^- -> H2}
    \]
\end{itemize}

\section*{Exercice 2 [7.5 points] - Coulisses de Lascaux IV}

\subsection*{Le béton - 3.5 points}
\begin{itemize}
    \item \textbf{Question 1 [0.5 point]} : Caractéristiques :
    \begin{itemize}
        \item Esthétique : aspect brut, effet de roche naturelle.
        \item Technique : grande durabilité et fluidité.
    \end{itemize}
    \item \textbf{Question 2 [0.5 point]} : Le béton armé est un composite car il combine une matrice (béton) et un renfort (acier).
    \item \textbf{Question 3 [0.5 point]} : Le béton autoplaçant facilite le remplissage des coffrages complexes.
    \item \textbf{Question 4 [1 point]} :
    \[
    V = \frac{m}{\rho} = \frac{730\,000\,\mathrm{kg}}{7800\,\mathrm{kg/m^3}} \approx 93.59\,\mathrm{m^3}
    \]
    \item \textbf{Question 5 [1 point]} :
    \begin{itemize}
        \item Nombre de camions : $\lceil 93.59 / 25 \rceil = 4$.
        \item Remplissage du dernier camion : $93.59 - 3 \times 25 = 18.59\,\mathrm{m^3}$ soit environ $74.36\%$.
    \end{itemize}
\end{itemize}

\subsection*{La structure métallique - 3 points}
\begin{itemize}
    \item \textbf{Question 1 [0.5 point]} : Les pieux sont en acier.
    \item \textbf{Question 2 [2.5 points]} :
    \begin{itemize}
        \item \textbf{a [0.5 point]} : \ce{Fe^{2+} + 2e^- -> Fe}.
        \item \textbf{b [0.5 point]} : Le fer (\ce{Fe}) est oxydé.
        \item \textbf{c [1 point]} : \ce{O2 + 2H2O + 4e^- -> 4OH^-}.
        \item \textbf{d [0.5 point]} : Le dioxygène (\ce{O2}) est réduit.
    \end{itemize}
\end{itemize}

\subsection*{Le verre - 1 point}
\begin{itemize}
    \item \textbf{Question 1 [0.5 point]} : Le verre est une céramique.
    \item \textbf{Question 2 [0.5 point]} : Principal composant : silice (\ce{SiO2}).
\end{itemize}

\section*{Exercice 3 [7.5 points] - The Public Collection}

\subsection*{Polymères - 7.5 points}
\begin{itemize}
    \item \textbf{Question 1 [0.5 point]} : Le Bisphénol A appartient à la famille des alcools aromatiques (présence d’un groupe -OH sur un cycle aromatique).
    \item \textbf{Question 2 [1 point]} : Groupes caractéristiques : hydroxyle (-OH) et méthylène (-CH3).
    \item \textbf{Question 3 [0.5 point]} : La fonction ester est présente dans le polycarbonate.
    \item \textbf{Question 4 [1 point]} : Il s'agit d'une polycondensation.
    \item \textbf{Question 5 [1 point]} : Motif élémentaire : schéma.
    \item \textbf{Question 6 [1 point]} : L'indice de polymérisation $n$ est le nombre de motifs répétitifs dans un polymère.
    \item \textbf{Question 7 [2.5 points]} :
    \begin{itemize}
        \item \textbf{a [1 point]} : Formule développée : \ce{CH3-CH=CH2}.
        \item \textbf{b [1 point]} : Formule brute : \ce{C3H6}.
        \item \textbf{c [0.5 point]} : Le propylène est un alcène (présence d'une double liaison \ce{C=C}).
    \end{itemize}
\end{itemize}

\end{document}
