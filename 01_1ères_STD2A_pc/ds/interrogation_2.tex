\documentclass[addpoints]{exam}
\usepackage{../../mypackages}
\usepackage{../../macros}

\title{Interrogation N°2 - Peinture / Minéraux / Restauration}
\author{N. Bancel}
\date{Novembre 2024}

\begin{document}

\textbf{Collège Lycée Suger}
\hfill
\textbf{Physique-Chimie} \\

\textbf{Année 2024-2025 - 3ème trimestre}
\hfill
\textbf{1ères STD2A} \par

{\let\newpage\relax\maketitle}

\begin{center}
\textbf{\textcolor{red}{Durée : 45 minutes. La calculatrice n'est pas autorisée}} \\
\textbf{\textcolor{red}{Une réponse donnée sans justification sera considérée comme fausse.}} \\
Cette interrogation contient \numquestions\ questions, sur \numpages\ pages et est notée sur \numpoints\ points. 
\end{center}

\section{Partie 1 : La peinture}

\begin{questions}

  \question[4] Brièvement définir chacun des constituants de la peinture, et donner un exemple (reproduire sur votre copie et remplir le tableau)

  \begin{center}
    \begin{tabular}{|>{\bfseries}l|p{7cm}|p{4cm}|}
    \hline
    Constituant & Définition & Exemple \\
    \hline
    Liant &  &  \\
    \hline
    Diluant / Solvant &  &  \\
    \hline
    Pigment & &  \\
    \hline
    Charges &  &  \\
    \hline
    Additifs &  &  \\
    \hline
    \end{tabular}
    \end{center}

  \question[1] Que signifie une peinture "sans solvant" ?
  \question[1] Expliquer la différence entre le séchage par évaporation et le séchage par siccavation.  
  \question[2] Indiquer la différence entre le pigment et le colorant.

  \end{questions}

\section{Partie 2 : Les minéraux}

\begin{questions}

\question[2] Qu'est ce qu'un solide amorphe ?
\question[2] Quel est le principal composant du verre ? 
\question[1] Qu'est ce que la température de transition vitreuse ? On pourra s'aider d'un schéma pour justifier. 
\question[1] Ce qui distingue une céramique d'un verre est que la céramique a une structure majoritairement cristalline. Citer 2 propriétés physiques intéressantes de la céramique.

\end{questions}

\section{Partie 3 : La restauration}

\begin{questions}

\question[2] Quelle est la différence entre une méthode destructive et une méthode non destructive ? Donner un exemple de chaque.
\question[2] Quel type d'information sur une oeuvre les méthodes vues en cours permettent-elles de déterminer ?
\question[1] Expliquer brièvement la méthode de datation au carbone 14 
\question[1] Donner les valeurs limites des longueurs d'onde du domaine visible.
\question[1] Les rayonnements infrarouges sont invisibles pour l'oeil humain. Citer un autre type de rayonnement invisible utilisé dans l'étude d'oeuvres d'art.

\end{questions}

\end{document}
