
\documentclass[12pt]{article}
\usepackage[margin=2cm]{geometry}
\usepackage[french]{babel}
\usepackage{amsmath,amssymb}
\usepackage{chemfig}
\usepackage{mhchem}
\usepackage{enumitem}
\usepackage{../../mypackages}
\usepackage{../../macros}


\title{Fiche d'exercices \\ Physique-Chimie}
\author{}
\date{}


\begin{document}

\maketitle

\textbf{Collège Lycée Suger}
\hfill
\textbf{Physique-Chimie} \\

\textbf{Année 2024-2025}
\hfill
\textbf{1ères STD2A} \par

{\let\newpage\relax\maketitle}



\section{Oxydo-réduction}

\subsection*{Objectifs}
\begin{itemize}[compacitem]
    \item Identifier l’oxydant et le réducteur.
    \item Déterminer ce qui est oxydé et ce qui est réduit.
    \item Écrire les demi-équations électroniques.
    \item Écrire une équation bilan.
    \item Vérifier que l’équation bilan est équilibrée.
\end{itemize}



\subsection{Réaction de corrosion du fer}

\begin{compactitem}
    \item La corrosion du fer \ce{Fe} en présence d’eau et d’oxygène conduit à la formation d’ions \ce{Fe^{2+}}.
    \item Ecrire la demi-equation électronique du fer.
    \item Ecrire le couple oxydant /réducteur.
    \item Le Fer est-il un oxydant ou un réducteur ?
    \item Le \ce{Fe^{2+}} est-il un oxydé ou réduit ?
    \item La demi-équation du dioxygène en milieu acide est : \ce{O2 + 4H+ + 4e- -> 2H2O}.
    \item Ecrire le couple oxydant /réducteur. Identifier l’oxydant et le réducteur.
    \item En déduire l’équation bilan de la réaction.
    \item Vérifier que l’équation est bien équilibrée.
\end{compactitem}

\subsection{Réaction entre zinc et ions cuivre(II)}

Le zinc \ce{Zn^{2+}} est plongé dans une solution contenant des ions \ce{Cu^{2+}}.

\begin{compactitem}
    \item Identifier les couples oxydant/réducteur mis en jeu.
    \item Identifier l’oxydant et le réducteur.
    \item Écrire les demi-équations électroniques.
    \item Écrire l’équation bilan.
    \item Vérifier l’équilibrage.
\end{compactitem}

\subsection{Réaction entre l’aluminium et les ions fer(III)}

\begin{compactitem}
    \item Une lame d’aluminium est plongée dans une solution contenant des ions \ce{Fe^{3+}}.
    \item Identifier l’oxydant et le réducteur.
    \item Écrire les deux demi-équations.
    \item Écrire l’équation bilan.
    \item Vérifier l’équilibre des charges et des masses.
\end{compactitem}

\subsection{Réaction avec les ions permanganate \ce{MnO4^-}}

\begin{compactitem}
    \item Les ions \ce{MnO4^-} réagissent avec les ions \ce{Fe^{2+}} en milieu acide.
    \item Écrire les deux couples oxydant/réducteur.
    \item Identifier l’oxydant et le réducteur.
    \item Écrire les deux demi-équations électroniques.
    \item Écrire l’équation bilan.
    \item Vérifier l’équilibrage.
\end{compactitem}

\subsection{Réaction entre le cuivre et les ions nitrate}

\begin{compactitem}
    \item Une lame de cuivre est plongée dans de l’acide nitrique dilué.
    \item Il se forme des ions \ce{Cu^{2+}}, du gaz \ce{NO} et de l’eau.
    \item Écrire les demi-équations électroniques des deux couples.
    \item Identifier l’oxydant, le réducteur et les couples associés.
    \item Écrire l’équation bilan.
    \item Vérifier l’équilibrage.
\end{compactitem}

\subsection{Classement de réactivité}

\begin{compactitem}
    \item On considère les couples suivants :
    \begin{compactitem}
        \item \ce{Fe^{2+} / Fe}
        \item \ce{Cu^{2+} / Cu}
        \item \ce{Zn^{2+} / Zn}
    \end{compactitem}
    \item Classer les métaux selon leur pouvoir réducteur.
    \item Classer les ions selon leur pouvoir oxydant.
    \item En déduire si une lame de zinc plongée dans une solution de \ce{Cu^{2+}} réagit.
    \item Écrire l’équation de la réaction.
\end{compactitem}

\section{Ondes électromagnétiques}

\subsection{Fréquence et longueur d’onde}

\begin{compactitem}
    \item La lumière rouge a une longueur d’onde de $700$ nm dans le vide.
    \item Calculer sa fréquence.
    \item Utiliser $c = 3{,}00 \times 10^8$ m/s.
    \item Quelle est la fréquence d’un rayonnement ultraviolet de longueur d’onde $200$ nm ?
\end{compactitem}

\subsection{Énergie d’un photon}

\begin{compactitem}
    \item On considère un rayonnement de fréquence $6{,}00 \times 10^{14}$ Hz.
    \item Calculer l’énergie d’un photon de ce rayonnement.
    \item Utiliser $h = 6{,}63 \times 10^{-34}$ J$\cdot$s.
    \item Déterminer l’énergie d’un photon de lumière bleue ($\lambda = 450$ nm).
\end{compactitem}

\subsection{Identification du type de rayonnement}


\begin{compactitem}
    \item On mesure une énergie de photon de $3{,}10 \times 10^{-19}$ J.
    \item À quel domaine électromagnétique ce rayonnement appartient-il (UV, visible, IR) ?
    \item Même question pour $2{,}00 \times 10^{-20}$ J.
\end{compactitem}

\subsection{Comparaison de deux rayonnements}

\begin{compactitem}
    \item Le rayonnement A a une longueur d’onde de $600$ nm, le rayonnement B de $300$ nm.
    \item Lequel a la plus grande fréquence ? Justifier.
    \item Lequel transporte le plus d’énergie ? Justifier par un calcul.
\end{compactitem}

\end{document}
