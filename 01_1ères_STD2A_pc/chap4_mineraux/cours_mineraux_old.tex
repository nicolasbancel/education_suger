\documentclass[a4paper,12pt]{article}
\usepackage{../../mypackages}
\usepackage{../../macros}
\usepackage{multicol}
\usepackage{multido}
\usepackage{graphicx}
\usepackage{geometry}
\geometry{margin=2cm}

% Option pour version professeur ou élève
\newif\ifprof
\proffalse % changer \proffalse en \proftrue pour la version professeur

% Couleur des réponses professeurs
\newcommand{\proftext}[1]{\ifprof {\textcolor{blue}{#1}} \fi}

\title{Chapitre 4 - Les minéraux}
\author{N. Bancel}
\date{Mars 2025}

\begin{document}

\textbf{Collège Lycée Suger}
\hfill
\textbf{Physique-Chimie} \\

\textbf{Année 2024-2025}
\hfill
\textbf{1\textsuperscript{\text{ère}} STD2A} \par

{\let\newpage\relax\maketitle}

% Résumé encadré
\begin{tcolorbox}[colback=blue!5!white, colframe=blue!80!black, title=\textbf{\underline{\large À retenir}}]
\begin{itemize}
    \item Le verre est un solide \textbf{amorphe} principalement composé de \textbf{silice (SiO$_2$)}.
    \item Il passe d'un état \textbf{caoutchouteux} à un état \textbf{solide vitreux} lors du refroidissement.
    \item Les céramiques sont des matériaux \textbf{rigides, résistants à la chaleur et à la corrosion}.
\end{itemize}
\end{tcolorbox}

\section*{1. Le verre : définition et structure}
\subsection*{1.1 Composition et nature du verre}
\begin{itemize}
    \item Le verre est principalement composé de \hspace{4cm} \\
    \proftext{(réponse : dioxyde de silicium, SiO$_2$)}
    \item C'est un solide \hspace{1cm}, c'est-à-dire que \hspace{1cm}.
    \proftext{(réponse : amorphe ; sa structure n’est pas régulière comme un cristal)}
\end{itemize}



\begin{center}
\textit{\textbf{Espace pour dessiner la structure amorphe et cristalline de la silice :}}

\vspace{3cm}
\end{center}

\subsection*{1.2 Transition vitreuse}
\textit{Représente ci-dessous le schéma de la transition vitreuse entre les états solide et caoutchouteux :}

\begin{center}
\vspace{4cm} % Espace pour graphique
\end{center}

\section*{2. Fabrication du verre et rôle des additifs}
\subsection*{2.1 Fusion et fondants}
\begin{itemize}
    \item La silice fond à environ \dotfillrule{0.15cm}.
    \proftext{(réponse : 1730°C)}
    \item On utilise des fondants (\dotfillrule{0.15cm}) pour \dotfillrule{0.15cm}.
    \proftext{(réponse : soude, potasse ; abaisser la température de fusion)}
\end{itemize}

\subsection*{2.2 Coloration du verre}
Complète le tableau suivant avec la couleur obtenue :

\begin{center}
\begin{tabular}{|c|c|}
\hline
\textbf{Oxyde métallique} & \textbf{Couleur du verre} \\
\hline
Oxyde de fer & \ifprof \textcolor{blue}{Vert} \else \dotfillrule{0.15cm} \fi \\
\hline
Oxyde de cobalt & \ifprof \textcolor{blue}{Bleu} \else \dotfillrule{0.15cm} \fi \\
\hline
Oxyde de cuivre & \ifprof \textcolor{blue}{Rouge} \else \dotfillrule{0.15cm} \fi \\
\hline
Oxyde de manganèse & \ifprof \textcolor{blue}{Violet ou bleu} \else \dotfillrule{0.15cm} \fi \\
\hline
Sélénium & \ifprof \textcolor{blue}{Orange à rouge} \else \dotfillrule{0.15cm} \fi \\
\hline
\end{tabular}
\end{center}

\section*{3. Les céramiques}
\subsection*{3.1 Définition et types}
\begin{itemize}
    \item Les céramiques sont des matériaux \dotfillrule{0.15cm}, souvent obtenus par cuisson à haute température.
    \proftext{(réponse : inorganiques, non métalliques)}
    \item Deux grandes familles :
    \begin{itemize}
        \item Céramiques \dotfillrule{0.15cm} : poterie, brique, porcelaine.
        \proftext{(réponse : traditionnelles)}
        \item Céramiques \dotfillrule{0.15cm} : freins, optique, électronique.
        \proftext{(réponse : techniques ou industrielles)}
    \end{itemize}
\end{itemize}

\subsection*{3.2 Propriétés des céramiques}
\begin{multicols}{2}
\begin{itemize}
    \item Très \dotfillrule{0.15cm}
    \item Température de fusion \dotfillrule{0.15cm}
    \item Résistantes à la \dotfillrule{0.15cm}
    \item Bons isolants \dotfillrule{0.15cm}
\end{itemize}
\proftext{\begin{itemize}
    \item (réponses : rigides, >2000°C, corrosion, thermiques ou électriques)
\end{itemize}}
\end{multicols}

\subsection*{3.3 Fabrication et coloration}
\begin{itemize}
    \item Matière de départ : \dotfillrule{0.15cm} \proftext{(poudre)}
    \item Cuisson = frittage $\rightarrow$ les grains \dotfillrule{0.15cm}
    \proftext{(réponse : se soudent sans passer par une phase fondue)}
    \item Possibilité de coloration par \dotfillrule{0.15cm} \proftext{(oxydes métalliques)}
\end{itemize}

\vspace{1cm}
\textit{Schéma : fabrication et frittage d’une céramique}

\vspace{4cm}

\end{document}
