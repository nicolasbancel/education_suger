\documentclass[a4paper,12pt]{article}
\usepackage{../../mypackages}
\usepackage{../../macros}
\usepackage{xcolor}
\usepackage{amsmath}

\setlength{\parindent}{0pt}

\begin{document}

\title{Correction : Détermination de la couleur d'un photon}
\author{N. Bancel}
\date{Octobre 2024}
\maketitle


\begin{figure}[H]
  \centering
  \includegraphics[width=1\linewidth]{img/exercice_12.jpg}
  \captionsetup{labelformat=empty}
  \caption{\label{} Activité 2 page 85}
\end{figure}


\section*{Correction de l'exercice N°12}

\subsection*{1. Formule de base}
L'énergie d'un photon est reliée à sa longueur d'onde par la relation :

\begin{equation}
    E = \frac{h c}{\lambda}
\end{equation}

avec :
\begin{compactenum}
    \item $E = 2.12$ eV (énergie du photon),
    \item $h = 6.626 \times 10^{-34}$ J.s (constante de Planck),
    \item $c = 3.00 \times 10^8$ m/s (vitesse de la lumière),
    \item $1$ eV $= 1.602 \times 10^{-19}$ J (conversion eV → J).
\end{compactenum}

\subsection*{2. Conversion de l'énergie en joules}
On convertit $E$ en joules :

\begin{equation}
    E = 2.12 \times 1.602 \times 10^{-19}
\end{equation}

\begin{equation}
    E \approx 3.39 \times 10^{-19} \text{ J}
\end{equation}

\subsection*{3. Calcul de la longueur d'onde}
On isole $\lambda$ dans la formule :

\begin{equation}
    \lambda = \frac{h c}{E}
\end{equation}

En remplaçant les valeurs :

\begin{equation}
    \lambda = \frac{(6.626 \times 10^{-34}) \times (3.00 \times 10^8)}{3.39 \times 10^{-19}}
\end{equation}

\begin{equation}
    \lambda \approx 5.86 \times 10^{-7} \text{ m} = 586 \text{ nm}
\end{equation}

\subsection*{4. Détermination de la couleur}
D'après le spectre visible, une longueur d'onde de **586 nm** correspond à une lumière de couleur **jaune-orange**.

\subsection*{5. Conclusion}
Le photon émis possède une longueur d'onde de **586 nm**, ce qui correspond à une émission lumineuse de couleur **jaune-orange**, comme on peut l’observer dans le spectre visible.



\section*{Correction de l'exercice 14}

\subsection*{1. Représentation du phénomène de transition}
L’électron effectue une transition de l’état $E_1 = -3.03$ eV vers l’état fondamental $E_0 = -5.14$ eV. Cette transition entraîne l’émission d’un photon dont l’énergie est donnée par :

\begin{equation}
    E = E_1 - E_0
\end{equation}

Schématiquement, on représente cette transition par un diagramme d’énergie où un électron passe d’un niveau supérieur à un niveau inférieur en libérant un photon.

\subsection*{2. Calcul de la longueur d'onde du photon émis}
\begin{compactenum}
    \item \textbf{Calcul de l'énergie du photon} :
    \begin{equation}
        E = (-3.03) - (-5.14)
    \end{equation}
    \begin{equation}
        E = 2.11 \text{ eV}
    \end{equation}
    
    \item \textbf{Conversion en joules} :
    \begin{equation}
        E = 2.11 \times 1.602 \times 10^{-19}
    \end{equation}
    \begin{equation}
        E \approx 3.38 \times 10^{-19} \text{ J}
    \end{equation}
    
    \item \textbf{Calcul de la longueur d’onde} :
    \begin{equation}
        \lambda = \frac{h c}{E}
    \end{equation}
    \begin{equation}
        \lambda = \frac{(6.626 \times 10^{-34}) \times (3.00 \times 10^8)}{3.38 \times 10^{-19}}
    \end{equation}
    \begin{equation}
        \lambda \approx 5.88 \times 10^{-7} \text{ m} = 588 \text{ nm}
    \end{equation}
\end{compactenum}

\subsection*{3. Justification que le rayonnement est visible}
Le spectre visible s'étend de **400 nm (violet) à 700 nm (rouge)**. 
La longueur d’onde obtenue ($588$ nm) est bien dans cet intervalle, donc le rayonnement émis est visible.

\subsection*{4. Détermination de la couleur de la lumière émise}
D’après le spectre fourni, une longueur d’onde de **588 nm** correspond à une lumière **jaune-orange**.

\subsection*{5. Représentation du spectre de la lampe}
La lampe à vapeur de sodium émet principalement dans la bande du **jaune-orange** avec un pic autour de **589 nm**. Son spectre est discontinu et dominé par cette teinte caractéristique.

\subsection*{6. Conclusion}
La transition électronique étudiée conduit à l’émission d’un photon de **588 nm**, correspondant à une lumière **jaune-orange**, ce qui est typique des lampes à vapeur de sodium utilisées dans l’éclairage public.



\end{document}
