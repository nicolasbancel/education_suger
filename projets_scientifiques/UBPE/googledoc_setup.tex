\documentclass{article}
\usepackage{tcolorbox}
\usepackage{tabularx}
\usepackage{hyperref}

\title{Tutorial: Creating a Google Document for the Balloon Project}
\author{N. Bancel}
\date{October 2024}

\begin{document}

\maketitle

\section*{Introduction}

In this tutorial, you will learn how to structure a Google Document following specific guidelines. This format will be used for our balloon project. Follow the steps below carefully to ensure your document is well-organized and easy to read.

\vspace{1em}

Note : questions will be asked directly on the Google Document, through comments and tasks.

\section{Step : Creating a new Google Document}

\begin{tcolorbox}[colback=blue!10!white, colframe=blue!75!black, title=Instructions]
  \begin{enumerate}
  \item Sign in to your Google account and go to Google Docs.  
  \item Create a new blank document.
\end{enumerate}
\end{tcolorbox}

Once your document is ready, you will begin by adding a title and section headings.

\section{Step : Adding a Title}

To add a title, follow these instructions:

\begin{tcolorbox}[colback=green!10!white, colframe=green!75!black, title=Instructions]
  \begin{enumerate}
    \item At the top of the document, type your title: \textbf{FirstName LastName - Un ballon pour l'école}.  
    \item Highlight the text. 
    \item From the toolbar, click on the "Styles" drop-down menu and select \textbf{Title}.   
  \end{enumerate}
\end{tcolorbox}

Your title should now be large and bold, centered at the top of your document.

\section{Step : Adding Headings}

To organize your document, you will need to create section headings.

\begin{tcolorbox}[colback=green!10!white, colframe=green!75!black, title=Instructions]
  \begin{enumerate}
    \item For each section, write a heading, such as \textbf{First Step} or \textbf{Information about the Balloon}.  
    \item Highlight the heading text.  
    \item From the toolbar, click on the "Styles" drop-down menu and select \textbf{Heading 1} for main sections.  
    \item For sub-sections, use \textbf{Heading 2} to organize smaller parts within your sections.  
  \end{enumerate}

\end{tcolorbox}

Headings help structure your document and make it easier to navigate.

\section{Step : Creating a Bulleted List}

You will need to create a bulleted list for some of the sections, such as under \textbf{Information about the Balloon}.

\begin{tcolorbox}[colback=green!10!white, colframe=green!75!black, title=Instructions]
  \begin{enumerate}
    \item Type the items you want to include in your list, for example:  
      \begin{itemize}
        \item How high will the balloon go?  
        \item What sensors will we place on the balloon? What will they measure?  
        \item What challenges might we face?  
      \end{itemize}
  \item Highlight the items.  
  \item Click the \textbf{Bulleted List} icon in the toolbar. 
    \end{enumerate} 
\end{tcolorbox}

This will turn your selected text into a well-organized bulleted list.

\section{Step : Inserting a Table}

Next, you will create a table under the section \textbf{The Balloon Components} to organize the materials and their details.

\begin{tcolorbox}[colback=green!10!white, colframe=green!75!black, title=Instructions]
\begin{enumerate} 
  \item Place your cursor where you want the table.  
  \item From the toolbar, click on \textbf{Insert} \> \textbf{Table}, then choose a 5x4 table.  
  \item In the first row, type the column headers:  
  \begin{itemize}
    \item Name  
    \item Purpose  
    \item Material  
    \item Weight  
    \item Volume  
  \end{itemize}
  \item Fill in the rest of the table with information about each component. 
\end{enumerate}  
\end{tcolorbox}

\begin{tabularx}{\linewidth}{|X|X|X|X|X|}
  \hline
  \textbf{Name} & \textbf{Purpose} & \textbf{Material} & \textbf{Weight} & \textbf{Volume} \\
  \hline
  & & & & \\
  & & & & \\
  & & & & \\
  \hline
\end{tabularx}

\vspace{1em}
Tables help organize detailed information in a clear format.

\section{Step : Adding an Image or Drawing}

You can also add a drawing or image to your document, for example, a sketch of the balloon setup.

\begin{tcolorbox}[colback=green!10!white, colframe=green!75!black, title=Instructions]
\begin{itemize}
  \item Click on \textbf{Insert} \> \textbf{Drawing} \> \textbf{New}.  
  \item Use the tools provided to create a simple sketch or diagram.  
  \item When finished, click \textbf{Save and Close}. The drawing will be added to your document.  
  \end{itemize}
\end{tcolorbox}

This is useful for illustrating your project.

\section{Step : Adding a Project Timeline}

Create a final section for the project timeline to outline the key stages of the project and their deadlines.

\begin{tcolorbox}[colback=green!10!white, colframe=green!75!black, title=Instructions]
\begin{enumerate}
  \item Use \textbf{Heading 2} for the title: \textbf{Project Timeline}.  
  \item Create a list or table with the key project stages and the corresponding dates
\end{enumerate}
\end{tcolorbox}

\section{Conclusion}

Now you know how to structure your Google Document with a title, headings, lists, tables, and images. Be sure to save your document frequently and review it for any missing elements.

\begin{tcolorbox}[colback=yellow!10!white, colframe=yellow!75!black, title=Additional Resources]
  \textbf{Useful Links}:  
  \begin{itemize}
    \item \href{https://support.google.com/docs/}{Google Docs Help Center}  
    \item \href{https://www.youtube.com/watch?v=ZnrfW2i37I8}{Google Docs Tutorial Video}  
  \end{itemize}
\end{tcolorbox}

\end{document}
