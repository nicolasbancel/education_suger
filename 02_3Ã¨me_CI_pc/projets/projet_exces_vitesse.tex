\documentclass[a4paper,12pt]{article}
\usepackage{../../mypackages}
\usepackage{../../macros}

\setlength{\parindent}{0pt}

\title{Projet - Les excès de vitesse sur l'autoroute}
\author{N. Bancel}
\date{Avril 2025}

\begin{document}

\textbf{Collège Lycée Suger}
\hfill
\textbf{Physique-Chimie} \\

\textbf{Année 2024-2025 - 3ème trimestre}
\hfill
\textbf{3ème CI} \par

{\let\newpage\relax\maketitle}

\begin{tcolorbox}[colback=gray!30, colframe=black]
  \textbf{Notes} : Cet exercice est volontairement peu guidé, il nécessite de formuler un problème de manière indépendante.
  
  \begin{compactitem}
    \item Je note encore trop de copies où le niveau de réponse est bien trop élevé pour un niveau 3ème, ce qui éveille des doutes sur le fait que vous jouiez le jeu et fassiez le travail seul.
    \item Je voudrais voir des devoirs "qui vous ressemblent" - ce sera beaucoup valorisé dans la note finale.
    \item \textcolor{blue}{Toute absence de message pour me poser des questions sur EcoleDirecte pendant les vacances impactera significativement la note "Anticipation \& Ponctualité" (sur 5, voir barème)}.
  \end{compactitem}
  
  \vspace{1em}
  Le seul format accepté pour le rendu sera un dossier Google Drive, dans lequel vous mettrez tous les documents (Google Doc, ou PDF, ou photos) de votre travail, et que vous partagerez
  \begin{compactenum}
    \item avec l'adresse email "nicolas.bancel@ecole-suger.com" 
    \item avec des droits "Editeur"
  \end{compactenum}
  \end{tcolorbox}

\section*{\textcolor{blue}{Sujet}}

La vitesse maximale autorisée sur l'autoroute est de \num{130} \unit[per-mode = symbol]{\kilo\meter\per\hour}.
On supposera que votre papa ou votre maman roule régulièrement au-dessus de cette vitesse limite. \par

\vspace{1em}

En vous basant sur quelques trajets classiques que vous effectuez en famille, construisez un argumentaire qui s'appuie \textbf{en grande partie} sur des notions de Physique pour les en dissuader.
Vous pouvez ajouter d'autres arguments, mais une majorité doit être d'ordre Physique. \par 

\vspace{1em}

Vous utiliserez à minima les concepts :
\begin{compactenum}
  \item \textbf{de durée}
  \item \textbf{d'énergie cinétique}
\end{compactenum}

\vspace{1em}

Dans la mesure du possible : essayez de tracer quelques graphiques qui montrent les valeurs que prennent quelques variables en fonction de la valeur de l'excès de vitesse.

\section*{\textcolor{blue}{Notions abordées}}

\begin{itemize}[noitemsep]
  \item Utilisation de formules du cours.
  \item Poser un problème / Définir les variables qui permettent sa résolution, et tester plusieurs valeurs pour ces variables
  \item Tracer des graphiques 
  \item Faire des estimations, formuler des hypothèses
  \item Construire un argumentaire
\end{itemize}

\section*{\textcolor{blue}{Le barème}}

\begin{table}[H]
  \centering
  \renewcommand{\arraystretch}{1.3} % Ajuste l'espacement vertical dans le tableau
  \begin{tabular}{|m{5cm}|c|m{9cm}|}
      \hline
      \textbf{Catégorie / Compétence} & \textbf{Points} & \textbf{Description / Attendu} \\
      \hline
      Anticipation \par Ponctualité & 5 & Si votre devoir est rendu à l'heure et que vous avez itéré avec moi en avance. Les devoirs nécessitent des échanges pour clarifier des points, donc poser des questions est attendu. Vous perdrez des points si vous vous y prenez à la dernière minute, ou ne m'écrivez pas sur EcoleDirecte. \\
      \hline
      Unités \par Conversions \par Formules & 2 & Justesse de vos calculs + toutes les variables doivent avoir une unité cohérente et être exprimées dans le système international (SI). Les conversions doivent être correctes (ex : ne pas mélanger km/h avec des minutes). \\
      \hline
      Sources & 3 & Les valeurs, méthodes utilisées doivent être justifiées avec une source (site web, expérience personnelle, etc.). \\
      \hline
      Qualité de la rédaction & 3 & Le devoir doit être bien rédigé, clair, structuré. \\
      \hline
      Respect du format du rendu & 3 & Tous les documents doivent être uploadés sur Google Drive et partagés avec mon adresse email Suger \\
      \hline
      Bon sens / "Jugeote" & 4 & Vous devez être capable d'interpréter vos résultats et de vérifier leur cohérence avec la réalité. Une conclusion qui semble absurde doit vous amener à questionner vos calculs et hypothèses. \\
      \hline
  \end{tabular}
  \caption{Barème d'évaluation}
  \label{tab:bareme}
\end{table}


\end{document}
