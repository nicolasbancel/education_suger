\documentclass[answers]{exam}
\usepackage{../../mypackages}
\usepackage{../../macros}


\SolutionEmphasis{\color{blue}}
\renewcommand{\solutiontitle}{\noindent}


\title{DST N°4 - Forces, Poids, Attraction gravitationnelle et énergies}
\author{N. Bancel}
\date{31 Mars 2025}

\begin{document}

\textbf{Collège Lycée Suger}
\hfill
\textbf{Physique-Chimie} \\

\textbf{Année 2024-2025 - 3ème trimestre}
\hfill
\textbf{3ème CI} \par

{\let\newpage\relax\maketitle}

\begin{center}
\textbf{\textcolor{red}{Durée : 1h30. Coefficient 2. La calculatrice EST autorisée}} \\
\textbf{\textcolor{red}{Une réponse donnée sans justification sera considérée comme fausse.}} \\
\textbf{Il est autorisé (et recommandé) d'admettre une réponse pour pouvoir avancer sur les questions suivantes} \\

Cette interrogation contient \numquestions\ questions, sur \numpages\ pages et est notée sur 20 points. 

\end{center}

\section*{Exercice 1 : Cours (4.5 points)}

\begin{questions}
  \question[1.5] Quelles sont les 4 caractéristiques d'une force ?
  \begin{solution}
    Une force est une action mécanique modélisée par un vecteur. Elle possède 4 caractéristiques :
  
    \begin{compactitem}
      \item \textbf{Point d'application} : endroit précis du corps où s'exerce la force (par exemple : le centre de gravité).
      \item \textbf{Direction} : droite selon laquelle la force agit (verticale, horizontale, oblique).
      \item \textbf{Sens} : orientation de la force sur sa direction (vers le haut, vers le bas, vers la droite, etc.).
      \item \textbf{Valeur (ou intensité)} : mesurée en newton (\SI{}{\newton}), elle indique la "puissance" de la force.
    \end{compactitem}
  
    Ces 4 caractéristiques sont essentielles pour représenter une force correctement sur un schéma ou dans un calcul.
    \end{solution}
  \question[1.5] Donner la formule du \textbf{poids} d'un corps. Prendre soin de préciser 3 de ses caractéristiques, définir chaque facteur, et préciser l'unité du poids.
  \begin{solution}
    Le poids est la force d'attraction exercée par la Terre sur un objet. Il dépend de la masse de l'objet et de l'intensité de la pesanteur.
  
    \[
    P = m \times g
    \]
  
    où 
  
    \begin{addmargin}[4em]{1em}
      \begin{compactitem}
          \item [$P$] : représente le poids (en \SI{}{\newton})
          \item [$m$] : représente la masse de l'objet (en \SI{}{\kilogram})
          \item [$g$] : représente l'intensité de la pesanteur (en \SI{}{\newton\per\kilogram})
      \end{compactitem}
    \end{addmargin}
  
    \textbf{Trois caractéristiques du poids} :
    \begin{compactitem}
      \item Son point d’application est le centre de gravité de l’objet.
      \item Sa direction est toujours verticale.
      \item Son sens est toujours vers le bas, dirigé vers le centre de la terre.
    \end{compactitem}
    \end{solution}
  \question[1.5] Donner les formules de l'énergie cinétique et de l'énergie potentielle de pesanteur. Indiquez la signification de chaque variable, et son unité. 
  \begin{solution}

    \subsection*{Énergie cinétique}
  
    L'énergie cinétique correspond à l'énergie que possède un objet en mouvement.
  
    \[
    E_c = \frac{1}{2} \times m \times v^2
    \]
  
    où 
  
    \begin{addmargin}[4em]{1em}
      \begin{compactitem}
          \item [$E_c$] : énergie cinétique (en \SI{}{\joule})
          \item [$m$] : masse de l’objet (en \SI{}{\kilogram})
          \item [$v$] : vitesse de l’objet (en \unit[per-mode = symbol]{\meter\per\second}) %\SI{}{\meter\per\second}
      \end{compactitem}
    \end{addmargin}
  
    \subsection*{Énergie potentielle de pesanteur}
  
    L’énergie potentielle de pesanteur correspond à l’énergie qu’un objet possède en raison de son altitude par rapport au sol.
  
    \[
    E_p = m \times g \times h
    \]
  
    où 
  
    \begin{addmargin}[4em]{1em}
      \begin{compactitem}
          \item [$E_p$] : énergie potentielle de pesanteur (en \SI{}{\joule})
          \item [$m$] : masse de l’objet (en \SI{}{\kilogram})
          \item [$g$] : intensité de la pesanteur (en \unit[per-mode = symbol]{\newton\per\kilogram}) %(en \SI{}{\newton\per\kilogram})
          \item [$h$] : hauteur par rapport au sol (en \SI{}{\meter})
      \end{compactitem}
    \end{addmargin}
  
    \end{solution}
\end{questions}

\textbf{Note importante} \\
\vspace{1em}

Le format attendu pour les questions 2. et 3. (formules) est le suivant (attention, la formule suivante est illustrative, elle n'a aucun sens ): 

\begin{tcolorbox}[colback=gray!10!white, colframe=gray, title=Attendu sur les formules]
  Phrase explicative sur ce que représente la formule

\[
  A = \frac{b \times c}{d}
  \]
  où 

  \begin{addmargin}[4em]{1em}
    \begin{compactitem}
        \item [A]: Définition rapide de A (Unité de A)
        \item [b]: Définition rapide de b (Unité de b)
        \item [c]: Définition rapide de c (Unité de c)
        \item [d]: Définition rapide de d (Unité de d)
    \end{compactitem}
    \end{addmargin}
  \end{tcolorbox}

\section*{Exercice 2 : Forces et poids (8 points)}

\begin{figure}[H]
  \centering
  \includegraphics[width=0.7\linewidth]{img/dst_04_01.jpeg}
  \captionsetup{labelformat=empty}
\end{figure}

\begin{figure}[H]
  \centering
  \includegraphics[width=0.7\linewidth]{img/dst_04_02.jpeg}
  \captionsetup{labelformat=empty}
\end{figure}

\textbf{Données :}
\begin{compactitem}
  \item Masse du Soleil : \( m_s = \SI{1.99e30}{\kilogram} \)
  \item Force d'attraction gravitationnelle entre deux corps de masses \( m_1 \) et \( m_2 \) distants de \( d \) :
  \[
  F = \SI{6.67e-11}{} \times \frac{m_1 \times m_2}{d^2}
  \]
  \item Intensité de la pesanteur terrestre : \( g = \SI{9.8}{\newton\per\kilogram} \)
\end{compactitem}

\begin{questions}
  \question[1.5] Faire la liste des interactions dans lesquelles est engagée la fusée \textit{Titan} lorsqu'elle est immobile, avant son lancement. Préciser s'il s'agit d'interaction de contact ou à distance.
  \begin{solution}
    La fusée est soumise à deux interactions principales :
  
    \begin{compactitem}
      \item \textbf{Poids (force gravitationnelle)} : interaction à distance exercée par la Terre sur la fusée.
      \item \textbf{Réaction du sol} : interaction de contact exercée par le sol, qui empêche la fusée de s’enfoncer.
    \end{compactitem}
  
    Ces deux forces se compensent lorsque la fusée est immobile.
    \end{solution}
  \question[1] Calculer le poids de la fusée \textit{Titan} lors de son décollage.
  \begin{solution}

    \subsection*{Formule}
    Le poids se calcule avec la relation :
  
    \[
    P = m \times g
    \]
  
    \subsection*{Conversion des unités}
    La masse de la fusée est donnée en tonnes : \( \SI{633}{tonnes} = \SI{633000}{\kilogram} \)
  
    \subsection*{Application numérique}
  
    \begin{align*}
    P &= 633000 \times 9.8 \\
    P &= 6203400
    \end{align*}
  
    \textbf{Conclusion} : Le poids de la fusée est de \(\SI{6.20e6}{\newton}\)
    \end{solution}
  \question[1.5] Le moteur de la fusée génère une force nommée "Pousée". Schématiser les forces exercées sur la fusée \textit{Titan} lorsqu'elle vient tout juste de quitter le sol, au moment de son décollage. Échelle : \SI{1}{\centi\meter} correspond à \SI{2e6}{\newton}.
  \begin{solution}
    \textbf{Forces en jeu}
    \vspace{1em}
    Deux forces s’exercent sur la fusée :
  
    \begin{compactitem}
      \item La \textbf{poussée} vers le haut : \(\SI{1.2e7}{\newton}\)
      \item Le \textbf{poids} vers le bas : \(\SI{6.2e6}{\newton}\)
    \end{compactitem}
    \vspace{1em}
    \textbf{Échelle graphique}
    \vspace{1em}
    \begin{compactitem}
      \item Puisque 1 cm $\equiv$ \SI{2e6}{\newton}, on peut calculer les longueurs des vecteurs des forces :
      \item Longueur du vecteur poussée :
      \[ 
      \frac{\num{12e6}}{\num{2e6}} = 6\,\text{cm} 
      \]
      \item Longueur du vecteur poids :
      \[
      \frac{\num{6.2e6}}{\num{2e6}} \approx 3.1\,\text{cm}
      \]
    \end{compactitem}
    \vspace{1em}

    \textbf{Points d'application et notations}
    \begin{compactitem}
      \item Il est attendu de justifier où placer les points d'application. 
      \item Pour le poids, c'est une force à distance exercée par la terre sur l'objet, elle est donc répartie équitablement sur toute la fusée : on choisit le centre de gravité de la fusée. Il est utile de l'écrire sur la copie
      \item Pour la force de poussée, il est cohérent de placer un "moteur" sur le dessin et de considérer que cette force a un point d'application (de départ) qui est le centre du moteur.
    \end{compactitem}
    
    \vspace{1em}
    
    \textbf{Attendus pour le schéma}
    \begin{compactitem}
      \item Les flèches de chaque force doivent être de couleur différente pour ajouter de la clarté 
      \item On doit nommer et définir les forces. A côté de chaque flèche, on doit voir : 
      \begin{compactitem}
        \item $\vec{F}$ à côté de la flèche bleue  
        \item $\vec{P}$ à côté de la flèche rouge 
        \item On définit ensuite $\vec{F}$  comme suit : $\vec{F}$ : Force de poussée. Et $\vec{P}$ : Poids
      \end{compactitem}
      \item Il est aussi valorisé de re dessiner l'échelle 
    \end{compactitem}
    
    \vspace{1em}

    \begin{figure}[H]
      \centering
      \includegraphics[width=0.7\linewidth]{img/dst_04_03.jpg}
      \captionsetup{labelformat=empty}
    \end{figure}
    
    
    \textbf{Conclusion} : Le vecteur poussée doit être tracé vers le haut, de 6 cm, et le vecteur poids vers le bas, de 3,1 cm. La fusée décolle car la poussée est plus grande que le poids.



    \end{solution}
  \question[0.5] Pourquoi la fusée \textit{Titan} peut-elle décoller ? 
  \begin{solution}
    La fusée peut décoller car la \textbf{force de poussée} exercée par les moteurs est \textbf{supérieure à son poids}.
    
    Il en résulte une force résultante dirigée vers le haut, qui provoque l’accélération de la fusée.

    \begin{figure}[H]
      \centering
      \includegraphics[width=0.7\linewidth]{img/dst_04_04.jpg}
      \captionsetup{labelformat=empty}
    \end{figure}

    \end{solution}

  \question[3.5] Forces gravitationnelles
    \begin{parts}
    \part[1.5] Schématiser les forces d'attraction gravitationnelle exercées entre le Soleil et Voyager I.
    \begin{solution}
      Deux forces gravitationnelles sont à représenter :
      \begin{compactitem}
        \item La force exercée par le Soleil sur Voyager I, dirigée vers le Soleil.
        \item La force exercée par Voyager I sur le Soleil, dirigée vers Voyager I.
      \end{compactitem}
  
      Ces forces sont de même valeur, mais de sens opposé. Ce sont des \textbf{forces d'interaction gravitationnelle}.

      \textcolor{red}{\textbf{Méthode et points d'attention}}

      \begin{compactitem}
        \item Les points d'application de ces forces sont les \textbf{centres} des corps
        \item La direction de ces forces est la droite qui relie les centres des corps.
        \item Leurs sens sont opposés 
        \item La valeur n'a pas encore été calculée ici donc rien n'est imposé en terme de longueur des flèches. La seule contrainte importante à garder en tête est que la longueur doit être la même parce que les forces d'attraction gravitationnelles sont de même intensité.
        
        \begin{figure}[H]
          \centering
          \includegraphics[width=0.7\linewidth]{img/dst_04_05.jpg}
          \captionsetup{labelformat=empty}
        \end{figure}
    

      \end{compactitem}


      \end{solution}

    \part[0.5] Comment la valeur de ces forces a-t-elle évoluée entre 2007 et 2017 ? Justifier (pas besoin de poser de calcul ici)
    \begin{solution}
      Rappel de la formule 

      \[
      \overrightarrow{F}_{sonde \rightarrow soleil} = \overrightarrow{F}_{Soleil \rightarrow Sonde} = G \times \frac{m_1 \times m_2}{d^2}
      \]

      La force gravitationnelle diminue quand la distance entre les deux objets augmente, car le dénominateur \( d^2 \) grandit. \\
      
      Entre 2007 (15 millions de km) et 2017 (21 millions de km), la distance entre les deux corps a augmenté : donc la force a \textbf{diminué}. \\

      \textbf{Réfléchir au caractère intuitif des choses} \\
      
      L'analyse mathématique via la formule est cohérente avec l'intuition : plus on éloigne deux objets l'un de l'autre, moins ils exercent de l'influence l'un sur l'autre

      \end{solution}
  
    \part[1.5] Calculer la valeur de la force d'attraction gravitationnelle exercée par le Soleil sur \textit{Voyager 1} en 2017.
    \begin{solution}
      \subsection*{Formule utilisée}
      \[
      F = G \times \frac{m_1 \times m_2}{d^2}
      \]
  
      \subsection*{Valeurs utilisées et conversions au système international}
  
      \begin{addmargin}[4em]{1em}
        \begin{compactitem}
          \item $F$ : force gravitationnelle (en \SI{}{\newton})
          \item $G$ : constante gravitationnelle = \SI{6.67e-11}{\newton\meter\squared\per\kilogram\squared}
          \item $m_1$ : masse du Soleil = \SI{1.99e30}{\kilogram}
          \item $m_2$ : masse de Voyager I = \SI{825}{\kilogram}
          \item $d$ : distance en mètres entre Voyager I et le Soleil en 2017 = $\SI{21e6}{\kilo\meter}$. \textcolor{red}{Le kilomètre n'est PAS une unité du système international. Il faut donc convertir cette donnée en mètres. 
          Rappel $\qty{1}{\km} = \qty{1e3}{\m}$}
          %d = \qty{21e6}{\kilo\meter} = \qtyproduct{21 x 1e6 x 1e3} \unit{meter} = \SI{2.1e10}{\meter}
          \begin{align*}
            d &= \qty{21e6}{\kilo\meter} \\
              &= 21 \times 10^6 \times 10^3 \unit{\meter} \\
              &= \SI{2.1e10}{\meter}
          \end{align*}
        \end{compactitem}
      \end{addmargin}
  
      \subsection*{Application numérique}
      \begin{align*}
      F &= \num{6.67e-11} \times \frac{\num{1.99e30} \times 825}{(\num{2.1e10})^2} \\
      F &= \qty{248.3}{\N}
      \end{align*}
  
      \textbf{Conclusion} : La force gravitationnelle exercée par le Soleil sur Voyager I en 2017 est d’environ \(\SI{250}{\newton}\)
      \end{solution}
    \end{parts}
\end{questions}


\section*{Exercice 3 : Énergie (3 points)}

\begin{questions}

\question[1.5] Calculer l'énergie cinétique du bobsleigh et des 2 athlètes lorsque le 2eme pilote vient de monter.

\begin{solution}

\subsection*{Formule de l'énergie cinétique}
\[
E_c = \frac{1}{2} \times m \times v^2
\]

\subsection*{Conversion et valeurs utilisées}
\begin{compactitem}
  \item Masse du bobsleigh : \SI{210}{\kilogram}
  \item Masse de chaque athlète : \SI{90}{\kilogram}
  \item Masse totale : \( m = 210 + 90 + 90 = \SI{390}{\kilogram} \)
  \item Vitesse : \( v = \SI{10}{\meter\per\second} \)
\end{compactitem}

\subsection*{Application numérique}
\begin{align*}
E_c &= \frac{1}{2} \times 390 \times 10^2 \\
E_c &= 0.5 \times 390 \times 100 \\
E_c &= 195 \times 100 \\
E_c &= 19500
\end{align*}

\textbf{Conclusion} : L'énergie cinétique du bobsleigh avec les deux athlètes est \(\SI{19500}{\joule}\)

\end{solution}

\question[1] Comparer l'énergie cinétique du bobsleigh dans la situation où 1 seul athlète est dedans, par rapport à l'énergie cinétique du bobsleigh quand les 2 athlètes sont dedans en calculant leur rapport.

\begin{solution}

\subsection*{Formule du rapport}
On compare deux énergies cinétiques :
\[
R = \frac{E_{c,\text{2 athlètes}}}{E_{c,\text{1 athlète}}}
\]

\subsection*{Valeurs numériques}
\begin{align*}
R &= \frac{19500}{15000} \\
R &= 1.3
\end{align*}

\textbf{Conclusion} : L'énergie cinétique du bobsleigh avec deux athlètes est 1,3 fois plus grande que celle avec un seul athlète.
\end{solution}

\question[0.5] Quel est l'intérêt de monter à deux dans un bobsleigh ?

\begin{solution}
Lorsque les deux athlètes montent dans le bobsleigh, la masse totale augmente. À vitesse égale, cela augmente l’énergie cinétique du bobsleigh. Une plus grande énergie cinétique permet de mieux résister aux frottements et donc de conserver plus de vitesse pendant la descente.
\end{solution}

\end{questions}

\section*{Exercice 4 : La pierre (4.5 points)}

\begin{questions}

\question[1] En utilisant la formule du poids, calculer la masse $m$ de la pierre. (on prendra \(g = \SI{10}{\newton\per\kilogram}\))

\begin{solution}

\subsection*{Formule utilisée}
\[
P = m \times g
\]

\subsection*{Application numérique}

\begin{align*}
m &= \frac{P}{g} \\
m &= \frac{1.5}{10} \\
m &= 0.15
\end{align*}

\textbf{Conclusion} : La masse de la pierre est de \(\SI{0.15}{\kilogram}\)
\end{solution}

\question[0.5] Calculer l'énergie mécanique au départ en utilisant la formule :
\[
E_m = E_p + E_c
\]

\begin{solution}

\subsection*{Données}
\begin{compactitem}
  \item $E_p = \SI{9}{\joule}$
  \item $E_c = \SI{0}{\joule}$. Justification : la vitesse est nulle au départ donc :
  \begin{align*}
  E_c &= \frac{1}{2} \times m \times v^2 \\
  E_c &= \frac{1}{2} \times 0.15 \times 0^2 \\
  E_c &= 0 \\
  \end{align*}
\end{compactitem}

\subsection*{Application numérique}
\begin{align*}
E_{m\_depart} &= 9 + 0 \\
E_{m\_depart} &= 9
\end{align*}

\textbf{Conclusion} : L’énergie mécanique de la pierre au départ est de \(\SI{9}{\joule}\)
\end{solution}

\question[1] En supposant que l'énergie mécanique reste constante pendant tout le mouvement de la pierre, et en utilisant la formule de l'énergie potentielle, déterminer la valeur de l'énergie cinétique de la pierre au moment où elle touche le sol.

\begin{solution}


  \subsection*{Principe}
  L’énergie mécanique reste constante. Au moment où la pierre touche l’eau, sa hauteur est nulle, donc son énergie potentielle est nulle. En effet : 
  \begin{align*}
    E_{p\_final} &= m \times g \times h_{final} \\
    E_{p\_final} &= 0.15 \times 9.8 \times 0 \\
    E_{p\_final} &= 0
  \end{align*}
  
  \subsection*{Formules}
  \begin{align*}
  E_{m\_depart} &= E_{m\_final} \\
  E_{m\_depart} &= E_{c\_final} + E_{p\_final} \\
  \intertext{Or on vient de démontrer que $E_{p\_final} = 0$ donc}
  E_{c\_final} &= E_{m\_depart} \\
  \intertext{Ainsi, }
  E_{c\_final} &= \qty{9}{\J}
  \end{align*}
\textbf{Conclusion} : L’énergie cinétique de la pierre au moment de toucher l’eau est de \(\SI{9}{\joule}\)
\end{solution}

\question[1] Démontrer que la vitesse d'un objet peut s'écrire sous la forme 
\[
v = \sqrt{\frac{2 \times E_c}{m}}
\]

\begin{solution}

\subsection*{Formule de départ}
On part de la formule de l’énergie cinétique :

\[
E_c = \frac{1}{2} \times m \times v^2
\]

\subsection*{Étapes de transformation}
\begin{align*}
E_c &= \frac{1}{2} \times m \times v^2 \\
2 \times E_c &= m \times v^2 \\
\frac{2 \times E_c}{m} &= v^2 \\
v &= \sqrt{\frac{2 \times E_c}{m}}
\end{align*}

\textbf{Conclusion} : La vitesse peut être exprimée sous la forme \( v = \sqrt{\frac{2 \times E_c}{m}} \)
\end{solution}

\question[1] En déduire la vitesse $v$ de la pierre à son arrivée dans l'eau.

\begin{solution}

\subsection*{Formule utilisée}
\[
v_{finale} = \sqrt{\frac{2 \times E_{c\_final}}{m}}
\]

\subsection*{Valeurs numériques}
\begin{compactitem}
  \item \( E_{c\_final} = \qty{9}{\J} \)
  \item \( m = \qty{0.15}{\kg} \)
\end{compactitem}

\subsection*{Application numérique}
\begin{align*}
v &= \sqrt{\frac{2 \times 9}{0.15}} \\
v &= \sqrt{\frac{18}{0.15}} \\
v &= \sqrt{120} \\
v &\approx \qty{10.95}{\m\per\second}
\end{align*}

\textbf{Conclusion} : La vitesse de la pierre à son arrivée dans l’eau est d’environ \(\SI{11.0}{\meter\per\second}\). Soit à peu près $\qty{39.4}{\km\per\hour}$.
\end{solution}

\end{questions}

\end{document}
