\documentclass[answers]{exam}
\usepackage{../../mypackages}
\usepackage{../../macros}


\SolutionEmphasis{\color{blue}}
\renewcommand{\solutiontitle}{\noindent}


\title{Brevet blanc N°2 - Physique-Chimie}

\begin{document}

\textbf{Collège Lycée Suger}
\hfill
\textbf{Physique-Chimie} \\

\textbf{Année 2024-2025 - 3ème trimestre}
\hfill
\textbf{Brevet Blanc} \par

{\let\newpage\relax\maketitle}

\begin{center}
\textbf{\textcolor{red}{Durée : 30 minutes. \\
L'usage de la calculatrice avec mode examen actif est autorisé. \\ 
L'usage de la calculatrice sans mémoire, « type collège » est autorisé.}} \\
\textbf{Toute réponse, même incomplète, montrant la démarche de recherche du candidat sera prise en compte dans la notation} \\
\textbf{L'épreuve de Physique comporte 8 questions et est notée sur 25.} \\
\end{center}

\section*{Mission Alpha}

\textit{Extrait du Brevet 2022}
\vspace{1em}

Le 23 avril 2021, l'astronaute français Thomas Pesquet a décollé depuis la base de Cap Canaveral en Floride (USA) pour rejoindre la station spatiale internationale en orbite autour de la Terre, avec trois autres membres d'équipage : c'est la mission Alpha.
L'équipage a rejoint la station spatiale internationale à bord du vaisseau spatial Crew Dragon, lancé par une fusée Falcon 9.

\begin{figure}[H]
  \centering
  \includegraphics[width=0.4\linewidth]{img/brevet_01.jpg}
\end{figure}


\vspace{0.3cm}
\textbf{Les parties 1 et 2 sont indépendantes.}

\subsection*{Partie 1 – L’eau et l’air dans la station spatiale.}

\textit{L’eau et l’air sont nécessaires à la vie des astronautes : leurs besoins sont assurés par différents procédés.}

\begin{questions}

  \question[4] Parmi les formules chimiques ci-dessous, recopier sur la copie les noms de celles qui correspondent à des molécules. Justifier la réponse.
    
  \begin{center}
  \begin{tabular}{lll}
  Diazote : \ce{N2} & Dioxygène : \ce{O2} \\
  Hydrogène : \ce{H} & Oxygène : \ce{O} \\
  Eau : \ce{H2O} & Azote : \ce{N} \\
  \end{tabular}
  \end{center}

\subsection*{Partie 2 – "Regardez le monde défiler"}
  
\textit{Thomas Pesquet a proposé de nombreuses photos et vidéos au cours des six mois passés dans la station spatiale internationale.}

  \question[4] La station spatiale est en mouvement circulaire et uniforme par rapport au centre de la Terre. Thomas Pesquet reste au hublot de la station spatiale pour prendre des photos.
  
  Déterminer si les deux affirmations suivantes sont vraies ou fausses et justifier.
  
  \begin{compactitem}
  \item Affirmation A : Thomas Pesquet est immobile par rapport à la station spatiale.
  \item Affirmation B : Thomas Pesquet est en mouvement par rapport au centre de la Terre.
  \end{compactitem}
  
  \question[4] \textbf{Données :}
  
  \begin{compactitem}
  \item Vitesse moyenne de la station spatiale internationale sur son orbite autour de la Terre : $v = \num{27600} \unit[per-mode = symbol]{\kilo\meter\per\hour}$.
  \item Distance moyenne parcourue par la station spatiale internationale sur son orbite autour de la Terre, pour un tour : $d = \SI{42600}{\kilo\meter}$.
  \item La durée $t$ (en h) nécessaire pour parcourir une distance $d$ (en km) à une vitesse moyenne $v$ (en \unit[per-mode = symbol]{\kilo\meter\per\hour}) s'écrit :
  \[
  t = \frac{d}{v}
  \]
  \end{compactitem}
  
  En 24 heures, la station spatiale internationale réalise plusieurs fois le tour de la Terre : ses occupants peuvent ainsi assister à de nombreux levers et couchers du Soleil.
  
  Montrer, par un calcul, que la durée $t$ nécessaire à la station spatiale internationale pour faire le tour de la Terre vaut environ \SI{1.5}{\hour}, soit \SI{1}{\hour} \SI{30}{\minute}.
  
  \question[2] \textbf{Question Bonus} En se basant sur le résultat précédent, combien de fois la station spatiale internationale fait-elle le tour de la terre en une journée complète ?

\section*{La Salanité de l'eau}

\textit{Extrait du Brevet 2021}
\vspace{1em}

La salinité d’une eau désigne la masse de sel dissous dans un litre de cette eau.
Le tableau suivant donne les caractéristiques de quatre eaux différentes.

\begin{figure}[H]
  \centering
  \includegraphics[width=0.7\linewidth]{img/brevet_02.jpg}
\end{figure}

  \question[3] Parmi les relations suivantes, indiquer celle qui permet de calculer la masse volumique $\rho$. Préciser ce que représentent $m$ et $V$.

  \begin{center}
    \renewcommand{\arraystretch}{2} % augmente l'espacement vertical
    \setlength{\tabcolsep}{20pt}   % augmente l'espacement horizontal
    \begin{tabular}{|c|c|c|}
    \hline
    Relation A & Relation B & Relation C \\
    \hline
    $\rho = \dfrac{m}{V}$ & $\rho = m \times V$ & $\rho = \dfrac{V}{m}$ \\
    \hline
    \end{tabular}
    \end{center}
  
  Pour trouver la masse volumique de l’eau à la surface de l’océan Atlantique Nord, on prélève un échantillon de \SI{50.0}{\milli\liter} de cette eau et on mesure sa masse soit \SI{51.2}{\gram}.
  
  \question[4] Calculer la masse volumique de cette eau.
  
  \question[2] En exploitant les données du tableau et le résultat de la question 2, indiquer comment la masse volumique évolue en fonction de la salinité.
  
  \question[2] Indiquer si la masse volumique d’une eau et sa salinité sont deux grandeurs proportionnelles. Justifier la réponse.


  \end{questions}


\end{document}
