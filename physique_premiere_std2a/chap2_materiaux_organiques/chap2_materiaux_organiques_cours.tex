\documentclass{article}
\usepackage{../../mypackages}
%\usepackage{../../macros}

% Variable de correction
\newif\ifWITHCORRECTION
\WITHCORRECTIONtrue % Mettre \WITHCORRECTIONfalse pour la version élève

% Commandes pour masquer du texte en fonction de la version
\newcommand{\corrige}[2]{\ifWITHCORRECTION #1 \else \underline{\hspace{#2}} \fi}

\title{Cours de Physique-Chimie : Les Matériaux Organiques}
\author{Professeur}
\date{}

\begin{document}

\maketitle

% Rappels de 2nde
\section{Rappels de 2nde}
\subsection{Couches électroniques et électrons de valence}
\begin{tcolorbox}[colback=green!10!white, colframe=green!75!black, title=Définition : ]
  \corrige{Les couches électroniques sont des niveaux d'énergie dans lesquels se répartissent les électrons autour du noyau. Les électrons de valence sont les électrons situés sur la couche externe.}{6cm}
\end{tcolorbox}

\subsection{Formation des ions et des molécules}
\begin{tcolorbox}[colback=blue!10!white, colframe=blue!75!black, title=Application]
  \corrige{L'ion sodium Na$^+$ se forme en perdant un électron, tandis que l'ion chlorure Cl$^-$ se forme en en gagnant un. Ensemble, ils forment une molécule de chlorure de sodium (NaCl).}{7cm}
\end{tcolorbox}

% Programme de 1ère
\section{Les Matériaux Organiques}

\subsection{Les chaînes carbonées}
\subsubsection{L'atome de carbone}
\begin{itemize}[noitemsep]
    \item L'atome de carbone est tétravalent.
    \item Les composés organiques contiennent du carbone.
    \item Chaînes saturées, insaturées, linéaires, ramifiées ou cycliques.
\end{itemize}

\subsubsection{Modélisation des molécules}
\begin{itemize}[noitemsep]
    \item \corrige{Formule brute : CH$_4$}{2cm}
    \item \corrige{Formule développée : H--C--H}{2cm}
\end{itemize}

\subsection{Hydrocarbures et Groupes Caractéristiques}
\subsubsection{Les alcanes}
\begin{tcolorbox}[colback=green!10!white, colframe=green!75!black, title=Définition : ]
  \corrige{Les alcanes sont des hydrocarbures saturés ne comportant que des liaisons simples entre les atomes de carbone.}{7cm}
\end{tcolorbox}

% Exemple de tableau
\subsubsection{Groupes caractéristiques}
\begin{tabular}{p{5cm}p{5cm}p{5cm}}
  \toprule
  Groupe fonctionnel & Exemple de molécule & Formule \\
  \midrule
  Alcool & \corrige{Éthanol}{2cm} & \ce{C2H5OH} \\
  Aldéhyde & \corrige{Formaldéhyde}{2cm} & \ce{CH2O} \\
  \bottomrule
\end{tabular}

\section{Les Polymères}
\subsection{Définition et propriétés}
\begin{tcolorbox}[colback=green!10!white, colframe=green!75!black, title=Définition : ]
  \corrige{Un polymère est une macromolécule formée par la répétition d'unités monomères.}{6cm}
\end{tcolorbox}

% Plastiques, élastomères, et fibres
\section{Plastiques, élastomères et fibres}
\subsection{Les plastiques}
\begin{itemize}[noitemsep]
    \item \corrige{Les plastiques sont des polymères synthétiques.}{4cm}
\end{itemize}

Hello test
\documentclass{article}
\usepackage{../../mypackages}
\usepackage{../../macros}

% Variable de correction
\newif\ifWITHCORRECTION
\WITHCORRECTIONtrue % Mettre \WITHCORRECTIONfalse pour la version élève

% Commandes pour masquer du texte en fonction de la version
\newcommand{\corrige}[2]{\ifWITHCORRECTION #1 \else \underline{\hspace{#2}} \fi}

\title{Cours de Physique-Chimie : Les Matériaux Organiques}
\author{Professeur}
\date{}

\begin{document}

\maketitle

% Rappels de 2nde
\section{Rappels de 2nde}
\subsection{Couches électroniques et électrons de valence}
\begin{tcolorbox}[colback=green!10!white, colframe=green!75!black, title=Définition : ]
  \corrige{Les couches électroniques sont des niveaux d'énergie dans lesquels se répartissent les électrons autour du noyau. Les électrons de valence sont les électrons situés sur la couche externe.}{6cm}
\end{tcolorbox}

\subsection{Formation des ions et des molécules}
\begin{tcolorbox}[colback=blue!10!white, colframe=blue!75!black, title=Application]
  \corrige{L'ion sodium Na$^+$ se forme en perdant un électron, tandis que l'ion chlorure Cl$^-$ se forme en en gagnant un. Ensemble, ils forment une molécule de chlorure de sodium (NaCl).}{7cm}
\end{tcolorbox}

% Programme de 1ère
\section{Les Matériaux Organiques}

\subsection{Les chaînes carbonées}
\subsubsection{L'atome de carbone}
\begin{itemize}[noitemsep]
    \item L'atome de carbone est tétravalent.
    \item Les composés organiques contiennent du carbone.
    \item Chaînes saturées, insaturées, linéaires, ramifiées ou cycliques.
\end{itemize}

\subsubsection{Modélisation des molécules}
\begin{itemize}[noitemsep]
    \item \corrige{Formule brute : CH$_4$}{2cm}
    \item \corrige{Formule développée : H--C--H}{2cm}
\end{itemize}

\subsection{Hydrocarbures et Groupes Caractéristiques}
\subsubsection{Les alcanes}
\begin{tcolorbox}[colback=green!10!white, colframe=green!75!black, title=Définition : ]
  \corrige{Les alcanes sont des hydrocarbures saturés ne comportant que des liaisons simples entre les atomes de carbone.}{7cm}
\end{tcolorbox}

% Exemple de tableau
\subsubsection{Groupes caractéristiques}
\begin{tabular}{p{5cm}p{5cm}p{5cm}}
  \toprule
  Groupe fonctionnel & Exemple de molécule & Formule \\
  \midrule
  Alcool & \corrige{Éthanol}{2cm} & \ce{C2H5OH} \\
  Aldéhyde & \corrige{Formaldéhyde}{2cm} & \ce{CH2O} \\
  \bottomrule
\end{tabular}

\section{Les Polymères}
\subsection{Définition et propriétés}
\begin{tcolorbox}[colback=green!10!white, colframe=green!75!black, title=Définition : ]
  \corrige{Un polymère est une macromolécule formée par la répétition d'unités monomères.}{6cm}
\end{tcolorbox}

% Plastiques, élastomères, et fibres
\section{Plastiques, élastomères et fibres}
\subsection{Les plastiques}
\begin{itemize}[noitemsep]
    \item \corrige{Les plastiques sont des polymères synthétiques.}{4cm}
\end{itemize}

Hello test

\begin{tcolorbox}[colback=green!10!white, colframe=green!75!black, title=Définitions : ]
  La famille des gaz nobles est la famille des éléments chimiques dont la couche de valence est \textbf{saturée} \par
  \vspace{1em}
  \begin{tabular}{lll}
    \toprule
    Numéro atomique & Element & Configuration électronique \\
    \midrule
    2 & Test & Test \\
    \bottomrule
  \end{tabular}

\end{tcolorbox}

\end{document}


\end{tcolorbox}

\end{document}
