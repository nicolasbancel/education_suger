\documentclass{article}
\usepackage{../../mypackages}

% Variable de correction
\newif\ifWITHCORRECTION
\WITHCORRECTIONtrue % Mettre \WITHCORRECTIONfalse pour la version élève

% Commandes pour masquer du texte en fonction de la version
\newcommand{\corrige}[2]{\ifWITHCORRECTION #1 \else \underline{\hspace{#2}} \fi}

\title{Cours de Physique-Chimie : Les Matériaux Organiques}
\author{Professeur}
\date{}

\begin{document}

\maketitle

% Rappels de 2nde
\section{Rappels de 2nde}
\subsection{Couches électroniques et électrons de valence}
\begin{tcolorbox}[colback=green!10!white, colframe=green!75!black, title=Définition : ]
  \corrige{Les couches électroniques sont des niveaux d'énergie dans lesquels se répartissent les électrons autour du noyau. Les électrons de valence sont les électrons situés sur la couche externe.}{6cm}
\end{tcolorbox}

\begin{tcolorbox}[colback=blue!10!white, colframe=blue!75!black, title=Application : Structure électronique]
  Exemple de l'atome d'oxygène (O) : \\
  Sa configuration électronique est : 1s$^2$ 2s$^2$ 2p$^4$. \\
  Cela signifie :
  \begin{itemize}[noitemsep]
    \item 2 électrons dans la première couche (1s$^2$)
    \item 2 électrons dans la sous-couche s de la deuxième couche (2s$^2$)
    \item 4 électrons dans la sous-couche p de la deuxième couche (2p$^4$)
  \end{itemize}
  Il possède donc 6 électrons de valence (2s$^2$ 2p$^4$).
\end{tcolorbox}

\begin{tcolorbox}[colback=blue!10!white, colframe=blue!75!black, title=Application : Autres configurations électroniques]
  \begin{itemize}[noitemsep]
    \item Hydrogène (H) : 1s$^1$
    \item Carbone (C) : 1s$^2$ 2s$^2$ 2p$^2$
    \item Sodium (Na) : 1s$^2$ 2s$^2$ 2p$^6$ 3s$^1$
  \end{itemize}
\end{tcolorbox}

\subsection{Formation des ions et des molécules}
\begin{tcolorbox}[colback=green!10!white, colframe=green!75!black, title=Définition : ]
  Un ion est un atome ou une molécule qui a gagné ou perdu des électrons, devenant ainsi chargé positivement ou négativement.
\end{tcolorbox}

\begin{tcolorbox}[colback=blue!10!white, colframe=blue!75!black, title=Exemple : Ions courants]
  \begin{itemize}[noitemsep]
    \item L'ion sodium (Na$^+$) : Perd un électron pour atteindre une configuration stable (1s$^2$ 2s$^2$ 2p$^6$).
    \item L'ion chlorure (Cl$^-$) : Gagne un électron pour compléter sa couche externe (1s$^2$ 2s$^2$ 2p$^6$ 3s$^2$ 3p$^6$).
  \end{itemize}
\end{tcolorbox}

\begin{tcolorbox}[colback=blue!10!white, colframe=blue!75!black, title=Application : Règle de l'octet et du duet]
  La règle de l'octet stipule que les atomes cherchent à obtenir une couche externe remplie de 8 électrons (ou 2 électrons pour les plus petits éléments comme l'hydrogène).
  \begin{itemize}[noitemsep]
    \item Exemple : Le fluor (F) gagne 1 électron pour atteindre 8 électrons sur sa dernière couche et forme un ion F$^-$.
    \item L'hydrogène (H) suit la règle du duet et cherche à obtenir 2 électrons sur sa couche externe en formant une liaison.
  \end{itemize}
\end{tcolorbox}

\subsection{Le tableau périodique de Mendeleïev}
\begin{tcolorbox}[colback=green!10!white, colframe=green!75!black, title=Définition : ]
  Le tableau périodique classe les éléments chimiques en fonction de leur numéro atomique et de leurs propriétés chimiques similaires.
\end{tcolorbox}

\begin{tcolorbox}[colback=blue!10!white, colframe=blue!75!black, title=Application : Position des éléments]
  \begin{itemize}[noitemsep]
    \item Les éléments dans une même colonne (groupe) ont des propriétés chimiques similaires et le même nombre d'électrons de valence.
    \item Exemple : Les éléments du groupe 1, comme le sodium (Na) et le potassium (K), ont un seul électron de valence.
  \end{itemize}
\end{tcolorbox}

% Programme de 1ère
\section{Les Matériaux Organiques}

\subsection{Les chaînes carbonées}
\subsubsection{L'atome de carbone}
\begin{itemize}[noitemsep]
    \item L'atome de carbone est tétravalent.
    \item Les composés organiques contiennent du carbone.
    \item Chaînes saturées, insaturées, linéaires, ramifiées ou cycliques.
\end{itemize}

\subsubsection{Modélisation des molécules}
\begin{itemize}[noitemsep]
    \item \corrige{Formule brute : CH$_4$}{2cm}
    \item \corrige{Formule développée : H--C--H}{2cm}
\end{itemize}

\subsection{Hydrocarbures et Groupes Caractéristiques}
\subsubsection{Les alcanes}
\begin{tcolorbox}[colback=green!10!white, colframe=green!75!black, title=Définition : ]
  \corrige{Les alcanes sont des hydrocarbures saturés ne comportant que des liaisons simples entre les atomes de carbone.}{7cm}
\end{tcolorbox}

% Exemple de tableau
\subsubsection{Groupes caractéristiques}
\begin{tabular}{p{5cm}p{5cm}p{5cm}}
  \toprule
  Groupe fonctionnel & Exemple de molécule & Formule \\
  \midrule
  Alcool & \corrige{Éthanol}{2cm} & \ce{C2H5OH} \\
  Aldéhyde & \corrige{Formaldéhyde}{2cm} & \ce{CH2O} \\
  \bottomrule
\end{tabular}

\section{Les Polymères}
\subsection{Définition et propriétés}
\begin{tcolorbox}[colback=green!10!white, colframe=green!75!black, title=Définition : ]
  \corrige{Un polymère est une macromolécule formée par la répétition d'unités monomères.}{6cm}
\end{tcolorbox}

% Plastiques, élastomères, et fibres
\section{Plastiques, élastomères et fibres}
\subsection{Les plastiques}
\begin{itemize}[noitemsep]
    \item \corrige{Les plastiques sont des polymères synthétiques.}{4cm}
\end{itemize}

\end{document}
