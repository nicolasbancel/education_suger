\documentclass{article}
\usepackage{../../mypackages}
\usepackage{../../macros}

% Variable de correction


% Variable de correction
% \newif\ifWITHCORRECTION
% \WITHCORRECTIONtrue % Mettre \WITHCORRECTIONfalse pour la version élève

% Commandes pour masquer du texte en fonction de la version
%\newcommand{\corrige}[2]{\ifWITHCORRECTION #1 \else \underline{\hspace{#2}} \fi}


\def\WITH_CORRECTION{YES}

\title{Chapitre 2 - Les Matériaux Organiques}
\author{N. Bancel}
\date{Septembre 2024}



\begin{document}

\maketitle

% Rappels de 2nde
\section{Rappels de 2nde}
\subsection{Couches électroniques et électrons de valence}
\begin{tcolorbox}[colback=green!10!white, colframe=green!75!black, title=Définitions : ]
  Les électrons se répartissent autour du noyau atomique selon des couches électroniques et des sous-couches. \par 
  \vspace{1em}
  Les \textbf{couches électroniques} sont notées \(n = 1, 2, 3\) etc \par 
  Chaque couche électronique est divisée en \textbf{sous-couches} qui sont notées par les lettres \(s\), \(p\), etc.
  \begin{itemize}[noitemsep]
    \item La sous-couche \(s\) peut contenir jusqu'à \trou{h}{10}{2 électrons.}
    \item La sous-couche \(p\) peut contenir jusqu'à \trou{h}{10}{6 électrons.}
  \end{itemize}

  Dans la configuration électronique à l'état fondamental d'un atome de numéro atomique \trou{h}{60}{inférieur ou égal à 18, les électrons \textit{ns} et \textit{np} associés à la plus grande valeur de \textit{n} sont appelés \textbf{électrons de valence}}
  
  Seuls les électrons de la couche externe (électrons de valence) participent aux liaisons entres atomes dans les molécules, ou à la formation d’ions. 
  
\end{tcolorbox}

\begin{tcolorbox}[colback=blue!10!white, colframe=blue!75!black, title=Application : Structure électronique]
  Exemple de l'atome d'oxygène (O) : Numéro atomique : \(Z = 8\) \\
  Sa configuration électronique est : \trou{h}{10}{\(1s^2 2s^2 2p^4\)}. \\
  Cela signifie :
  \begin{itemize}[noitemsep]
    \item \trou{h}{10}{2} électrons dans la première couche (1s$^2$)
    \item \trou{h}{10}{2} électrons dans la sous-couche s de la deuxième couche (2s$^2$)
    \item \trou{h}{10}{4} électrons dans la sous-couche p de la deuxième couche (2p$^4$)
  \end{itemize}
  Il possède donc \trou{h}{20}{6 électrons de valence (2s$^2$ 2p$^4$).}
\end{tcolorbox}


\end{document}
