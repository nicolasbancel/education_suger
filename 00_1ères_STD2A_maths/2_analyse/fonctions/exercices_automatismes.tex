\documentclass[a4paper,12pt]{article}
\usepackage{../../../mypackages}
\usepackage{../../../macros}


\usepackage{pgfplots}
    \pgfplotsset{
    compat=1.11,
  }

\setlength{\parindent}{0pt}


\begin{document}

\def\WITH_CORRECTION{NO}

\title{Chapitre 1 - Les fonctions}
\author{N. Bancel}
\date{Septembre 2024}
\maketitle

\section{Identification d'un polynôme de degré 2 et des coefficients a, b, et c}

Dans les fonctions ci-dessous, déterminer si ce sont des polynômes de degré 2. Et si oui, déterminer la valeur des coefficients a, b, et c

\subsection{Sans développement}

\begin{itemize}
  \item \(f(x) = 3 x^2 - 7x + 3\) : 
  \item \(g(x) = \frac{1}{2} x^2 - 5x + \frac{5}{3}\) : 
  \item \(h(x) = 4 - 2 x^2\) :
  \item \(j(x) = 5x - 3\) : 
  \item \(k(x) = 5x^4 - 3x^3 + 6x - 8\) : 
\end{itemize}

\subsection{Avec développement}

\begin{itemize}
  \item \(f(x) = (x - 4)(5 - 2x)\) : 
  \vspace{1em}
  \item \(g(x) = (x + 2)^2 - 3x\) : 
  \vspace{1em}
  \item \(h(x) = (x + 1)(x^2 - 3)\) :
  \vspace{1em}
  \item \(i(x) = 2x (1 - 2x) + 4x^2\) :
\end{itemize}

\section{Racines de polynômes de degré 2}

Faire les exercices du livre 

\begin{itemize}
  \item Exercice N°39 page 170
  \item Exercice N°42 page 170
  \item Exercice N°47 page 170
\end{itemize}

\end{document}
