\documentclass{article}

% Packages nécessaires
\usepackage{ifthen}  % Pour les conditions
\usepackage{calc}    % Pour calculer la largeur et la hauteur
\usepackage{amsmath} % Pour les formules mathématiques
\usepackage{amssymb} % Symboles supplémentaires mathématiques
\usepackage{float}   % Pour la gestion des figures
\usepackage{totcount} % Pour calculer la hauteur totale

% Booléen pour basculer entre la version professeur et élève
\newboolean{profversion}
\setboolean{profversion}{false}  % true = version professeur, false = version élève

% Définition de la commande \ans avec gestion de l'espace vertical uniquement si nécessaire
\newcommand{\anstest}[1]{%
    \ifthenelse{\boolean{profversion}}%
    {#1} % Version professeur : afficher la réponse
    {%
        % Calcul de la largeur du contenu
        \newlength{\ansContentWidth}%
        \settowidth{\ansContentWidth}{#1}%
        % Si c'est un environnement mathématique, on le traite différemment
        \ifmmode
            % Si en mode mathématique, calculer la hauteur totale
            \newlength{\ansContentHeight}%
            \settototalheight{\ansContentHeight}{#1}%
            \parbox[t][\ansContentHeight][c]{\ansContentWidth}{\underline{\hspace{\ansContentWidth}}}%
        \else
            % Si ce n'est que du texte, juste un underline de la bonne largeur
            \underline{\hspace{\ansContentWidth}}%
        \fi
    }
}

\title{Cours de Physique-Chimie : Les Matériaux Organiques}
\author{Professeur}
\date{}

\begin{document}

\section*{Exemple d'utilisation}

\ifthenelse{\boolean{profversion}}%
    {\textbf{Version professeur}}%
    {\textbf{Version élève}}

\subsection*{Question avec du texte simple}

Voici une question avec réponse :  
"La capitale de la France est \anstest{Paris}."

\bigskip

Un autre exemple avec une réponse plus longue :  
"Le nombre d'électrons dans la couche externe de l'atome d'oxygène est \anstest{6}."

\bigskip

\subsection*{Question avec une formule (et du contenu mathématique)}

Voici une question avec une formule mathématique :  
\anstest{%
\begin{figure}[H]
  \centering
  
  \[
  c = \frac{V_s}{V}
  \]
  \begin{tabular}{@{}>{$}l<{$}l@{}}
    c & \text{compacité ("taux réel d'occupation de l'espace")}\\
    V_s & \text{Volume occupé par les noyaux au sein du cube} \\
    V & \text{Volume du cube qui englobe les noyaux} \\
  \end{tabular}
\end{figure}
}

\bigskip

Et voici où se termine ma box pour écrire la formule.

\end{document}
