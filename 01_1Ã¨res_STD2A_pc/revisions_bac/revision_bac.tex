\documentclass[a4paper,12pt]{article}
\usepackage{../../mypackages}
\usepackage{../../macros}

\SetLabelAlign{myright}{\hss\llap{$#1$}}
\newlist{where}{description}{1}
\setlist[where]{labelwidth=2cm,labelsep=1em,
                leftmargin=!,align=myright,font=\normalfont}


% \definecolor{SectionColor}{HTML}{1A73E8}      % Blue
% \definecolor{SubsectionColor}{HTML}{EA4335}   % Red

\DeclareUnicodeCharacter{202F}{FIX ME!!!!}

% Apply colors to section titles
\sectionfont{\color{blue}}
\subsectionfont{\color{magenta}}

\title{Révisions du BAC - Physique-Chimie}
\author{N. Bancel}
\date{Mai 2025}


\begin{document}

\textbf{Collège Lycée Suger}
\hfill
\textbf{Physique-Chimie} \\

\textbf{Année 2024-2025 - 3ème trimestre}
\hfill
\textbf{1ères STD2A} \par

{\let\newpage\relax\maketitle}

%\begin{center}
%\textbf{\textcolor{red}{Infos importantes}} \\
%\end{center}

\begin{tcolorbox}[colback=blue!10!white, colframe=blue!75!black, title=Concepts importants à retenir]
  \begin{compactitem}
    \item Taux d'évolution d'une grandeur
    \item Pourcentages
    \item Fractions irréductibles
    \item Equations du 1er degré
  \end{compactitem}
\end{tcolorbox}

\section*{L'énergie / Les ondes électromagnétiques}

\begin{tcolorbox}[colback=blue!5!white, colframe=blue!75!black, title=Méthode, breakable]
  \textbf{Étapes à suivre pour relier énergie, fréquence et longueur d'onde d'un photon :}
  \begin{compactitem}
    \item \textbf{Relier énergie et fréquence :} Utiliser la relation $E = h \times \nu$ pour calculer la fréquence $\nu$ (en \si{\hertz}), si l'énergie $E$ est donnée (la constante de Planck $h$ est \textbf{toujours} donnée).
    \item \textbf{Déduire la longueur d'onde à partir de la fréquenc :} Utiliser $\lambda = \dfrac{c}{\nu}$ pour trouver la longueur d'onde $\lambda$ (en \si{\meter}).
    \item \textbf{Convertir la longueur d'onde en nanomètre :} $1\,\text{nm} = \SI{1e-9}{\meter}$.
    \item \textbf{Identifier le type de rayonnement :} Comparer la longueur d'onde obtenue avec les domaines connus :
    \begin{compactitem}
      \item Infrarouge : $\lambda > \SI{700}{\nano\meter}$
      \item Visible : $\SI{400}{\nano\meter} \leq \lambda \leq \SI{700}{\nano\meter}$
      \item Ultraviolet : $\lambda < \SI{400}{\nano\meter}$
    \end{compactitem}
  \end{compactitem}
  \end{tcolorbox}

  \textcolor{red}{Il est aussi important de connaître les types d'ondes électro magnétiques, et les longueurs d'ondes essentielles}

\begin{figure}[H]
  \centering
  \includegraphics[width=\linewidth]{img/03.png}
\end{figure}

\subsection*{Exemple d'exercice corrigé}

\begin{figure}[H]
  \centering
  \includegraphics[width=0.8\linewidth]{img/01.jpg}
  \caption*{\textit{Extrait de BAC STD2A 2019}}
\end{figure}

\begin{figure}[H]
  \centering
  \includegraphics[width=0.8\linewidth]{img/02.jpg}
\end{figure}


Donnée : $E = \SI{1.42e-19}{\joule}$
Constante de Planck : $h = \SI{6.63e-34}{\joule\second}$

\vspace{0.5em}

\textbf{1. Calcul de la fréquence :}
\[
\nu = \dfrac{E}{h} = \dfrac{\SI{1.42e-19}{\joule}}{\SI{6.63e-34}{\joule\second}} \approx \SI{2.14e14}{\hertz}
\]

\vspace{0.5em}

\textbf{2. Calcul de la longueur d'onde :}
\[
\lambda = \dfrac{c}{\nu} = \dfrac{\SI{3.00e8}{\meter\per\second}}{\SI{2.14e14}{\hertz}} \approx \SI{1.40e-6}{\meter}
\]

\[
\lambda = \SI{1.40e-6}{\meter} = \SI{1400}{\nano\meter}
\]

\vspace{0.5em}

\textbf{3. Conclusion :}

$\SI{1400}{\nano\meter} > \SI{700}{\nano\meter}$, donc il s'agit bien d'un rayonnement \textbf{infrarouge}.

\subsection*{Exercice à faire vous-même}

\vspace{1em}
\textit{Extrait du BAC 2019 - Polynésie}

\begin{figure}[H]
  \centering
  \includegraphics[width=0.8\linewidth]{img/04.jpg}
\end{figure}

\begin{figure}[H]
  \centering
  \fbox{\includegraphics[width=0.8\linewidth]{img/05.jpg}}
\end{figure}

\section*{Ondes électromagnétiques - Définitions à connaître par coeur}

\begin{tabular}{p{0.28\textwidth} p{0.68\textwidth}}
  \toprule
  \textbf{Terme} & \textbf{Définition} \\
  \midrule
  \textbf{Lumière (modèle ondulatoire)} & 
  La lumière peut être décrite comme une onde électromagnétique qui se propage dans le vide et engendre une variation locale du champ électrique et magnétique.
  Direction de propagation de l’énergie de cette onde : rayon lumineux. \\
  
  \midrule

  \textbf{Lumière (modèle particulaire)} &
  La lumière est également décrite comme un ensemble de particules appelées \textbf{photons}. Un rayon lumineux est alors la trajectoire de propagation de ces particules de masse nulle se déplaçant dans la vide à la vitesse de la lumière : $c$ \\

  \midrule
  
  \textbf{Photon} &
  Un photon est une particule de lumière, sans masse, sans charge, qui transporte une énergie proportionnelle à la fréquence de l’onde associée : $E = h \times \nu$. \\
  \midrule
  
  \textbf{Rayonnements invisibles utilisés en peinture} &
  Les rayonnements \textbf{infrarouges} (IR) et \textbf{ultraviolets} (UV) sont souvent utilisés pour révéler des couches sous-jacentes dans les œuvres picturales, comme les dessins préparatoires ou les restaurations anciennes. \\
  \midrule
  
  \textbf{Limites du spectre visible} &
  La lumière visible correspond aux longueurs d’onde comprises entre \SI{400}{\nano\meter} (violet) et \SI{750}{\nano\meter} (rouge). En dessous, on parle d’ultraviolet ; au-dessus, d’infrarouge. \\
  \bottomrule
  \end{tabular}

  \section*{Les métaux}


\begin{tabular}{p{0.28\textwidth} p{0.68\textwidth}}
  \toprule
  \textbf{Terme} & \textbf{Définition} \\
  \midrule
  \textbf{Composition du verre} & 
  Le verre est principalement composé de silice (dioxyde de silicium) \ce{SiO_2} \\
  \midrule

  \textbf{Solide amorphe} &
  Solide dont la structure d'atomes n'est pas régulière (contrairement à un solide cristallin) \\

  \midrule
  
  \textbf{Transition vitreuse} &
  Le phénomène de transition vitreuse consiste en un passage d'un état caoutchouteux à un état vitreux, solide (et inversement)

  \begin{figure}[H]
    \centering
    \includegraphics[width=0.4\linewidth]{img/06.jpg} 
  \end{figure} \\

  \midrule
  
  \textbf{Céramique} &
  Ensemble des matériaux inorganiques, non métalliques et qui nécessitent de hautes températures lors de leur fabrication. Structure généralement cristalline. \\
  \midrule
  
  \textbf{Propriétés des céramiques} &
  \begin{compactitem}
    \item Les céramiques sont généralement très rigide
    \item Elles ont une température de fusion très élevée, supérieure à 2000°C.
    \item Insensibles à la corrosion
    \item Bonne résistance à l'usure
    \item Inertes chimiquement, bons isolants
  \end{compactitem} \\
  \bottomrule
  \end{tabular}
  

\section*{Les métaux}


\begin{tabular}{p{0.28\textwidth} p{0.68\textwidth}}
  \toprule
  \textbf{Terme} & \textbf{Définition} \\
  \midrule
  \textbf{Métal pur} & 
  Métal composé d'un seul élément chimique. Exemple Fer \ce{Fe}, Or \ce{Au}, Argent \ce{Ag}. \\
  \midrule

  \textbf{Alliage} &
  Métal composé d'au moins deux éléments chimiques \\
  \midrule
  
  \textbf{Composition de l'acier} &
  Fer + Carbone \\

  \midrule

  \textbf{Composition du bronze} &
  Cuivre + Etain \\
  
  \bottomrule
  \end{tabular}


  \section*{Les autres types de matériaux}


\begin{tabular}{p{0.28\textwidth} p{0.68\textwidth}}
  \toprule
  \textbf{Terme} & \textbf{Définition} \\
  \midrule
  \textbf{Matériau organique} & 
  Matériau dont la molécule est composée d'un squelette carboné (constitué d'atomes de carbones liés les uns aux autres) \\
  \midrule

  \textbf{Matériau composite} &
  Combinaison d'au moins deux matériaux différents, \textbf{non miscibles}. Permet de combiner les avantages de chacun des composants. Il est composé de
  \begin{compactitem}
    \item \textbf{Un renfort} : squelette du matériau. Assure la solidité (exemples : fibres, cailloux dans béton)
    \item \textbf{Une matrice} : Enveloppe le renfort, assure la cohésion du matériau
  \end{compactitem}
  Exemple de matériau composite : le béton armé \\
  \midrule
  
  \textbf{Composition de l'acier} &
  Fer + Carbone \\

  \midrule

  \textbf{Composition du bronze} &
  Cuivre + Etain \\
  
  \bottomrule
  \end{tabular}
  
  \section*{La photographie}

 \subsection*{Formules essentielles}

\begin{tcolorbox}[colback=blue!5!white, colframe=blue!75!black,
                  title=Formules essentielles en optique photographique, breakable]
\begin{compactitem}
  \item \textbf{Relation de conjugaison (lentille mince)} :\\
        \[
        \frac{1}{\overline{OA'}} - \frac{1}{\overline{OA}} = \frac{1}{f'}
        \]\\
        \vspace{1em}
        avec \emph{distances algébriques} orientées vers la droite.
        \begin{compactitem}
          \item \(\overline{OA}\) : distance objet (positive si objet réel, négative si virtuel).
          \item \(\overline{OA'}\) : distance image (positive si réelle, négative si virtuelle).
          \item \(f'\) : distance focale (positive pour lentille convergente, négative pour divergente).
        \end{compactitem}

  \item \textbf{Grandissement (rapport linéaire)} :\\
        \[
        \gamma
           = \frac{\overline{A'B'}}{\overline{AB}}
           = \frac{\overline{OA'}}{\overline{OA}}
          \]\\
    \begin{compactitem}
    \item Si $\gamma<1$ l’image est réduite par rapport à la taille de l'objet
    \item Si $\gamma>1$ elle est agrandie.
    \end{compactitem}
  \item \textbf{Nombre d’ouverture} :\\
        \[
          N = \frac{f'}{D}
        \]\\
        \vspace{1em}
        avec \(D\) le diamètre du diaphragme, et f' est la distance focale.
        \begin{compactitem}
          \item Formule inverse : 
          \[
            D = \dfrac{f'}{N}
          \]
          \item Quand \(N\) \emph{augmente} \(\Rightarrow\) \(D\) \emph{diminue} \(\Rightarrow\) moins de lumière atteint le capteur.
          \item Quand \(N\) \emph{diminue} \(\Rightarrow\) \(D\) \emph{augmente} \(\Rightarrow\) plus de lumière atteint le capteur.
        \end{compactitem}

  \item \textbf{Flux lumineux (stops)} :\\
        Passer d’un nombre d’ouverture au suivant (par ex. \(N = 4 \rightarrow N = 5{,}6\)) divise par 2 la quantité de lumière reçue ; l’opération inverse la multiplie par 2. Un « stop » = un facteur 2 de flux lumineux.
\end{compactitem}
\end{tcolorbox}

\subsection*{Méthode – vérifier qu'un sujet tient sur le capteur}

\begin{tcolorbox}[colback=blue!10!white, colframe=blue!75!black, title=Méthode : le sujet tient‑il dans l’image ?, breakable]
\textbf{Étapes à suivre :}
\begin{compactitem}
\item Mesurer ou relever la distance objet‑lentille (OA) .
\item Calculer OA' via la relation de conjugaison.
\item Déterminer le grandissement : $\frac{OA'}{OA}$
\item Grâce à l'énoncé, on a la longueur / hauteur de l'objet (AB) (c'est la taille de l'objet photographié)
\item En déduire la hauteur (ou largeur) de l’image : (A'B') .
\item Comparer  aux dimensions utiles du capteur.
\item Si l’image est plus grande que le capteur, elle ne peut pas être intégralement capturée sur la photo 
\item Pour la faire rentrer : reculer l’appareil ou choisir une focale plus courte.
\end{compactitem}
\end{tcolorbox}

\subsection*{Définitions}

\begin{tabular}{p{0.35\textwidth} p{0.68\textwidth}}
\toprule
\textbf{Terme} & \textbf{Définition} \\
\midrule
\textbf{ISO} & Sensibilité du capteur ; un ISO élevé permet d’exposer avec moins de lumière mais accroît le bruit numérique. \\
\midrule
\textbf{Distance focale } & Distance entre le centre optique et le foyer image d’une lentille ou d’un objectif. Elle fixe l’angle de champ (courte focale : grand angle, longue focale : téléobjectif). \\
\midrule
\textbf{Diaphragme} & Orifice circulaire de l’objectif dont le diamètre  règle la quantité de lumière et la profondeur de champ ; relié à  par . \\
\midrule
\textbf{Profondeur de champ} & Intervalle de distances où les objets sont nets. Elle diminue quand on ouvre le diaphragme (petit ), allonge la focale ou rapproche le sujet. \\
\midrule
\textbf{Grandissement } & Rapport de la taille de l’image sur la taille de l’objet ; permet de vérifier qu’un sujet tient sur le capteur. \\
\midrule
\textbf{Capteur} & Surface photosensible (ex. : \SI{23.5}{\milli\metre} × \SI{15.6}{\milli\metre} en APS‑C) qui convertit la lumière en signal numérique. \\
\midrule
\textbf{Image} & Reproduction optique formée par l’objectif sur le capteur ; codée en pixels. \\
\midrule
\textbf{Pixel} & Plus petite unité d’une image ou d’un capteur ; stocke l’information de couleur et de luminosité. \\
\midrule
\textbf{Définition (de l'image ou du capteur)} & Nombre total de pixels : \(N_{\text{px}} = N_x \times N_y\), où \\
& \(N_x\) est le nombre de pixels horizontaux \\
& \(N_y\) le nombre de pixels verticaux ; \\
& Le résultat s’exprime généralement en mégapixels (Mpx). \\
\midrule
\textbf{Nombre d’ouverture \(N\)} & \(N = f'/D\) avec \(D\) diamètre du diaphragme. Si \(N\) augmente \(\Rightarrow\) \(D\) diminue \(\Rightarrow\) moins de lumière passe. \\
\midrule
\textbf{Temps de pose \(t\)} & Durée pendant laquelle l’obturateur est ouvert ; doubler \(t\) double la lumière reçue. \\
\midrule
\textbf{DPI (dots per inch)} & Résolution d’impression : nombre de points par pouce ; plus le DPI est élevé, plus l’impression est fine. \\
\bottomrule
\end{tabular}


\end{document}
