\documentclass[answers]{exam}
\usepackage{/Users/nicolasbancel/git/education_suger/mypackages}
\usepackage{/Users/nicolasbancel/git/education_suger/macros}
\usetikzlibrary{calc}

\SolutionEmphasis{\color{blue}}
\renewcommand{\solutiontitle}{\noindent}

\newcommand{\cmark}{\ding{51}}

%\usepackage{blindtext}

\renewcommand{\arraystretch}{1.5} % Augmente l'espacement vertical entre les lignes du tableau
\newcolumntype{C}{>{\centering\arraybackslash}m{2cm}}


\SetLabelAlign{myright}{\hss\llap{$#1$}}
\newlist{where}{description}{1}
\setlist[where]{labelwidth=2cm,labelsep=1em,
                        leftmargin=!,align=myright,font=\normalfont}

\setlength{\parindent}{0pt}

\title{Fiche d'exercices d'optique.}
\author{N. Bancel}
\date{Juin 2025}

\begin{document}


\textbf{Collège Lycée Suger}
\hfill
\textbf{Physique-Chimie} \\

\textbf{Année 2024-2025}
\hfill
\textbf{1ères STD2A} \par

{\let\newpage\relax\maketitle}
%\maketitle


\section*{Extrait BAC 2019 - Metropole - A FAIRE CHEZ VOUS}

\begin{figure}[H]
  \centering
  \fbox{\includegraphics[width=0.8\linewidth]{img/annale_2_optique_1.jpg}}
\end{figure}


\begin{figure}[H]
  \centering
  \fbox{\includegraphics[width=0.8\linewidth]{img/annale_2_optique_2.jpg}}
\end{figure}

\begin{figure}[H]
  \centering
  \fbox{\includegraphics[width=0.8\linewidth]{img/annale_2_optique_3.jpg}}
\end{figure}


\section*{Extrait BAC 2019 - Exercice corrigé}

\begin{figure}[H]
  \centering
  \fbox{\includegraphics[width=0.8\linewidth]{img/annale_optique_1.jpg}}
\end{figure}

\begin{figure}[H]
  \centering
  \fbox{\includegraphics[width=0.8\linewidth]{img/annale_optique_2.jpg}}
\end{figure}

\begin{figure}[H]
  \centering
  \fbox{\includegraphics[width=0.8\linewidth]{img/annale_optique_3.jpg}}
\end{figure}


\begin{figure}[H]
  \centering
  \fbox{\includegraphics[width=0.8\linewidth]{img/annale_optique_4.jpg}}
\end{figure}



\subsection*{PARTIE C – Le viaduc de Millau (7,5 pts)}

\begin{questions}

%----------------------------------------------------------------
\question[C.1] Indiquer à quelle caractéristique est liée la valeur « 100 ISO ».

  \begin{solution}
  La valeur \textbf{ISO} caractérise \textbf{la sensibilité du capteur à la lumière}.  
  Plus l’ISO est élevé, plus le capteur est sensible (il nécessite donc moins de lumière pour une exposition correcte).  
  \end{solution}


%----------------------------------------------------------------
\question[C.2] Nommer l’élément de l’objectif lié au nombre d’ouverture $N$.

  \begin{solution}
  Le nombre d’ouverture $N$ est directement lié au \textbf{diaphragme} de l’objectif, c’est-à-dire l’orifice circulaire qui laisse passer la lumière. La relation a retenir est : 
\[
  N = \frac{f}{D}
\]
  où $f$ est la distance focale de l’objectif et $D$ le diamètre de l’ouverture du diaphragme.

  Autrement dit, le diamètre du diaphragme est donné par la formule :
\[
  D = \frac{f}{N}
\]
  Ainsi, plus $N$ est grand, plus le diaphragme est petit et moins de lumière atteint le capteur.

  \end{solution}


%----------------------------------------------------------------
\subsection*{C.3 – Recherche d’une faible profondeur de champ}

\begin{questions}

\question[C.3.1] Définir la profondeur de champ en photographie.

  \begin{solution}
  La \textbf{profondeur de champ} est la région de l’espace, devant et derrière le plan de mise au point, dans laquelle les objets apparaissent nets ; au-delà de cette zone, ils deviennent flous.  
  \end{solution}

\question[C.3.2] Expliquer comment agir sur $N$ pour obtenir ce résultat.

  \begin{solution}
  Pour réduire la profondeur de champ et créer du flou d’arrière-plan, il faut \textbf{ouvrir le diaphragme, c'est-à-dire augmenter le diamètre du diaphragme. Ce qui correspond à prendre des valeurs de $N$ plus petites}. Pour rappel :
  
  \[
  D = \frac{f}{N}
  \]
  où 
  \begin{addmargin}[4em]{1em}
  \begin{compactitem}
    \item[$D$] : diamètre du diaphragme
    \item[$f$] : distance focale de l’objectif
    \item[$N$] : nombre d’ouverture
  \end{compactitem}
  \end{addmargin}

  \vspace{1em}

  Donc si on diminue la valeur de $N$ (passer par exemple de $N=11$ à $N=5.6$), $D$ augmente. \\
  Un plus grand diamètre d'ouverture entraîne une profondeur de champ plus faible, car la zone nette est plus réduite.

  \vspace{1em}

  \textit{Interprétation / Exemple} : Prenez l'équivalent avec votre pupille : quand vous voulez voir plus distinctement un ensemble de choses (lointaines), \textbf{vous froncez les yeux pour réduire la taille de votre pupille} (équivalent au dimaètre du diaphragme) et ainsi augmenter la profondeur de champ.
  \textbf{Lorsque vous avez les pupilles dilatées (diamètre plus grand)}, vous voyez moins distinctement les objets éloignés, les alentours sont floutés, et vous ne voyez qu'un ensemble réduit de choses (la profondeur de champ est réduite).
  
\end{solution}

\end{questions}


%----------------------------------------------------------------
\subsection*{C.4 – Nouvelle ouverture : $N=5{,}6$}

\begin{questions}

\question[C.4.1] Indiquer si l’image sera sur- ou sous-exposée si seul $N$ est modifié. Justifier.

  \begin{solution}
  \textbf{Raisonnement}  
  \begin{compactitem}
    \item Le passage de $N=11$ à $N=5.6$ correspond à \SI{-2}{IL} (deux “stops”) : on passe de $N=11$ à $N=8$ (1 stop : multiplication par 2 de la quantité de lumière), puis de $N=8$ à $N=5.6$ (1 autre stop : multiplication encore par 2 de la quantité de lumière). le flux lumineux est donc multiplié par $2^2=4$.
    \item Le temps de pose restant le même et égal à \SI[parse-numbers=false]{\tfrac{1}{250}}{\second}, quatre fois plus de lumière atteindra le capteur.
  \end{compactitem}

  \textbf{Conclusion} : l’image sera \textbf{surexposée}.  
  \end{solution}



\question[C.4.2] Déterminer le temps de pose nécessaire pour conserver une exposition correcte.

  \begin{solution}
  \textbf{1. Raisonnement (lettres)}  

  Pour compenser une ouverture $4$ fois plus lumineuse, il faut que le temps de pose soit $4$ fois plus court :
  \[
    t_{\text{nouveau}} = \frac{t_{\text{ancien}}}{4}
  \]

  \textbf{2. Conversion (SI)}  

  \[
    t_{\text{ancien}} = \SI[parse-numbers=false]{\tfrac{1}{250}}{\second}
  \]

  \textbf{3. Application numérique}  

  \[
    t_{\text{nouveau}} = \frac{\tfrac{1}{250}}{4} = \frac{1}{1000}
  \]

  \[
    t_{\text{nouveau}} = \SI[parse-numbers=false]{\tfrac{1}{1000}}{\second}
  \]

  \textbf{4. Conclusion} : parmi les vitesses disponibles (Document 5), le photographe choisira \SI[parse-numbers=false]{\tfrac{1}{1000}}{\second}.  
  \end{solution}

\end{questions}


%----------------------------------------------------------------
\subsection*{C.5 – Caractéristiques de l’image}

\begin{questions}

\question[C.5.1] Calculer la définition de l’image.

  \begin{solution}
  \textbf{1. Raisonnement}  
  La définition est le nombre total de pixels :
  \[
    N_{\text{px}} = N_x \times N_y
  \]

  \textbf{2. Conversion} : les dimensions sont déjà en pixels (\num{6016} px × \num{4000} px).

  \textbf{3. Application numérique}  

  \[
    N_{\text{px}} = 6016 \times 4000 = 24064000
  \]

  \[
    \boxed{N_{\text{px}} = \num{24,1}\,\text{Mpixels}}
  \]

  \textbf{4. Conclusion} : l’image contient environ \num{24} millions de pixels (24 Mpx).  
  \end{solution}



\question[C.5.2] Indiquer à quelle caractéristique est liée la valeur « 300 dpi ».

  \begin{solution}
  Les \emph{dots per inch} (dpi) expriment \textbf{la résolution d’impression} : c’est le nombre de points imprimés par pouce linéaire, ce qui détermine la finesse (et donc la taille physique maximale) d’un tirage papier à partir du fichier.  
  \end{solution}

\end{questions}


%----------------------------------------------------------------
\subsection*{C.6 – Plan rapproché du pylône}

Données :  
\begin{compactitem}
  \item Distance focale : $f' = \SI{200}{\milli\metre} = \SI{0.200}{\metre}$  
  \item Hauteur pylône : $h = \SI{87.0}{\metre}$  
  \item Distance objet-objectif : $OA = \SI{50.0}{\metre}$  
  \item Capteur : \SI{23.5}{\milli\metre} × \SI{15.6}{\milli\metre}
\end{compactitem}

\begin{questions}

\question[C.6.1] Réaliser un schéma.

  \begin{solution}
  \textit{(Schéma descriptif)}  
  Tracer :
  \begin{compactitem}
    \item l’axe optique horizontal ;
    \item la lentille mince centrée en $O$, de foyers $F$ (objet) et $F'$ (image) ;
    \item le pylône $AB$ (segment vertical à gauche à distance $OA$) ;
    \item les rayons :  
      \begin{compactitem}
        \item rayon passant par $O$ (non dévié) ;  
        \item rayon parallèle à l’axe passant par $F'$ après la lentille.  
        \item \textcolor{red}{Tout rayon parallèle à l'axe optique est dévié pour passer par un foyer. Et inversement tout rayon passant par un des foyers (image ou objet) ressort parallèle à l'axe optique}
      \end{compactitem}
    \item leur intersection définit $A'B'$ (image inversée du pylône) à la distance $OA'$.
  \end{compactitem}

  \begin{figure}[H]
  \centering
  \includegraphics[width=0.8\linewidth]{img/annale_optique_5.png}
  \caption{Schéma de la formation de l'image par une lentille mince.}
  \end{figure}

  \end{solution}



\question[C.6.2] Calculer $OA'$ (position de l’image).
\begin{solution}
\textbf{1. Relation de conjugaison (distances algébriques)}  
\[
  \frac{1}{\overline{OA'}} - \frac{1}{\overline{OA}} \;=\; \frac{1}{f'}
\]

\begin{addmargin}[4em]{1em}
\begin{compactitem}
  \item[\(\overline{OA}\)] : distance objet-lentille  
  \item[\(\overline{OA'}\)] : distance image-lentille (à déterminer)  
  \item[\(f'\)] : distance focale
\end{compactitem}
\end{addmargin}

\medskip
{\color{red}
\textbf{Distances algébriques} – l’axe optique est \emph{orienté vers la droite}.  
Toute distance mesurée à gauche de \(O\) est donc \emph{négative}, et toute distance
mesurée à droite est \emph{positive}. Dans notre cas, la longueur \(\overline{OA}\) devra donc prendre une valeur négative, car le pylône est à gauche de la lentille. Et la longueur \(\overline{OA'}\) sera positive, car l’image se forme à droite de la lentille, dans le sens de l’axe optique.  
}
\medskip

Pour plus de détails, voir la vidéo : \href{https://www.youtube.com/watch?v=rBXHw1L-6y0}{Relation de CONJUGAISON : calculer la distance focale | 1\textsuperscript{ère} | Physique}

\medskip
\textbf{2. Conversion (S.I.)}  
\[
\begin{aligned}
  \overline{OA} &= \SI{-50.0}{\meter}
                 = \SI{-5.00e4}{\milli\metre} \\[4pt]
  f'            &= \SI{200}{\milli\metre}
\end{aligned}
\]

\textbf{3. Application numérique (mm)}  

\begin{align*}
  \frac{1}{\overline{OA'}} 
    &= \frac{1}{f'} - \frac{1}{\overline{OA}} \\[4pt]
    &= \frac{1}{200} - \Bigl(-\frac{1}{5.00\times10^{4}}\Bigr) \\[4pt]
    &= 0.00500 - (-0.00002) \\[4pt]
    &= 0.00498 
\end{align*}

\vspace{0.6em}
On a donc :  
\[
  \boxed{\displaystyle \frac{1}{\overline{OA'}} = 0{,}00498}
\]

\vspace{0.6em}
Pour obtenir \(\overline{OA'}\), \textbf{on prend le réciproque} (on « inverse » la
fraction) : si \(\dfrac{1}{x}=a\), alors \(x=\dfrac{1}{a}\).

\[
  \overline{OA'} = \frac{1}{0.00498}
  \;\simeq\; \SI{2.01e2}{\milli\metre}.
  \]
\[
  \boxed{\overline{OA'} \approx \SI{201}{\milli\metre}\;}
\]

\textbf{4. Conclusion}  
L’image se forme \(\approx\!\SI{201}{\milli\metre}\) derrière la lentille,
légèrement au-delà du foyer (\(\SI{200}{\milli\metre}\)).


\end{solution}

\question[C.6.3] L’image du pylône apparaît-elle en entier ?

\begin{solution}
%------------------------------------------------------------
% Pourquoi calculer A'B' ?
%------------------------------------------------------------
\textbf{Principe préalable}

Pour que le pylône apparaisse \emph{entièrement} sur la photo, il faut que la 
hauteur de son image \(\overline{A'B'}\) n’excède pas la dimension utile du 
capteur (\SI{23.5}{\milli\metre} côté le plus long).  
La relation de \emph{grandissement}  
\[
  \gamma = \frac{\overline{A'B'}}{\overline{AB}}
          = -\frac{\overline{OA'}}{\overline{OA}}
\]
montre qu’il suffit de connaître 

\begin{addmargin}[4em]{1em}
\begin{compactitem}
  \item[\(\overline{OA}\)]  : distance objet-lentille (pylône à \SI{-50.0}{m} de la lentille) : \cmark C'est bon, on connait cette donnée, ell est donnée dans l'énoncé
  \item[\(\overline{OA'}\)] : distance image-lentille (\(\SI{201}{mm}\) : \cmark C'est bon, on connait cette donnée, on l'a calculée à la question précédente)  
  \item[\(\overline{AB}\)]  : hauteur réelle du pylône (\SI{87.0}{m}) : \cmark C'est bon, on connait cette donnée, ell est donnée dans l'énoncé
\end{compactitem}
\end{addmargin}

pour pouvoir déterminer la hauteur de l’image \(\overline{A'B'}\) du pylône.

%------------------------------------------------------------
% 1) Formule
%------------------------------------------------------------
\bigskip
\textbf{1. Formule du grandissement}

\[
  \gamma \;=\; \frac{\overline{A'B'}}{\overline{AB}}
  \;=\; \frac{\overline{OA'}}{\overline{OA}}.
\]

%------------------------------------------------------------
% 2) Données
%------------------------------------------------------------
\bigskip
\textbf{2. Données numériques}

\[
  \overline{AB} = \SI{87.0}{\meter} = \SI{8.70e4}{\milli\metre},\qquad
  \overline{OA} = -\SI{5.00e4}{\milli\metre},\qquad
  \overline{OA'} = \SI{2.01e2}{\milli\metre}.
\]

%------------------------------------------------------------
% 3) Application numérique
%------------------------------------------------------------
\bigskip
\textbf{3. Application numérique}

\begin{align*}
  |\gamma| &= \frac{\overline{OA'}}{|\overline{OA}|}
            = \frac{201}{50\,000}
            \;\simeq\; 0.00402, \\[6pt]
  |\overline{A'B'}| &= |\gamma|\,|\overline{AB}|
                     = 0.00402 \times 87\,000
                     \;\simeq\; \SI{3.50e2}{\milli\metre}.
\end{align*}

\[
  \boxed{\displaystyle \overline{A'B'} \approx \SI{350}{\milli\metre}}
\]

%------------------------------------------------------------
% 4) Interprétation
%------------------------------------------------------------
\bigskip
\textbf{4. Conclusion – cadrage sur le capteur}

La hauteur calculée de l’image (\(\approx\SI{350}{\milli\metre}\)) dépasse de 
loin la dimension maximale du capteur (\(\SI{23.5}{\milli\metre}\)).  
\textbf{Le pylône ne tient donc pas en entier dans le cadre} ; il faudra 
reculer l’appareil ou employer une focale plus courte pour l’intégralité de la 
scène.
\end{solution}

\end{questions}

\end{questions}


\end{document}
