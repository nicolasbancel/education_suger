\documentclass[answers]{exam}
\usepackage{../../mypackages}
\usepackage{../../macros}


\SolutionEmphasis{\color{blue}}
\renewcommand{\solutiontitle}{\noindent}


\title{BAC Blanc}
\author{N. Bancel}
\date{30 Avril 2025}

\begin{document}


\begin{figure}[H]
  \centering
  \includegraphics[width=0.4\linewidth]{img/bac/0.jpg}
\end{figure}

\vspace{1em}
\vspace{1em}

\textbf{BACCALAUREAT 3 - 1ères STD2A}

\textbf{Date de l'épreuve : } \textcolor{blue}{30 Avril 2025} \\
\textbf{Matière: } \textcolor{blue}{Physique-Chimie} \\
\textbf{Durée de l'épreuve : } \textcolor{blue}{2 heures} \\
\textbf{Classe : } \textcolor{blue}{1ère STD2A} \\
\textbf{Nom de l'enseignant : } \textcolor{blue}{Nicolas Bancel} \\
\textbf{L'usage de la calculatrice en mode examen est autorisée : } \textcolor{blue}{OUI} \\

\newpage

\textbf{Collège Lycée Suger}
\hfill
\textbf{Physique-Chimie} \\

\textbf{Année 2024-2025 - 3ème trimestre}
\hfill
\textbf{1ères STD2A} \par

{\let\newpage\relax\maketitle}

\begin{center}
\textbf{\textcolor{red}{Durée : 2 heures. La calculatrice en mode examen est autorisée}} \\
\textbf{\textcolor{red}{Une réponse donnée sans justification sera considérée comme fausse.}} \\
\textbf{Total sur 20 points }
\end{center}

\textit{Extrait et inspiré du BAC 2015}

\section*{Partie A : Le bleu et l'art (9.5 points)} 

\vspace{1em}

\begin{figure}[H]
  \centering
  \includegraphics[width=\linewidth]{img/bac/1.jpg}
\end{figure}


\begin{figure}[H]
  \centering
  \includegraphics[width=\linewidth]{img/bac/2.jpg}
\end{figure}

\begin{figure}[H]
  \centering
  \includegraphics[width=\linewidth]{img/bac/3.jpg}
\end{figure}

\subsection*{Gravure d'une plaque de cuivre et "bain bleu" (5 points)}

\begin{questions}

\question[1] La plaque de cuivre subit plusieurs traitements.
\begin{parts}
  \part[0.5] Quel est le rôle du vernis appliqué sur cette plaque ?
  \part[0.25] Quel est le rôle du bain dans l'acide nitrique (appelé "eau forte") ?
  \part[0.25] Quel est le rôle du bain à l'eau claire ?
\end{parts}


%Q1E1%
\question[4] Réaction du cuivre avec l'acide nitrique.
\begin{parts}
  \part[1] Pour expliquer la formation de la solution bleue, écrire la
          demi-équation électronique associée au couple
          \ce{Cu^{2+}(aq) / Cu(s)} traduisant la transformation subie
          par le métal cuivre.

  \part[0.5] Le métal cuivre est-il un oxydant ou un réducteur ? Justifier.

  \part[0.5] Les ions nitrate appartiennent au couple redox :
          \[
              \ce{NO3^- (aq) + 4 H+ (aq) + 3 e- \rightleftharpoons NO (g) + 2 H2O (l)} .
          \]
          Écrire ce couple oxydant/réducteur.
  \part[1] Justifier que l'équation du dessus est bien équilibrer, en masse et en charge.
  \part[0.5] Les ions nitrate sont-ils réduits ou oxydés lorsque la solutio se colore en bleu ? (On rappelle que les ions  \ce{Cu^{2+}} donnent la couleur bleue en solution aqueuse)
  \part[0.5] Écrire l'équation bilan de la réaction entre les ions \ce{NO3^- (aq)} et la plaque métallique de cuivre.
\end{parts}

\end{questions}


%Q2E1%
\subsection*{Analyses d'un pigment bleu (4.5 points)}

\begin{questions}

\question[0.5] Le bleu égyptien est-il un pigment naturel ou synthétique ? Justifier à l'aide du document.

%Q3E1%
\question[0.5] De nombreuses analyses ont été réalisées à partir de prélèvements de céramiques. Définir la famille des céramiques.

%Q4E1%
\question[0.5] Après analyse, le bleu égyptien a été qualifié de \emph{matériau composite}. Donner la définition de ce terme.

%Q5E1%
\question[0.5] L'analyse a également mis en évidence la présence de zones \emph{amorphes}. Préciser le sens du terme "amorphe".


%Q6E1%
\question[2.5] Analyse aux rayons X.
\begin{parts}
  \part[0.5] À quel type d'ondes appartiennent les rayons X ?
  \part[0.5] Les rayons X sont assimilés à un ensemble de corpuscules porteurs d'énergie. Comment appelle-t-on ces corpuscules ?
  \part[1.5] La valeur de l'énergie d'un rayonnement X est \SI{5,68e-19}{\joule}. Calculer sa fréquence.
\end{parts}

\end{questions}


%Q7E1%
\section*{Partie B : Le cube d'eau de Pékin (8 points)}

\begin{figure}[H]
  \centering
  \includegraphics[width=\linewidth]{img/bac/2_1.jpg}
\end{figure}

\begin{figure}[H]
  \centering
  \includegraphics[width=\linewidth]{img/bac/2_2.jpg}
\end{figure}

\begin{figure}[H]
  \centering
  \includegraphics[width=\linewidth]{img/bac/2_3.jpg}
\end{figure}

\begin{figure}[H]
  \centering
  \includegraphics[width=\linewidth]{img/bac/2_4.jpg}
\end{figure}

\subsection*{Le cube d'eau (1.5 points)}

\begin{questions}
  \question[0.5] Rappeler les deux principaux constituants de l'alliage formant la structure du cube.
  
%Q1E2%
\question[0.5] À quelle famille de matériaux appartient l'ETFE constituant la façade du cube ?
  
%Q2E2%
\question[0.5] Donner deux raisons justifiant l'emploi de l'ETFE plutôt que du verre minéral.
\end{questions}


%Q3E2%
\subsection*{Photographie du cube d'eau (6.5 points)}

Un touriste utilise l'APN (l'appareil Photo Numérique) décrit dans le document 6, équipé d'un objectif $18$–\SI{135}{mm} dont la focale est réglée à \SI{35}{mm}. Un soir, il place son appareil à \SI{300}{m} d'une des façades du cube.  

\begin{questions}

\question[1] Reproduire (hors échelle) le schéma ci-dessous puis construire l'image $A'B'$ de l'objet $AB$ (la façade) à l'aide de trois rayons particuliers.

\begin{figure}[H]
  \centering
  \includegraphics[width=\linewidth]{img/bac/5.jpg}
\end{figure}


%Q4E2%
\question[1] D'après l'énoncé, déterminer les longueurs $OA$ et $OF$. On se place dans un repère de centre O, où les valeurs des abscisses positives sont vers la droite (du côté de F'). Justifier comment on passe d'une écriture vectorielle de la relation de conjugaison :

\[
  \frac{1}{\overline{OA'}} \;-\; \frac{1}{\overline{OA}}
  \;=\; \frac{1}{\overline{OF'}} 
  \;=\; \frac{1}{f'}
  \;=\; \frac{1}{f}
\]

à une écriture utilisant des longueurs (toutes positives) :

\[
  \frac{1}{f} = \frac{1}{d_o} + \frac{1}{d_i}
\]

où $d_o$ est la distance entre l'objectif et l'objet, et $d_i$ est la distance entre l'objectif et l'image de l'objet.


%Q5E2%
\question[4] Image et focalisation
\begin{parts}
\part[1] En utilisant la formule simplifiée précédente, démontrer que : 
\[
d_i = \frac{f d_o}{d_o - f}
\]
Indice : il faut ramener des fractions au même dénominateur.

\part[1] En déduire à quelle distance de l'objectif l'image se forme. Cette longeur correspond donc à la longueur $OA'$
\part[1] En considérant que ce qui est photographié (AB) est la largeur du cube de Pékin, que vaut AB ? En déduire, grâce à la formule de l'agrandissement, la largeur de l'image du cube sur le capteur.
\part[0.5] Montrer alors que le cube apparaîtra entier sur la photo (c'est-à-dire qu'il est entièrement capté par le capteur).
\end{parts}


%Q6E2%
\question[1] Sans changer de position, le touriste, après avoir pris la photo 1 ci-dessous, réalise la photo 2. Quel changement de réglage a-t-il opéré ?

\begin{figure}[H]
  \centering
  \includegraphics[width=0.8\linewidth]{img/bac/6.jpg}
\end{figure}

\end{questions}


%Q7E2%
\section*{Partie C : Le grand bleu (3.5 points)}

La plongée subaquatique nécessite une bonne acuité visuelle, car la vue est le principal sens qui permet la communication entre plongeurs. Cependant, l'œil humain n'est pas adapté pour voir sous l'eau de façon nette et certains phénomènes comme la disparition des couleurs avec la profondeur sont amplifiés sous l'eau.

\begin{figure}[H]
  \centering
  \includegraphics[width=\linewidth]{img/bac/3_1.jpg}
\end{figure}

\begin{figure}[H]
  \centering
  \includegraphics[width=\linewidth]{img/bac/3_2.jpg}
\end{figure}

\subsection*{La vision et ses défauts (3.5 points)}

Dans l'air et dans l'eau, les grandeurs caractéristiques de l'oeil changent.

\begin{questions}
  \question[1] D'après les documents et vos connaissances, comment la distance focale f' d'un œil au repos évolue-t-elle lorsqu'on passe de l'air à l'eau ?
  
%Q1E3%
\question[1.5] La vision floue sous l'eau peut être comparée à un défaut de l'œil.
    \begin{parts}
      \part[1] D'après les documents et vos connaissances, sous l'eau, où se forme l'image d'un objet lointain pour un œil normal n'accommodant pas ?
      \part[0.5] La correction apportée par un masque de plongée correspond-elle à celle d'une lentille convergente ou divergente ?
    \end{parts}
\end{questions}


%Q2E3%
\end{document}
