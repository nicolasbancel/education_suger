\documentclass[answers]{exam}
\usepackage{../../mypackages}
\usepackage{../../macros}


\SolutionEmphasis{\color{blue}}
\renewcommand{\solutiontitle}{\noindent}


\title{BAC Blanc}
\author{N. Bancel}
\date{30 Avril 2025}

\begin{document}


\begin{figure}[H]
  \centering
  \includegraphics[width=0.4\linewidth]{img/bac/0.jpg}
\end{figure}

\vspace{1em}
\vspace{1em}

\textbf{BACCALAUREAT 3 - 1ères STD2A}

\textbf{Date de l'épreuve : } \textcolor{blue}{30 Avril 2025} \\
\textbf{Matière: } \textcolor{blue}{Physique-Chimie} \\
\textbf{Durée de l'épreuve : } \textcolor{blue}{2 heures} \\
\textbf{Classe : } \textcolor{blue}{1ère STD2A} \\
\textbf{Nom de l'enseignant : } \textcolor{blue}{Nicolas Bancel} \\
\textbf{L'usage de la calculatrice en mode examen est autorisée : } \textcolor{blue}{OUI} \\

\newpage

\textbf{Collège Lycée Suger}
\hfill
\textbf{Physique-Chimie} \\

\textbf{Année 2024-2025 - 3ème trimestre}
\hfill
\textbf{1ères STD2A} \par

{\let\newpage\relax\maketitle}

\begin{center}
\textbf{\textcolor{red}{Durée : 2 heures. La calculatrice en mode examen est autorisée}} \\
\textbf{\textcolor{red}{Une réponse donnée sans justification sera considérée comme fausse.}} \\
\textbf{Total sur 20 points }
\end{center}

\textit{Extrait et inspiré du BAC 2015}

\section*{Partie A : Le bleu et l'art (9.5 points)} 

\vspace{1em}

\begin{figure}[H]
  \centering
  \includegraphics[width=\linewidth]{img/bac/1.jpg}
\end{figure}


\begin{figure}[H]
  \centering
  \includegraphics[width=\linewidth]{img/bac/2.jpg}
\end{figure}

\begin{figure}[H]
  \centering
  \includegraphics[width=\linewidth]{img/bac/3.jpg}
\end{figure}

\subsection*{Gravure d'une plaque de cuivre et "bain bleu" (5 points)}

\begin{questions}

\question[1] La plaque de cuivre subit plusieurs traitements.
\begin{parts}
  \part[0.5] Quel est le rôle du vernis appliqué sur cette plaque ?
  \part[0.25] Quel est le rôle du bain dans l'acide nitrique (appelé "eau forte") ?
  \part[0.25] Quel est le rôle du bain à l'eau claire ?
\end{parts}



\begin{solution}
\subsection*{Partie (a) : Rôle du vernis appliqué sur la plaque}

Le vernis appliqué sur la plaque de cuivre sert de protection contre l'action de l'acide. Il recouvre la plaque entièrement, et seules les parties où le vernis a été retiré à l'aide d'un outil seront attaquées par l'acide, permettant ainsi de créer le motif souhaité.

\subsection*{Partie (b) : Rôle du bain dans l'acide nitrique (eau forte)}

L'acide nitrique ("eau forte") attaque les zones de la plaque de cuivre où le vernis a été retiré. Son rôle est de créer des creux dans ces zones spécifiques, creusant ainsi le métal pour définir le motif de la gravure.

\subsection*{Partie (c) : Rôle du bain à l'eau claire}

Le bain à l'eau claire est utilisé pour rincer la plaque après l'immersion dans l'acide nitrique. Cela sert à stopper l'action de l'acide en éliminant les résidus acides restants, afin de protéger le métal et éviter une corrosion non désirée.
\end{solution}

\question[4] Réaction du cuivre avec l'acide nitrique.
\begin{parts}
  \part[1] Pour expliquer la formation de la solution bleue, écrire la
          demi-équation électronique associée au couple
          \ce{Cu^{2+}(aq) / Cu(s)} traduisant la transformation subie
          par le métal cuivre.

  \part[0.5] Le métal cuivre est-il un oxydant ou un réducteur ? Justifier.

  \part[0.5] Les ions nitrate appartiennent au couple redox :
          \[
              \ce{NO3^- (aq) + 4 H+ (aq) + 3 e- \rightleftharpoons NO (g) + 2 H2O (l)} .
          \]
          Écrire ce couple oxydant/réducteur.
  \part[1] Justifier que l'équation du dessus est bien équilibrer, en masse et en charge.
  \part[0.5] Les ions nitrate sont-ils réduits ou oxydés lorsque la solutio se colore en bleu ? (On rappelle que les ions  \ce{Cu^{2+}} donnent la couleur bleue en solution aqueuse)
  \part[0.5] Écrire l'équation bilan de la réaction entre les ions \ce{NO3^- (aq)} et la plaque métallique de cuivre.
\end{parts}

\end{questions}



\begin{solution}

\subsection*{Partie (a) : Demi-équation électronique du cuivre}

La demi-équation électronique pour la transformation du cuivre métallique en ions cuivre(II) est :
\[
\ce{Cu(s) \rightarrow Cu^{2+}(aq) + 2e^-}
\]
où le cuivre solide (\ce{Cu(s)}) perd deux électrons pour former des ions cuivre(II) (\ce{Cu^{2+}}).

\subsection*{Partie (b) : Nature du métal cuivre}

Le cuivre est un réducteur, car il perd des électrons pour former des ions \ce{Cu^{2+}}. Dans une réaction redox, l'espèce qui perd des électrons est un réducteur.

\subsection*{Partie (c) : Couple oxydant/réducteur des ions nitrate}

Le couple redox donné est :
\[
\ce{NO3^- (aq) / NO (g)}
\]

\subsection*{Partie (d) : Vérification de l'équilibre de l'équation}

L'équation donnée est :
\[
\ce{NO3^- (aq) + 4 H+ (aq) + 3 e- \rightleftharpoons NO (g) + 2 H2O (l)}
\]
Vérification en termes de masse :
\begin{compactitem}
    \item Hydrogènes : 4 à gauche (\ce{4H+}) et 4 à droite (dans \ce{2H2O}),
    \item Oxygènes : 3 à gauche (\ce{NO3^-}) et 3 à droite (1 dans \ce{NO} et 2 dans \ce{2H2O}),
    \item Azote : 1 à gauche (\ce{NO3^-}) et 1 à droite (\ce{NO}).
\end{compactitem}

Vérification en termes de charge :
\begin{compactitem}
    \item Charges : 0 à gauche (\ce{NO3^-} \,(-1) + \ce{4H+} \,(+4) = +3, équilibré par \ce{3e-} \,(-3)), et 0 à droite (neutre).
\end{compactitem}

L'équation est donc bien équilibrée en masse et en charge.

\subsection*{Partie (e) : Nature des ions nitrate}

Les ions nitrate sont réduits, car ils gagnent des électrons (passant de \ce{NO3^-} à \ce{NO}).

\subsection*{Partie (f) : Équation bilan de la réaction}

L'équation bilan de la réaction entre les ions \ce{NO3^- (aq)} et le cuivre métallique est :
\[
\ce{3 Cu(s) + 2 NO3^- (aq) + 8 H+ (aq) \rightarrow 3 Cu^{2+}(aq) + 2 NO (g) + 4 H2O (l)}
\]

\end{solution}

\subsection*{Analyses d'un pigment bleu (4.5 points)}

\begin{questions}

\question[0.5] Le bleu égyptien est-il un pigment naturel ou synthétique ? Justifier à l'aide du document.


\begin{solution}
\subsection*{Nature du bleu égyptien}

Le bleu égyptien est un pigment synthétique. Selon le \emph{Document 2}, il est obtenu par la cuisson de différents matériaux, notamment de la cuprorivaïte (CaCuSi\(_4\)O\(_{10}\)), en présence d'espèces siliceuses (quartz et/ou tridymite) et de résidus de fabrication dans une phase silicatée. Ce processus de création en laboratoire indique que le pigment est synthétique, car il nécessite un assemblage et une transformation de composants spécifiques dans des conditions contrôlées, distincts d'une simple occurrence naturelle.
\end{solution}

\question[0.5] De nombreuses analyses ont été réalisées à partir de prélèvements de céramiques. Définir la famille des céramiques.


\begin{solution}
\subsection*{Définition de la famille des céramiques}

La céramique est un matériau inorganique, non métallique, obtenu par la chaleur. Elle se divise en plusieurs familles, notamment :

\begin{compactitem}
    \item \textbf{Céramiques traditionnelles} : elles incluent les produits en argile, comme la poterie, les briques, et les tuiles. Ces céramiques sont constituées principalement de minéraux argileux, de quartz, et de feldspaths.
    \item \textbf{Céramiques techniques} : aussi appelées céramiques avancées ou modernes, elles sont utilisées pour leurs propriétés mécaniques, thermiques et électriques exceptionnelles. Cela inclut des matériaux comme les nitrures de silicium et les oxydes d'aluminium.
\end{compactitem}

Les céramiques sont donc des matériaux polyvalents, trouvant des applications dans divers domaines allant de l'art et la construction à la technologie de pointe.
\end{solution}

\question[0.5] Après analyse, le bleu égyptien a été qualifié de \emph{matériau composite}. Donner la définition de ce terme.


\begin{solution}
\subsection*{Définition d'un matériau composite}

Un \emph{matériau composite} est un matériau constitué de deux ou plusieurs constituants distincts qui conservent leur identité et ne se dissolvent pas ou ne se fusionnent pas. Ces constituants sont généralement un matériau de renforcement, tel que des fibres (de verre, de carbone, etc.), et une matrice, qui maintient les fibres ensemble. L'objectif principal d'un matériaux composite est d'atteindre des propriétés supérieures, que ce soit en termes de résistance, de poids, ou d'autres caractéristiques physiques, par rapport à chacun des matériaux individuels.

Dans le cas du bleu égyptien, il s'agit d'un matériau composite parce qu'il associe plusieurs phases minérales, telles que la cuprorivaïte et des espèces siliceuses, pour obtenir les propriétés souhaitées du pigment.
\end{solution}

\question[0.5] L'analyse a également mis en évidence la présence de zones \emph{amorphes}. Préciser le sens du terme "amorphe".



\begin{solution}
\subsection*{Définition du terme "amorphe"}

Le terme \emph{amorphe} désigne un état dans lequel les matériaux ne présentent pas une structure cristalline organisée. Les zones amorphes sont caractérisées par un agencement désordonné des atomes, contrairement aux zones cristallines où les atomes sont ordonnés selon un motif répétitif. Ce désordre atomique se traduit souvent par des propriétés physiques différentes, telles que la transparence ou la facilité de moulage, par rapport aux formes cristallines du même matériau.
\end{solution}

\question[2.5] Analyse aux rayons X.
\begin{parts}
  \part[0.5] À quel type d'ondes appartiennent les rayons X ?
  \part[0.5] Les rayons X sont assimilés à un ensemble de corpuscules porteurs d'énergie. Comment appelle-t-on ces corpuscules ?
  \part[1.5] La valeur de l'énergie d'un rayonnement X est \SI{5,68e-19}{\joule}. Calculer sa fréquence.
\end{parts}

\end{questions}



\begin{solution}

\subsection*{Partie (a) : Type d'ondes des rayons X}

Les rayons X appartiennent à la famille des ondes électromagnétiques.

\subsection*{Partie (b) : Corpuscules des rayons X}

Les corpuscules porteurs d'énergie des rayons X sont appelés \emph{photons}.

\subsection*{Partie (c) : Calcul de la fréquence d'un rayonnement X}

Pour calculer la fréquence d'un rayonnement, on utilise la relation de Planck :
\[
E = h \times \nu
\]
où

\begin{addmargin}[4em]{1em}
    \begin{compactitem}
        \item [$E$] : représente l'énergie du rayonnement (\SI{5,68e-19}{\joule})
        \item [$h$] : représente la constante de Planck (\SI{6,63e-34}{\joule\second})
        \item [$\nu$] : représente la fréquence à déterminer
    \end{compactitem}
\end{addmargin}

\textbf{Raisonnement avec formules mathématiques :}

En réarrangeant la formule pour résoudre en fonction de la fréquence \(\nu\), nous obtenons :
\[
\nu = \frac{E}{h}
\]

\textbf{Application numérique :}

\[
\nu = \frac{5,68 \times 10^{-19}}{6,63 \times 10^{-34}} = 8,57 \times 10^{14}
\]

\textbf{Conclusion :}

La fréquence du rayonnement X est \(\nu = \SI{8,57e14}{\hertz}\).

\end{solution}

\section*{Partie B : Le cube d'eau de Pékin (8 points)}

\begin{figure}[H]
  \centering
  \includegraphics[width=\linewidth]{img/bac/2_1.jpg}
\end{figure}

\begin{figure}[H]
  \centering
  \includegraphics[width=\linewidth]{img/bac/2_2.jpg}
\end{figure}

\begin{figure}[H]
  \centering
  \includegraphics[width=\linewidth]{img/bac/2_3.jpg}
\end{figure}

\begin{figure}[H]
  \centering
  \includegraphics[width=\linewidth]{img/bac/2_4.jpg}
\end{figure}

\subsection*{Le cube d'eau (1.5 points)}

\begin{questions}
  \question[0.5] Rappeler les deux principaux constituants de l'alliage formant la structure du cube.
  

\begin{solution}
\begin{addmargin}[1em]{0em}
Les deux principaux constituants de l'alliage formant la structure du cube d'eau de Pékin sont :

\begin{compactitem}
\item L'acier, qui constitue la structure principale ressemblant à un grand nid.
\item L'ETFE (éthylène tétrafluoroéthylène), utilisé pour les bulles extérieures en tant que revêtement, qui enferment les sacs à air.
\end{compactitem}
\end{addmargin}
\end{solution}

\question[0.5] À quelle famille de matériaux appartient l'ETFE constituant la façade du cube ?
  

\begin{solution}
\begin{addmargin}[1em]{0em}
L'ETFE (éthylène tétrafluoroéthylène) appartient à la famille des polymères. Plus précisément, il s'agit d'un copolymère, ce qui signifie qu'il est composé de deux types différents de monomères : l'éthylène et le fluoréthylène. Les polymères sont des matériaux plastiques connus pour leur légèreté, leur transparence, et leur résistance, des caractéristiques qui sont essentielles pour les applications architecturales comme celles du cube d'eau de Pékin.
\end{addmargin}
\end{solution}

\question[0.5] Donner deux raisons justifiant l'emploi de l'ETFE plutôt que du verre minéral.
\end{questions}



\begin{solution}
\begin{addmargin}[1em]{0em}
L'emploi de l'ETFE plutôt que du verre minéral pour le revêtement du Cube d'eau de Pékin se justifie pour plusieurs raisons :

\begin{compactitem}
    \item \textbf{Légèreté et flexibilité :} L'ETFE est beaucoup moins dense que le verre, ce qui permet de réduire le poids total de la structure. En outre, sa flexibilité lui permet de bien résister à la déformation causée par les mouvements ou les forces environnementales comme le vent.
    \item \textbf{Transmission lumineuse :} L'ETFE laisse passer environ 90\% de la lumière visible, permettant ainsi une utilisation efficace de la lumière naturelle tout en diminuant la nécessité d'un éclairage artificiel. En plus, il absorbe une grande partie de la lumière infrarouge, aidant ainsi à réguler la température intérieure.
\end{compactitem}

Ces caractéristiques rendent l'ETFE non seulement fonctionnel mais aussi économique à long terme en réduisant les coûts de maintenance et d'énergie.
\end{addmargin}
\end{solution}

\subsection*{Photographie du cube d'eau (6.5 points)}

Un touriste utilise l'APN (l'appareil Photo Numérique) décrit dans le document 6, équipé d'un objectif $18$–\SI{135}{mm} dont la focale est réglée à \SI{35}{mm}. Un soir, il place son appareil à \SI{300}{m} d'une des façades du cube.  

\begin{questions}

\question[1] Reproduire (hors échelle) le schéma ci-dessous puis construire l'image $A'B'$ de l'objet $AB$ (la façade) à l'aide de trois rayons particuliers.

\begin{figure}[H]
  \centering
  \includegraphics[width=\linewidth]{img/bac/5.jpg}
\end{figure}



\begin{solution}
Pour reproduire le schéma demandé et construire l'image $A'B'$ de l'objet $AB$, nous devons suivre les étapes suivantes en utilisant trois rayons particuliers :

1. **Rayons particuliers à utiliser :**

   \begin{compactitem}
       \item \textbf{Le rayon parallèle à l'axe optique :} Tout rayon incident parallèle à l'axe optique, après réfraction, passe par le foyer image $F'$ de la lentille.
       \item \textbf{Le rayon passant par le centre optique :} Tout rayon passant par le centre optique de la lentille ne dévie pas.
       \item \textbf{Le rayon passant par le foyer objet $F$ :} Tout rayon incident passant par le foyer objet $F$, après réfraction, sort parallèle à l'axe optique.
   \end{compactitem}

2. **Construction de $A'B'$ :**

   \begin{compactitem}
       \item Dessinez le rayon $AB$ et prolonger ce rayon au-delà de la lentille.
       \item Tracez le premier rayon parallèle à l'axe optique vers la lentille, puis faites-le passer par le foyer image $F'$.
       \item Tracez le deuxième rayon passant par le centre optique; il continue en ligne droite.
       \item Tracez le troisième rayon passant par le foyer objet $F$, puis le faire sortir parallèle à l'axe optique.
   \end{compactitem}

3. **Intersection des rayons :**

   Là où les trois rayons réfractés se croisent est le point où se forme l'image $A'B'$ de l'objet $AB$.

En rejoignant le point d'intersection de ces rayons par une droite perpendiculaire à l'axe optique, vous obtenez la position et la taille de l'image $A'B'$.

En conclusion, cette méthode graphique avec rayons particuliers permet de localiser précisément et de construire l'image formée par une lentille convergente.

\end{solution}

\question[1] D'après l'énoncé, déterminer les longueurs $OA$ et $OF$. On se place dans un repère de centre O, où les valeurs des abscisses positives sont vers la droite (du côté de F'). Justifier comment on passe d'une écriture vectorielle de la relation de conjugaison :

\[
  \frac{1}{\overline{OA'}} \;-\; \frac{1}{\overline{OA}}
  \;=\; \frac{1}{\overline{OF'}} 
  \;=\; \frac{1}{f'}
  \;=\; \frac{1}{f}
\]

à une écriture utilisant des longueurs (toutes positives) :

\[
  \frac{1}{f} = \frac{1}{d_o} + \frac{1}{d_i}
\]

où $d_o$ est la distance entre l'objectif et l'objet, et $d_i$ est la distance entre l'objectif et l'image de l'objet.



\begin{solution}
Pour déterminer les longueurs $OA$ et $OF$, et justifier la transformation de la relation de conjugaison vectorielle à une forme utilisant des longueurs positives, suivons les étapes suivantes :

1. \textbf{Relation vectorielle :} 
   La relation de conjugaison donnée est :
   \[
   \frac{1}{\overline{OA'}} - \frac{1}{\overline{OA}} = \frac{1}{\overline{OF'}}
   \]
   où $\overline{OA'}$, $\overline{OA}$ et $\overline{OF'}$ sont des grandeurs algébriques, pouvant être négatives ou positives selon leur position par rapport à $O$.

2. \textbf{Passage aux longueurs :} 
   En optique, les distances sont souvent prises avec des longueurs positives afin de simplifier l'analyse :

   - Notez que $d_o = \lvert \overline{OA} \rvert$, la distance objet.
   - $d_i = \lvert \overline{OA'} \rvert$, la distance image.
   - $f = \lvert \overline{OF'} \rvert$, la distance focale.

   En considérant que les distances mesurées à droite de $O$ sont positives, et à gauche de $O$ sont négatives :
   - Si $OA$ est à gauche de $O$, $\overline{OA}$ est négatif : $d_o = -\overline{OA}$.
   - Si $OA'$ est à droite de $O$, $\overline{OA'}$ est positif : $d_i = \overline{OA'}$.

3. \textbf{Forme positive de la relation :} 
   Cette convention conduit à l'équation classique :
   \[
   \frac{1}{f} = \frac{1}{d_o} + \frac{1}{d_i}
   \]
   où :

   \begin{addmargin}[4em]{1em}
   \begin{compactitem}
       \item [$f$] : longueur focale
       \item [$d_o$] : distance entre l'objectif et l'objet
       \item [$d_i$] : distance entre l'objectif et l'image de l'objet
   \end{compactitem}
   \end{addmargin}

4. \textbf{Conclusion :} 
   En utilisant les longueurs positives, nous simplifions l'analyse tout en gardant la cohérence des signes grâce aux conventions optiques utilisées pour les grandeurs algébriques. Ainsi, l'écriture vectorielle est transformée en une forme pratique utilisant uniquement des longueurs absolues positives.
\end{solution}

\question[4] Image et focalisation
\begin{parts}
\part[1] En utilisant la formule simplifiée précédente, démontrer que : 
\[
d_i = \frac{f d_o}{d_o - f}
\]
Indice : il faut ramener des fractions au même dénominateur.

\part[1] En déduire à quelle distance de l'objectif l'image se forme. Cette longeur correspond donc à la longueur $OA'$
\part[1] En considérant que ce qui est photographié (AB) est la largeur du cube de Pékin, que vaut AB ? En déduire, grâce à la formule de l'agrandissement, la largeur de l'image du cube sur le capteur.
\part[0.5] Montrer alors que le cube apparaîtra entier sur la photo (c'est-à-dire qu'il est entièrement capté par le capteur).
\end{parts}



\begin{solution}

\begin{enumerate}[label=\textbf{(\alph*)}]
  \item  \textbf{Démonstration de la relation donnant la distance image $d_i$}

        La relation de conjugaison d’une lentille mince est  

        \[
          \frac{1}{f}=\frac{1}{d_o}+\frac{1}{d_i},
        \]

        où  
        $f$ est la distance focale,  
        $d_o$ la distance objet–lentille et  
        $d_i$ la distance lentille–image.  

        On isole $\dfrac{1}{d_i}$ puis on inverse :

        \[
        \frac{1}{d_i}= \frac{1}{f}-\frac{1}{d_o}
        \quad\Longrightarrow\quad
        d_i=\frac{1}{\dfrac{1}{f}-\dfrac{1}{d_o}}
        =\frac{1}{\dfrac{d_o-f}{fd_o}}
        =\boxed{\,\displaystyle d_i=\frac{f\,d_o}{d_o-f}\, }.
        \]

  \item  \textbf{Distance de mise au point (longueur $OA'$)}

        Les données expérimentales sont  
        \[
          f=35\;\text{mm},\qquad  
          d_o = 300\;\text{m}=3,0\times10^{5}\;\text{mm}.
        \]

        \[
          d_i=\frac{f\,d_o}{d_o-f}
               =\frac{35\times3,0\times10^{5}}{3,0\times10^{5}-35}
               \simeq 35{,}0\; \text{mm}.
        \]

        L’image se forme donc à environ \(\boxed{d_i\simeq 35\;\text{mm}}\) derrière l’objectif, pratiquement au plan focal (légèrement en-deçà de 0,05 mm).

  \item  \textbf{Largeur de l’image du cube sur le capteur}

        Le grandissement linéaire est  

        \[
          \gamma = \frac{d_i}{d_o}\simeq\frac{35{,}0}{3,0\times10^{5}}
          \approx 1,17\times10^{-4}.
        \]

        Pour la largeur réelle du cube  
        \(AB = 177\;\text{m}=1,77\times10^{5}\;\text{mm}\),

        \[
          A'B' = |\gamma|\;AB 
          \simeq 1,17\times10^{-4}\times1,77\times10^{5}
          \simeq \boxed{20,7\;\text{mm}}.
        \]

  \item  \textbf{Le cube est-il entièrement capté ?}

        Le capteur du boîtier (Doc. 6) mesure \(23{,}2\;\text{mm}\) de large.  
        Comme \(A'B' \simeq 20,7\;\text{mm} < 23,2\;\text{mm}\), la projection de toute la façade du cube tient dans la largeur utile du capteur.  

        \[
          \boxed{\text{Oui, le cube apparaîtra en entier sur la photographie.}}
        \]

\end{enumerate}


\end{solution}

\question[1] Sans changer de position, le touriste, après avoir pris la photo 1 ci-dessous, réalise la photo 2. Quel changement de réglage a-t-il opéré ?

\begin{figure}[H]
  \centering
  \includegraphics[width=0.8\linewidth]{img/bac/6.jpg}
\end{figure}

\end{questions}



\begin{solution}
Le changement de réglage que le touriste a opéré entre la photo 1 et la photo 2 est probablement une modification de la focale de l'appareil photo (zoom). 

\begin{addmargin}[1em]{0em}
\textbf{Explication :} 
\begin{compactitem}
    \item Sur la photo 1, il est probable que l'appareil soit réglé sur une focale plus large, permettant de capturer une plus grande portion de la scène, typiquement utilisée pour les paysages.
    \item Sur la photo 2, en revanche, la focale ajustée est plus longue, ce qui signifie que l'objectif a été "zoomé" pour agrandir une partie spécifique de la scène.
\end{compactitem}
Ce changement de focale modifie l'angle de vue ainsi que les proportions des objets capturés, permettant ainsi de se concentrer davantage sur certains détails du paysage ou sujet visé.
\end{addmargin}
\end{solution}

\section*{Partie C : Le grand bleu (3.5 points)}

La plongée subaquatique nécessite une bonne acuité visuelle, car la vue est le principal sens qui permet la communication entre plongeurs. Cependant, l'œil humain n'est pas adapté pour voir sous l'eau de façon nette et certains phénomènes comme la disparition des couleurs avec la profondeur sont amplifiés sous l'eau.

\begin{figure}[H]
  \centering
  \includegraphics[width=\linewidth]{img/bac/3_1.jpg}
\end{figure}

\begin{figure}[H]
  \centering
  \includegraphics[width=\linewidth]{img/bac/3_2.jpg}
\end{figure}

\subsection*{La vision et ses défauts (3.5 points)}

Dans l'air et dans l'eau, les grandeurs caractéristiques de l'oeil changent.

\begin{questions}
  \question[1] D'après les documents et vos connaissances, comment la distance focale f' d'un œil au repos évolue-t-elle lorsqu'on passe de l'air à l'eau ?
  

\begin{solution}
Lorsqu'on passe de l'air à l'eau, la distance focale $f'$ de l'œil au repos subit un changement. D'après les documents fournis et nos connaissances sur l'optique, on observe ce qui suit :

\begin{compactitem}
  \item Dans l'air, la lumière passe de l'air (indice de réfraction plus faible) à l'œil (indice de réfraction plus élevé), ce qui permet à l'œil de former une image nette sur la rétine (Document 7.a).
  \item Dans l'eau, l'indice de réfraction est proche de celui de l'œil, ce qui réduit la capacité de réfraction de la cornée. Par conséquent, l'image ne se forme plus sur la rétine mais au-delà, provoquant une vision floue (Document 7.b et Document 8).
  \item Mathématiquement, cela correspond à une augmentation de la distance focale $f'$ de l'œil. En effet, la réduction de la différence d'indice entre l'extérieur et l'intérieur de l'œil entraîne une diminution de la convergence de la lumière, augmentant ainsi $f'$.
\end{compactitem}

Conclusion: La distance focale $f'$ d'un œil au repos augmente lorsqu'on passe de l'air à l'eau, rendant la vision sous l'eau sans correction optique (comme un masque de plongée) moins nette.
\end{solution}

\question[1.5] La vision floue sous l'eau peut être comparée à un défaut de l'œil.
    \begin{parts}
      \part[1] D'après les documents et vos connaissances, sous l'eau, où se forme l'image d'un objet lointain pour un œil normal n'accommodant pas ?
      \part[0.5] La correction apportée par un masque de plongée correspond-elle à celle d'une lentille convergente ou divergente ?
    \end{parts}
\end{questions}



\begin{solution}

\subsection*{Partie 1 : Formation de l'image sous l'eau}

La vision sous-marine pour un œil normal, lorsqu’il n’accommode pas, est affectée par le fait que l’indice de réfraction de l’eau est proche de celui de l'œil. Par conséquent, la capacité de réfraction de la cornée diminue.

\begin{compactitem}
  \item Sous l'eau, l'image d'un objet lointain pour un œil normal non accommodé se forme en arrière de la rétine. Cela est dû à la réduction de la convergence de la lumière causée par la diminution de l’indice de réfraction entre l'eau et l'œil. Cela est corroboré par le Document 7.b qui montre que l'image se forme derrière l'œil, provoquant une vision floue.
\end{compactitem}

\subsection*{Partie 2 : Correction par un masque de plongée}

Un masque de plongée permet de rétablir la vision nette des objets sous l’eau. 

\begin{compactitem}
  \item La correction apportée par un masque de plongée est similaire à celle d'une lentille convergente. En effet, comme indiqué dans le Document 7.c, le masque rétablit la capacité de focalisation sur la rétine, en introduisant une interface air-œil qui remplace l'interface eau-œil, augmentant ainsi l'effet de convergence qui était diminué dans l'eau.
\end{compactitem}

\end{solution}

\end{document}
